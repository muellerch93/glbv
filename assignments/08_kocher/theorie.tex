\documentclass{article}

\usepackage{amsmath,amsthm,amssymb}
\usepackage{commath}
\usepackage{mathtools}
\usepackage{enumerate}
\usepackage{subcaption}
\usepackage{float}
\usepackage{tikz}
\usepackage[margin=1in]{geometry}
\usepackage{multicol}

\usetikzlibrary{positioning, shapes, automata, arrows}

\setlength{\parindent}{0pt}
\setlength{\parskip}{8pt}

\usepackage[utf8]{inputenc}
\begin{document}
\title{Assignment 8 \\ Advanced Algorithms \& Data Structures PS}%
\author{Christian Müller 1123410 \\ Daniel Kocher, 0926293}%
\maketitle

{\bfseries Aufgabe 16}%

Geben Sie f{\"u}r die Operationen {\"u}ber die Datenstruktur aus der Aufgabe 15
eine Potentialfunktion an und zeigen Sie mit Hilfe dieser Potentialfunktion, dass
die amortisierten Kosten einer Operation konstant sind.

\begin{minipage}{.35\textwidth}
\begin{table}[H]
  \centering
  \begin{tabular}{l|l}
    Op$_i$  & $c_i$       \tabularnewline
    \hline\hline
    1       & 2           \tabularnewline
    2       & 2 (teuer)   \tabularnewline
    3       & 3 (teuer)   \tabularnewline
    4       & 2           \tabularnewline
    5       & 5 (teuer)   \tabularnewline
    6       & 2           \tabularnewline
    7       & 2           \tabularnewline
    8       & 2           \tabularnewline
    9       & 9 (teuer)   \tabularnewline
    10      & 2           \tabularnewline
    11      & 2           \tabularnewline
    12      & 2           \tabularnewline
    13      & 2           \tabularnewline
    14      & 2           \tabularnewline
    15      & 2           \tabularnewline
    16      & 2           \tabularnewline
    17      & 17 (teuer)  \tabularnewline
    18      & 2           \tabularnewline
    \hline
  \end{tabular}
  \caption{Kosten f{\"u}r $1 \leq i \leq 18$}
  \label{tbl:costs}
\end{table}
\end{minipage}
\begin{minipage}{.55\textwidth}
\begin{center}
$c_i=\begin{cases} i$, falls $i=2^k+1$ mit $k \in \mathbb{N}  \\ 2$, sonst.$  \end{cases}$
\end{center}

Die Potentialfunktion sei folgendermaßen definiert: 
\begin{equation}
  \phi(D_0) = 0\qquad\phi(D_i)=2i-2^{\lfloor{ld(i-1)}\rfloor+1}+1
\end{equation}
Diese Potentialfunktion $\phi(D_i)$ entspricht genau dem Kredit nach der $i$-ten
Operation (f{\"u}r $i > 0$) wenn man die Bankkonto-Methode anwendet.

Einige Gleichungen, die wir im Zuge der Berechnung der amortisierten Kosten
benutzen werden:
\begin{equation}
  \label{eq:2-power-kpl1}
  2^{k+1} = 2(i-1)\text{, wenn }i=2^{k}+1
\end{equation}
\begin{equation}
  \label{eq:2-power-k}
  2^k=(i-1)\text{, wenn }i\neq2^{k}+1
\end{equation}
\begin{equation}
  \label{eq:ld-equality}
  \lfloor ld(i-1) \rfloor = \lfloor ld(i-2) \rfloor\text{, wenn }i\neq2^{k}+1
\end{equation}

Gleichung~\ref{eq:ld-equality} gilt aufgrund der Tatsache, dass es für jedes
beliebige $k$ zwischen $2^k + 1$ und $2^{k+1} + 1$ immer ein $i$ gibt, sodass
$i > 2^k + 1$ gilt. \newline
\end{minipage}

Für alle Operationen $i$ mit $i=2^k+1$ gilt also: \\
\begin{equation}
\phi(D_i) = 2i-2^{\lfloor{ld(i-1)}\rfloor+1}+1 = 2i-2^{k+1}+1
\end{equation}
Mit Gleichung~\ref{eq:2-power-kpl1} erh{\"a}lt man dann
\begin{equation}
\phi(D_i) = 2i - 2(i - 1) + 1 = 2i - 2i + 2 + 1 = 3
\end{equation}

Für alle Operationen $i$ mit $i\neq2^k+1$ gilt: \\
\begin{equation}
\phi(D_i)=2i-(i-1)+1=2i-i+2=i+2
\end{equation}
Überlegung:
Unser Potential steigt also zwischen den Operationen $i=2^{j-1}+1$ und $k=2^{j}+1$ stetig an, um dann nach der Operation $u=2^{j}$ seinen Maximalwert zu erreichen, welcher von der Operation $k=2^{j}+1$ aufgebraucht wird.

Für die amortisierten Kosten gilt: 
\begin{itemize}
  \item Für alle Operationen $i$ mit $i=2^k+1$:\\
    \begin{align*}
    \hat{c_i} = c_i + \phi(D_i) - \phi(D_{i-1}) \\
    = i + 2i - 2^{\lfloor ld(i - 1) \rfloor + 1} + 1 - \left( 2(i - 1) - 2^{\lfloor ld((i - 1) - 1) \rfloor + 1} + 1 \right) \\
    = 3i - 2^{k + 1} + 1 - \left( 2i - 2 - 2^k + 1 \right) \\
    \stackrel{\mathrm{Gl.~\ref{eq:2-power-kpl1}}}= 3i - 2(i - 1) + 1 - 2i + 2 + 2^k - 1 \\
    \stackrel{\mathrm{Gl.~\ref{eq:2-power-k}}}= i - 2i + 2 + 1 + 2 - 1 + (i - 1) \\
    = 3 \leq 3 \Rightarrow\text{ in }O(1)
    \end{align*}
  \item Für alle Operationen $i$ mit $i\neq2^k+1$:\\
    \begin{align*}
    \hat{c_i} = 2 + 2i - 2^{\lfloor ld(i - 1) \rfloor + 1} + 1 - \left( 2(i - 1) - 2^{\lfloor ld(i - 2) \rfloor + 1} + 1 \right) \\
    = 2 + 2i - 2^{\lfloor ld(i - 1) \rfloor + 1} + 1 - 2i + 2 + 2^{\lfloor ld(i - 2) \rfloor + 1} - 1 \\
    \stackrel{\mathrm{Gl.~\ref{eq:ld-equality}}}= 4 \leq 4 \Rightarrow\text{ in }O(1)
    \end{align*}
\end{itemize}

\begin{equation}
\Longrightarrow \hat{c_i} = \begin{cases}3\text{, wenn }i = 2^k + 1 \\ 4\text{, sonst}\end{cases}\in O(1)
\end{equation}

Der Vollst{\"a}ndigkeit halber sei noch kurz bewiesen, dass unsere
Potentialfunktion $\phi(D_i)$ nicht negativ ist:
\begin{equation}
  \phi(D_i) = 2i - 2^{\lfloor ld(i - 1) \rfloor + 1} + 1 \geq 2i - 2(i - 1) + 1 = 2i - 2i + 2 + 1 = 3 \geq 0
\end{equation}

\clearpage

{\bfseries Aufgabe 17}%

Sei $Q$ eine Binomial Queue, die anfangs genau einen Binomialbaum $B_1$ mit den
Schl{\"u}sseln $13$ und $21$ enth{\"a}lt. F{\"u}gen Sie die Schl{\"u}ssel $3$,
$7$, $15$, $18$ und $27$ in die Queue ein. Beachten Sie dabei, dass beim
Einf{\"u}gen eines Schl{\"u}ssels $k$ in eine Binomial Queue $Q$ zuerst eine
Binomial Queue $Q'$ mit einem Objekt (mit Schl{\"u}ssel $k$) erzeugt wird und
anschlie{\ss}end $Q$ und $Q'$ vereinigt werden. Geben Sie nach jedem Schritt die
resultierende Queue an (verwenden Sie dabei die Child-Sibling-Parent Darstellung).

\tikzstyle{elem}=[draw, circle, thick, fill=blue!20, minimum size=10mm]
\tikzstyle{pointer}=[->, >=stealth, thick]
\tikzstyle{pointer-sibling}=[pointer, dotted]
\tikzstyle{pointer-child}=[pointer, solid]
\tikzstyle{pointer-parent}=[pointer, dashed]

Im Folgenden werden die einzelnen Schritte dargestellt, wobei f{\"u}r die
Child-Sibling-Parent Darstellung die folgenden Pfeile verwendet werden:
\begin{figure}[H]
  \centering
  \begin{tikzpicture}
    \draw[pointer-child] (0, 0) -- ++(0.5, 0) node[right] {\ldots Child-Pointer};
    \draw[pointer-parent] (5, 0) -- ++(0.5, 0) node[right] {\ldots Parent-Pointer};
    \draw[pointer-sibling] (10, 0) -- ++(0.5, 0) node[right] {\ldots Sibling-Pointer};
  \end{tikzpicture}
\end{figure}

\begin{multicols}{2}
\begin{figure}[H]
  \centering
  \begin{tikzpicture}
    \node[elem] (e-13) {$13$};
    \node[elem, below = 1 of e-13] (e-21) {$21$}
      edge [pointer-parent, bend right] (e-13);
    \draw[pointer-child] (e-13.south) -- (e-21.north);
  \end{tikzpicture}
  \caption{Ausgangs-Queue $Q$}
\end{figure}
\columnbreak
\begin{figure}[H]
  \centering
  \begin{tikzpicture}
    \node[elem] (e-13) {$13$};
    \node[elem, below = 1 of e-13] (e-21) {$21$}
      edge [pointer-parent, bend right] (e-13);
    \draw[pointer-child] (e-13.south) -- (e-21.north);

    \node[elem, right = 1 of e-13] (e-3) {$3$};

    \node[below right = 1 of e-3] (meld) {$\Longrightarrow$};
    \node[above = 0 of meld] (meld-t) {$meld$};

    \node[elem, right = 2 of e-3] (e-3) {$3$};
    \node[elem, right = 1 of e-3] (e-13) {$13$};
    \node[elem, below = 1 of e-13] (e-21) {$21$}
      edge [pointer-parent, bend right] (e-13);
    \draw[pointer-child] (e-13.south) -- (e-21.north);
    \draw[pointer-sibling] (e-3.east) -- (e-13.west);
  \end{tikzpicture}
  \caption{Einf{\"u}gen von $3$ ($meld$: $B_0$ vor $B_1$)} 
\end{figure}
\end{multicols}

\begin{figure}[H]
  \centering
  \begin{tikzpicture}
    \node[elem] (e-3) {$3$};
    \node[elem, right = 1 of e-3] (e-13) {$13$};
    \node[elem, below = 1 of e-13] (e-21) {$21$}
      edge[pointer-parent, bend right] (e-13);
    \draw[pointer-child] (e-13.south) -- (e-21.north);
    \draw[pointer-sibling] (e-3.east) -- (e-13.west);
    
    \node[elem, right = 1 of e-13] (e-7) {$7$};
    
    \node[below right = 1 of e-7] (meld-1) {$\Longrightarrow$};
    \node[above = 0 of meld-1] (meld-1-t) {$meld$};

    \node[elem, right = 2 of e-7] (e-3) {$3$};
    \node[elem, below = 1 of e-3] (e-7) {$7$}
      edge[pointer-parent, bend right] (e-3);
    \draw[pointer-child] (e-3.south) -- (e-7.north);
    \node[elem, right = 1 of e-3] (e-13) {$13$};
    \node[elem, below = 1 of e-13] (e-21) {$21$}
      edge[pointer-parent, bend right] (e-13);
    \draw[pointer-child] (e-13.south) -- (e-21.north);
    \draw[pointer-sibling] (e-3.east) -- (e-13.west);

    \node[below right = 1 of e-13] (meld-2) {$\Longrightarrow$};
    %\node[above = 0 of meld-2] (meld-2-t) {$meld$};

    \node[elem, right = 4 of e-13] (e-3) {$3$};
    \node[elem, below = 1 of e-3] (e-7) {$7$}
      edge[pointer-parent, bend right] (e-3);
    \node[elem, left = 1 of e-7] (e-13) {$13$}
      edge[pointer-parent, bend left] (e-3);
    \draw[pointer-sibling] (e-13.east) -- (e-7.west);
    \node[elem, below = 1 of e-13] (e-21) {$21$}
      edge[pointer-parent, bend right] (e-13);
    \draw[pointer-child] (e-13.south) -- (e-21.north);
    \draw[pointer-child] (e-3.south west) -- (e-13.north east);
  \end{tikzpicture}
  \caption{Einf{\"u}gen von $7$ ($meld$ vereinigt zweimal: $B_0$ und $B_1$)} 
\end{figure}

\begin{figure}[H]
  \centering
  \begin{tikzpicture}
    \node[elem] (e-3) {$3$};
    \node[elem, below = 1 of e-3] (e-7) {$7$}
      edge[pointer-parent, bend right] (e-3);
    \node[elem, left = 1 of e-7] (e-13) {$13$}
      edge[pointer-parent, bend left] (e-3);
    \draw[pointer-sibling] (e-13.east) -- (e-7.west);
    \node[elem, below = 1 of e-13] (e-21) {$21$}
      edge[pointer-parent, bend right] (e-13);
    \draw[pointer-child] (e-13.south) -- (e-21.north);
    \draw[pointer-child] (e-3.south west) -- (e-13.north east);

    \node[elem, right = 1 of e-3] (e-15) {$15$};

    \node[right = 3 of e-7] (meld) {$\Longrightarrow$};
    \node[above = 0 of meld] (meld-t) {$meld$};
    
    \node[elem, right = 5 of e-3] (e-15) {$15$};
    \node[elem, right = 3 of e-15] (e-3) {$3$};
    \draw[pointer-sibling] (e-15.east) -- (e-3.west);
    \node[elem, below = 1 of e-3] (e-7) {$7$}
      edge[pointer-parent, bend right] (e-3);
    \node[elem, left = 1 of e-7] (e-13) {$13$}
      edge[pointer-parent, bend left] (e-3);
    \draw[pointer-sibling] (e-13.east) -- (e-7.west);
    \node[elem, below = 1 of e-13] (e-21) {$21$}
      edge[pointer-parent, bend right] (e-13);
    \draw[pointer-child] (e-13.south) -- (e-21.north);
    \draw[pointer-child] (e-3.south west) -- (e-13.north east);
  \end{tikzpicture}
  \caption{Einf{\"u}gen von $15$ ($meld$: $B_0$ vor $B_2$)} 
\end{figure}

\begin{figure}[H]
  \centering
  \begin{tikzpicture}
    \node[elem] (e-15) {$15$};
    \node[elem, right = 3 of e-15] (e-3) {$3$};
    \draw[pointer-sibling] (e-15.east) -- (e-3.west);
    \node[elem, below = 1 of e-3] (e-7) {$7$}
      edge[pointer-parent, bend right] (e-3);
    \node[elem, left = 1 of e-7] (e-13) {$13$}
      edge[pointer-parent, bend left] (e-3);
    \draw[pointer-sibling] (e-13.east) -- (e-7.west);
    \node[elem, below = 1 of e-13] (e-21) {$21$}
      edge[pointer-parent, bend right] (e-13);
    \draw[pointer-child] (e-13.south) -- (e-21.north);
    \draw[pointer-child] (e-3.south west) -- (e-13.north east);

    \node[elem, right = 1 of e-3] (e-18) {$18$};

    \node[right = 3 of e-7] (meld) {$\Longrightarrow$};
    \node[above = 0 of meld] (meld-t) {$meld$};

    \node[elem, right = 3 of e-18] (e-15) {$15$};
    \node[elem, below = 1 of e-15] (e-18) {$18$}
      edge[pointer-parent, bend right] (e-15);
    \draw[pointer-child] (e-15.south) -- (e-18.north);
    \node[elem, right = 3 of e-15] (e-3) {$3$};
    \draw[pointer-sibling] (e-15.east) -- (e-3.west);
    \node[elem, below = 1 of e-3] (e-7) {$7$}
      edge[pointer-parent, bend right] (e-3);
    \node[elem, left = 1 of e-7] (e-13) {$13$}
      edge[pointer-parent, bend left] (e-3);
    \draw[pointer-sibling] (e-13.east) -- (e-7.west);
    \node[elem, below = 1 of e-13] (e-21) {$21$}
      edge[pointer-parent, bend right] (e-13);
    \draw[pointer-child] (e-13.south) -- (e-21.north);
    \draw[pointer-child] (e-3.south west) -- (e-13.north east);

  \end{tikzpicture}
  \caption{Einf{\"u}gen von $18$ ($meld$ vereinigt einmal: $B_0$)} 
\end{figure}

\begin{figure}[H]
  \centering
  \begin{tikzpicture}
    \node[elem] (e-15) {$15$};
    \node[elem, below = 1 of e-15] (e-18) {$18$}
      edge[pointer-parent, bend right] (e-15);
    \draw[pointer-child] (e-15.south) -- (e-18.north);
    \node[elem, right = 3 of e-15] (e-3) {$3$};
    \draw[pointer-sibling] (e-15.east) -- (e-3.west);
    \node[elem, below = 1 of e-3] (e-7) {$7$}
      edge[pointer-parent, bend right] (e-3);
    \node[elem, left = 1 of e-7] (e-13) {$13$}
      edge[pointer-parent, bend left] (e-3);
    \draw[pointer-sibling] (e-13.east) -- (e-7.west);
    \node[elem, below = 1 of e-13] (e-21) {$21$}
      edge[pointer-parent, bend right] (e-13);
    \draw[pointer-child] (e-13.south) -- (e-21.north);
    \draw[pointer-child] (e-3.south west) -- (e-13.north east);

    \node[elem, right = 1 of e-3] (e-27) {$27$};

    \node[right = 3 of e-7] (meld) {$\Longrightarrow$};
    \node[above = 0 of meld] (meld-t) {$meld$};

    \node[elem, right = 3 of e-27] (e-27) {$27$};
    \node[elem, right = 1 of e-27] (e-15) {$15$};
    \draw[pointer-sibling] (e-27.east) -- (e-15.west);
    \node[elem, below = 1 of e-15] (e-18) {$18$}
      edge[pointer-parent, bend right] (e-15);
    \draw[pointer-child] (e-15.south) -- (e-18.north);
    \node[elem, right = 3 of e-15] (e-3) {$3$};
    \draw[pointer-sibling] (e-15.east) -- (e-3.west);
    \node[elem, below = 1 of e-3] (e-7) {$7$}
      edge[pointer-parent, bend right] (e-3);
    \node[elem, left = 1 of e-7] (e-13) {$13$}
      edge[pointer-parent, bend left] (e-3);
    \draw[pointer-sibling] (e-13.east) -- (e-7.west);
    \node[elem, below = 1 of e-13] (e-21) {$21$}
      edge[pointer-parent, bend right] (e-13);
    \draw[pointer-child] (e-13.south) -- (e-21.north);
    \draw[pointer-child] (e-3.south west) -- (e-13.north east);
  \end{tikzpicture}
  \caption{Einf{\"u}gen von $27$ ($meld$: $B_0$ vor $B_1$ vor $B_2$)} 
\end{figure}

\end{document}
