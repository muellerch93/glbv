\documentclass{article}
\usepackage{graphicx}
\usepackage[utf8]{inputenc}
\usepackage[linesnumbered]{algorithm2e}
\usepackage[margin=.75in]{geometry}
\newenvironment{algorithmic}{%
\renewenvironment{algocf}[1][h]{}{}% pass over the floating stuff
\algorithm
}{%
\endalgorithm
}
\begin{document}
%
\title{Assignment 1 \\ Advanced Algorithms \& Data Structures PS}%
\author{Christian Müller 1123410 \\ Daniel Kocher, 0926293}%
\maketitle%
\clearpage%
%


\begin{algorithm}
\begin{algorithmic}[1]
  \SetKwProg{Fn}{Function}{}{}%
  \SetKwInOut{Input}{Input}%
  \SetKwInOut{Output}{Output}%
  \SetKwInOut{Variables}{Variables}%
%
  \Input{
    Menge $S$ bestehend aus vertikalen Segmenten und Endpunkten von horizontalen Segmenten, welche bezüglich ihrer x-und y Koordinaten  paarweise verschieden sind. \\
    $S$ sei nach horizontalen Koordinaten sortiert (Algorithmus ist leicht zu adaptieren falls nicht).
  }%
  \Output{Alle Schnittpunkte von vertikalen Segmenten mit horizontalen Segmenten, von denen mindestens ein Endpunkt in $S$ ist}%
  \Variables{
    $L(S)$ enth{\"a}lt die y-Koordinaten aller linken Eckpunkte in $S$, deren rechter Partner nicht in $S$. \\
    $R(S)$ enth{\"a}lt die y-Koordinaten aller rechten Eckpunkte in $S$, deren linker Partner nicht in $S$. \\
    $V(S)$ enth{\"a}lt die y-Intervalle der vertikalen Segmente in $S$. \\
    $L(S)$ und $R(S)$ sind nach steigenden y-Koordinaten sortierte, verkettete Listen. \\
    $V(S)$ sind nach steigenden unteren Endpunkten sortierte, verkettete Listen.
  }%
  \Fn{ReportCuts($S$)}{%
    \If{$|S| \le 1$}{
      \tcp{Initialisierung von $L(S)$, $R(S)$ und $V(S)$}
      Sei $s = (x, y)\in S$\;
      \BlankLine%
      \lIf{$s$ ist linker Endpunkt}{$L(S) \leftarrow \{y\}$, $R(S) \leftarrow \emptyset$, $V(S) \leftarrow \emptyset$}%
      \BlankLine%
      \lElseIf{$s$ ist rechter Endpunkt}{$L(S) \leftarrow \emptyset$, $R(S) \leftarrow \{y\}$, $V(S) \leftarrow \emptyset$}% 
      \BlankLine%
      \lElseIf{$S$ enth{\"a}lt nur das vertikale Segmente $v$}{$L(S) \leftarrow \emptyset$, $R(S) \leftarrow \emptyset$, $V(S) \leftarrow \{[y_1, y_2]\}$}% 
      \BlankLine%
      \Return \tcp*{Rekursionsende}%
    }%
  
    \BlankLine\BlankLine%
    \tcp{Divide Schritt}%
    Teile $S$ per vertikaler Trennlinie in zwei (nahezu) gleich gro{\ss}e Teile $S_1$ und $S_2$ auf\;%
    \BlankLine\BlankLine%
    \tcp{Conquer Schritt}%
    ReportCuts($S_1$)\;%
    ReportCuts($S_2$)\;%
    \BlankLine\BlankLine%
    \tcp{$L(S_i)$, $R(S_i)$, $V(S_i)$ f{\"u}r $i = 1, 2$ bekannt $\Rightarrow$ Merge Schritt}
    \tcp{Berichte Segmentschnittpunkte (Paare $(h, v)$)}
    IntersectAndReport($R(S_2) \setminus L(S_1)$,$V(S_1)$)\;
    IntersectAndReport($L(S_1) \setminus R(S_2)$,$V(S_2)$)\;
    
    \BlankLine\BlankLine%
    \tcp{Aktualisiere $L(S)$, $R(S)$ und $V(S)$ f{\"u}r $S = S_1 \cup S_2$}
    $L(S) = \left( L(S_1) \setminus R(S_2) \right) \cup L(S_2)$\;
    $R(S) = \left( R(S_2) \setminus L(S_1) \right) \cup R(S_1)$\;
    $V(S) = V(S_1) \cup V(S_2)$\;
  }%


  \end{algorithmic}
\end{algorithm}
\begin{algorithm}                     
\begin{algorithmic} [1]  
\SetKwProg{Fn}{Function}{}{}%                 % enter the algorithmic environment
  \Fn{IntersectAndReport($H(S)$,$V(S)$)}{
  	\tcp{Sei $\dot{V}=(v_1,v_2,..,v_n)$ mit $v_i=(y_{unten},y_{oben})$ eine einfach verkettete aufsteigend sortierte (gemäß $y_{unten}$) Liste die alle Element aus $V(S)$ enthält.}
  	\tcp{Sei $\dot{H}=(h_1,h_2,..,h_n)$ mit $h_i=(x,y)$ eine einfach verkettete aufsteigend sortierte (gemäß $y$) Liste die alle Element aus $V(S)$ enthält.}
	$currHor \leftarrow h_1$\;
	$currVert \leftarrow v_1$\;
\While{$currVert$ $\neq$ $null$}{
	
	\eIf{intersects($currVert$,$currHor$)}{
		print($currHor$,$currVert$)\;
		$tmpHor \leftarrow currHor.next$\;
		\While{$tmpHor$ $\neq$ $null$ \textbf{and} intersects($currVert$,$tmpHor$)}{
			print($tmpHor$,$currVert$)\;
			$tmpHor \leftarrow tmpHor.next$\;
		}
		$currVert \leftarrow currVer.next$\;
	}{
		\If{$currHor.y < currVert.y_{unten}$}{
			$currHor \leftarrow currHor.next$\;
			}
	}
}
  
  }
\end{algorithmic}
\end{algorithm}


\end{document}
