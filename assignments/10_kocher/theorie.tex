\documentclass{article}

\usepackage{amsmath,amsthm,amssymb}
\usepackage{commath}
\usepackage{mathtools}
\usepackage{enumerate}
\usepackage{subcaption}
\usepackage{float}
\usepackage{tikz}
\usepackage[margin=1in]{geometry}
\usepackage{multicol}

\usetikzlibrary{positioning, shapes, automata, arrows, backgrounds, shapes.geometric}

\setlength{\parindent}{0pt}
\setlength{\parskip}{8pt}

\usepackage[utf8]{inputenc}
\begin{document}
\title{Assignment 11 \\ Advanced Algorithms \& Data Structures PS}%
\author{Christian Müller 1123410 \\ Daniel Kocher, 0926293}%
\maketitle

{\bfseries Aufgabe 20}%

Sei $Q$ ein anfangs leerer Fibonacci-Heap. Geben Sie eine Folge von Schl{\"u}sseln
sowie eine Folge von Operationen auf die von Ihnen gew{\"a}hlte Folge von
Schl{\"u}sseln an, sodass der resultierende Fibonacci-Heap aus zwei B{\"a}umen
$B_1$ und $B_2$ mit jeweils $\frac{n}{2}$ Knoten besteht, wobei
{\parskip0pt
  \begin{itemize}
    \item die Wurzel von $B_1$ mindestens Grad $4$ hat und
    \item $B_2$ ein entarteter Baum ist, in dem jeder innere Knoten genau einen
      Sohn besitzt.
  \end{itemize}
}
Sie d{\"u}rfen davon ausgehen, dass $\frac{n}{2}$ die Form $2^k$ mit $k \geq 3$
hat.

\textbf{Hinweis:} Konstruieren Sie zun{\"a}chst den Baum $B_1$ und
anschlie{\ss}end den Baum $B_2$, sodass w{\"a}hrend der Konstruktion von $B_2$
der Baum $B_1$ sich nicht mehr {\"a}ndert.

\textbf{Zur Konstruktion von Baum $B_1$:} \newline
Es reicht hier eine Folge von $17$ Aufrufen von $insert$ und ein
anschlie{\ss}ender Aufruf von $deletemin$. Die Schl{\"u}ssel werden dabei gro{\ss}
genug gew{\"a}hlt, sodass wir noch genug Spielraum f{\"u}r die Konstruktion von
Baum $B_2$ haben.

\textbf{Zur Konstruktion von Baum $B_2$:} \newline
Um einen entarteten Baum zu bekommen, ben{\"o}tigen wir die Operationen $insert$,
$deletemin$ sowie $decreasekey$. Zu Beginn f{\"u}gen wir $3$ Schl{\"u}ssel ein
mit anschlie{\ss}ender $deletemin$ Operation. Damit erhalten wir einen Baum mit
zwei linear verketteten Elementen. Per $insert$ f{\"u}gen wir jeweils drei Knoten
ein, wobei diese drei Knoten kleiner als das bisherige Minimum sind. Danach
f{\"u}hren wir eine $deletemin$ Operation durch. Dies hat zur Folge, dass im Zuge
der $consolidate$ Operation, die verbleibenden Knoten (ohne $B_1$)
zusammengef{\"u}hrt werden, da beide Rang $1$ besitzen. Im Anschluss wird die
Operation $decreasekey(i, j)$ ausgef{\"u}hrt, wobei $i =$ der gr{\"o}{\ss}te der
zuvor eingef{\"u}gten Schl{\"u}ssel, und $j = $ der kleinste der zuvor
eingef{\"u}gten Schl{\"u}ssel ist. Dadurch wird der Knoten von $B_2$ abgeschnitten
und als neues Minimum in den Fibonacci-Heap eingef{\"u}gt. Das anschlie{\ss}ende
$deletemin$ entfernt diesen Knoten dann und es verbleiben $B_1$ (unver{\"a}ndert)
und $B_2$ (entarteter Baum). Diese Vorgehensweise kann nun so oft wiederholt
werden bis sich $\frac{n}{2}$ linear verkettete Elemente in $B_2$ befinden. Alle
eingef{\"u}gten Schl{\"u}ssel m{\"u}ssen kleiner sein, als das Minimum des
aktuellen Fibonacci-Heaps. Damit wird (1) verhindert, dass wir w{\"a}hrend der
Konstruktion von $B_2$ den Baum $B_1$ ver{\"a}ndern (wie im Hinweis erw{\"a}hnt),
und (2) erzwungen, dass $B_2$ in eine linear verkettete Liste entartet.

\clearpage

Daraus ergibt sich folgende Operationsfolge (Schl{\"u}ssel werden in den
jeweiligen Operationen angegeben).

\textbf{Konstruktion von $B_1$:} \newline
$insert(50)$, $insert(51)$, $insert(52)$, $insert(53)$, $insert(54)$, \newline
$insert(55)$, $insert(56)$, $insert(57)$, $insert(58)$, $insert(59)$, \newline
$insert(60)$, $insert(61)$, $insert(62)$, $insert(63)$, $insert(64)$, \newline
$insert(65)$, $insert(66)$, $deletemin()$.

\textbf{Konstruktion von $B_2$:} \newline
$insert(40)$, $insert(39)$, $insert(38)$, $deletemin()$, \newline
$insert(36)$, $insert(37)$, $insert(38)$, $deletemin()$, $decreasekey(38, 36)$, $deletemin()$, \newline
$insert(34)$, $insert(35)$, $insert(36)$, $deletemin()$, $decreasekey(36, 34)$, $deletemin()$, \newline
$insert(32)$, $insert(33)$, $insert(34)$, $deletemin()$, $decreasekey(34, 32)$, $deletemin()$, \newline
$insert(30)$, $insert(31)$, $insert(32)$, $deletemin()$, $decreasekey(32, 30)$, $deletemin()$, \newline
$insert(28)$, $insert(29)$, $insert(30)$, $deletemin()$, $decreasekey(30, 28)$, $deletemin()$, \newline
$insert(26)$, $insert(27)$, $insert(28)$, $deletemin()$, $decreasekey(28, 26)$, $deletemin()$, \newline
$insert(24)$, $insert(25)$, $insert(26)$, $deletemin()$, $decreasekey(26, 24)$, $deletemin()$, \newline
$insert(22)$, $insert(23)$, $insert(24)$, $deletemin()$, $decreasekey(24, 22)$, $deletemin()$, \newline
$insert(20)$, $insert(21)$, $insert(22)$, $deletemin()$, $decreasekey(22, 20)$, $deletemin()$, \newline
$insert(18)$, $insert(19)$, $insert(20)$, $deletemin()$, $decreasekey(20, 18)$, $deletemin()$, \newline
$insert(16)$, $insert(17)$, $insert(18)$, $deletemin()$, $decreasekey(18, 16)$, $deletemin()$, \newline
$insert(14)$, $insert(15)$, $insert(16)$, $deletemin()$, $decreasekey(16, 14)$, $deletemin()$, \newline
$insert(12)$, $insert(13)$, $insert(14)$, $deletemin()$, $decreasekey(14, 12)$, $deletemin()$, \newline
$insert(10)$, $insert(11)$, $insert(12)$, $deletemin()$, $decreasekey(12, 10)$, $deletemin()$.

F{\"u}r das Einf{\"u}gen wird angenommen, dass wir immer rechts vom aktuellen
Minimum einf{\"u}gen. Dieselbe Annahme wurde bei \texttt{consolidate} getroffen,
d.h. es wird beim rechten Zwilling der zuvor entfernten Minimums begonnen.

\tikzstyle{elem}=[draw, circle, thick, fill=blue!20, minimum size=10mm]
\tikzstyle{elem-marked}=[elem, fill=red!20]
\tikzstyle{pointer}=[->, >=stealth, thick, solid]
\tikzstyle{pointer-child}=[pointer, black]
\tikzstyle{pointer-parent}=[pointer, red]
\tikzstyle{pointer-left}=[pointer, blue]
\tikzstyle{pointer-right}=[pointer, green!65!black]
\tikzstyle{pointer-min}=[pointer, <-, ultra thick]

Im Folgenden werden einige Schritte dargestellt, wobei f{\"u}r die
Child-Parent-Left-Right Darstellung die folgenden Pfeile und Knoten verwendet
werden:
\begin{figure}[H]
  \centering
  \begin{tikzpicture}
    \draw[pointer-child] (0, 0) -- ++(0.5, 0) node[right] (child) {\ldots Child-/Min-Pointer};
    \draw[pointer-parent] (5, 0) -- ++(0.5, 0) node[right] {\ldots Parent-Pointer};
    \draw[pointer-left] (9, 0) -- ++(0.5, 0) node[right] (left) {\ldots Left-Pointer};
    \draw[pointer-right] (13, 0) -- ++(0.5, 0) node[right] {\ldots Right-Pointer};
    \node[elem, below left = 0.5 and 0 of child] (normal-node) {$x$};
    \node[right = 0 of normal-node] {\ldots Normaler Knoten};
    \node[elem-marked, below  left = 0.5 and 0 of left] (marked-node) {$x$};
    \node[right = 0 of marked-node] {\ldots Markierter Knoten};
  \end{tikzpicture}
\end{figure}

\begin{figure}[H]
  \centering
  \begin{tikzpicture}[scale=0.65, every node/.style={scale=0.65}, trim left = -3cm]
    \node[elem] (e-51) {$51$};
    \node[elem, below left = 1.5 and 3 of e-51] (e-52) {$52$};
    \node[elem, right = 1 of e-52] (e-53) {$53$};
    \node[elem, below = 1.25 of e-53] (e-54) {$54$};
    \node[elem, right = 3 of e-53] (e-55) {$55$};
    \node[elem, below left = 1.5 and 1 of e-55] (e-56) {$56$};
    \node[elem, below right = 1.5 and 1 of e-55] (e-57) {$57$};
    \node[elem, below = 1 of e-57] (e-58) {$58$};
    \node[elem, right = 7 of e-55] (e-59) {$59$};
    \node[elem, below left = 1.5 and 3 of e-59] (e-60) {$60$};
    \node[elem, right = 1 of e-60] (e-61) {$61$};
    \node[elem, below = 1.25 of e-61] (e-62) {$62$};
    \node[elem, right = 3 of e-61] (e-63) {$63$};
    \node[elem, below left = 1.5 and 1 of e-63] (e-64) {$64$};
    \node[elem, below right = 1.5 and 1 of e-63] (e-65) {$65$};
    \node[elem, below = 1 of e-65] (e-66) {$66$};

    \draw[pointer-min] (e-51.north) -- ++(0, 0.5);
    \path[pointer-left] (e-51.west) edge[in=20, out=160, looseness=7] (e-51.east);
    \path[pointer-right] (e-51.east) edge[in=-160, out=-20, looseness=7] (e-51.west);
    \path[pointer-child] (e-51.west) edge[bend right] (e-52.north);
    \path[pointer-left] (e-52.west) edge[in=20, out=160, looseness=.75] (e-59.east);
    \path[pointer-right] (e-52.east) edge[bend right] (e-53.west);
    \path[pointer-parent] (e-52.north) edge[bend right] (e-51.west);
    \path[pointer-left] (e-53.west) edge[bend right] (e-52.east);
    \path[pointer-right] (e-53.east) edge[bend right] (e-55.west);
    \path[pointer-child] (e-53.south) edge[bend right] (e-54.north);
    \path[pointer-parent] (e-53.north east) edge[bend right] (e-51.south west);
    \path[pointer-left] (e-54.west) edge[in=20, out=160, looseness=7] (e-54.east);
    \path[pointer-right] (e-54.east) edge[in=-160, out=-20, looseness=7] (e-54.west);
    \path[pointer-parent] (e-54.north) edge[bend right] (e-53.south);
    \path[pointer-left] (e-55.west) edge[bend right] (e-53.east);
    \path[pointer-right] (e-55.east) edge[bend right=10] (e-59.west);
    \path[pointer-child] (e-55.south west) edge[bend right] (e-56.north);
    \path[pointer-parent] (e-55.north west) edge[bend left] (e-51.south);
    \path[pointer-left] (e-56.west) edge[in=20, out=160, looseness=2] (e-57.east);
    \path[pointer-right] (e-56.east) edge[bend right] (e-57.west);
    \path[pointer-parent] (e-56.north) edge[bend right] (e-55.south west);
    \path[pointer-left] (e-57.west) edge[bend right] (e-56.east);
    \path[pointer-right] (e-57.east) edge[in=-160, out=-20, looseness=2] (e-56.west);
    \path[pointer-child] (e-57.south) edge[bend right] (e-58.north);
    \path[pointer-parent] (e-57.north) edge[bend right] (e-55.south east);
    \path[pointer-left] (e-58.west) edge[in=20, out=160, looseness=7] (e-58.east);
    \path[pointer-right] (e-58.east) edge[in=-160, out=-20, looseness=7] (e-58.west);
    \path[pointer-parent] (e-58.north) edge[bend right] (e-57.south);
    \path[pointer-left] (e-59.west) edge[bend right=10] (e-55.east);
    \path[pointer-right] (e-59.east) edge[in=-160, out=-20, looseness=0.75] (e-52.west);
    \path[pointer-child] (e-59.south west) edge[bend right=20] (e-60.north);
    \path[pointer-parent] (e-59.north) edge[bend right] (e-51.east);
    \path[pointer-left] (e-60.west) edge[in=20, out=160, looseness=1.25] (e-63.east);
    \path[pointer-right] (e-60.east) edge[bend right] (e-61.west);
    \path[pointer-parent] (e-60.north) edge[bend right=20] (e-59.south west);
    \path[pointer-left] (e-61.west) edge[bend right] (e-60.east);
    \path[pointer-right] (e-61.east) edge[bend right] (e-63.west);
    \path[pointer-child] (e-61.south) edge[bend right] (e-62.north);
    \path[pointer-parent] (e-61.north east) edge[bend right] (e-59.south);
    \path[pointer-left] (e-62.west) edge[in=20, out=160, looseness=7] (e-62.east);
    \path[pointer-right] (e-62.east) edge[in=-160, out=-20, looseness=7] (e-62.west);
    \path[pointer-parent] (e-62.north) edge[bend right] (e-61.south);
    \path[pointer-left] (e-63.west) edge[bend right] (e-61.east);
    \path[pointer-right] (e-63.east) edge[in=-160, out=-20, looseness=1.25] (e-60.west);
    \path[pointer-child] (e-63.south west) edge[bend right] (e-64.north);
    \path[pointer-parent] (e-63.north) edge[bend right] (e-59.south east);
    \path[pointer-left] (e-64.west) edge[in=20, out=160, looseness=2] (e-65.east);
    \path[pointer-right] (e-64.east) edge[bend right] (e-65.west);
    \path[pointer-parent] (e-64.north) edge[bend right] (e-63.south west);
    \path[pointer-left] (e-65.west) edge[bend right] (e-64.east);
    \path[pointer-right] (e-65.east) edge[in=-160, out=-20, looseness=2] (e-64.west);
    \path[pointer-child] (e-65.south) edge[bend right] (e-66.north);
    \path[pointer-parent] (e-65.north west) edge[bend left] (e-63.south);
    \path[pointer-left] (e-66.west) edge[in=20, out=160, looseness=7] (e-66.east);
    \path[pointer-right] (e-66.east) edge[in=-160, out=-20, looseness=7] (e-66.west);
    \path[pointer-parent] (e-66.north) edge[bend right] (e-65.south);
  \end{tikzpicture}
  \caption{Baum $B_1$ mit $\frac{n}{2} = 16$ Knoten, wobei die Wurzel Grad $4$
    hat.
  }
\end{figure}

Konstruktion von $B_1$ ist abgeschlossen und die Bedingung aus der Angabe ist erf{\"u}llt.
$\Longrightarrow \frac{n}{2} = 16$. Das bedeutet, wird ben{\"o}tigen f{\"u}r $B_2$
einen entarteten Baum mit $16$ Elementen (d.h. eine linear verkettete Liste mit
$16$ Elementen).

Beispielhaft seien hier die ersten 3 Zeilen aus der Konstruktion von $B_2$
dargestellt. Der Baum $B_1$ wird nicht mehr explizit dargestellt, da er sich im
Zuge der Konstruktion von $B_2$ nicht ver{\"a}ndert.

\tikzstyle{tree}=[draw,shape border uses incircle, isosceles triangle,
  shape border rotate=90,yshift=-1.45cm, thick
]

\begin{figure}[H]
  \centering
  \begin{tikzpicture}[scale=0.65, every node/.style={scale=0.65}, trim left = -3cm]
    \node[elem] (e-51) {$51$};
    \node[tree, below = -0.95 of e-51] (b1) {$B_1$};

    \node[elem, right = 1.5 of e-51] (e-39) {$39$};
    \node[elem, below = 1 of e-39] (e-40) {$40$};

    \path[pointer-left] (e-51.west) edge[in=20, out=160, looseness=2] (e-39.east);
    \path[pointer-right] (e-51.east) edge[bend right] (e-39.west);
    \path[pointer-left] (e-39.west) edge[bend right] (e-51.east);
    \path[pointer-right] (e-39.east) edge[in=-160, out=-20, looseness=2] (e-51.west);
    \path[pointer-child] (e-39.south) edge[bend right] (e-40.north);
    \path[pointer-left] (e-40.west) edge[in=20, out=160, looseness=7] (e-40.east);
    \path[pointer-right] (e-40.east) edge[in=-160, out=-20, looseness=7] (e-40.west);
    \path[pointer-parent] (e-40.north) edge[bend right] (e-39.south);

    \draw[pointer-min] (e-39.north) -- ++(0, 0.5);
  \end{tikzpicture}
  \caption{Nach $insert(40)$, $insert(39)$, $insert(38)$, $deletemin()$.}
\end{figure}

\clearpage

\begin{figure}[H]
  \centering
  \begin{tikzpicture}[scale=0.65, every node/.style={scale=0.65}, trim left = 0]
    \node[elem] (e-51) {$51$};
    \node[tree, below = -0.95 of e-51] (b1) {$B_1$};

    \node[elem, right = 1.5 of e-51] (e-39) {$39$};
    \node[elem, below = 1 of e-39] (e-40) {$40$};
    \node[elem, right = 1 of e-39] (e-36) {$36$};
    \node[elem, right = 1 of e-36] (e-38) {$38$};
    \node[elem, right = 1 of e-38] (e-37) {$37$};

    \path[pointer-left] (e-51.west) edge[in=20, out=160, looseness=0.75] (e-37.east);
    \path[pointer-right] (e-51.east) edge[bend right] (e-39.west);
    \path[pointer-left] (e-39.west) edge[bend right] (e-51.east);
    \path[pointer-right] (e-39.east) edge[bend right] (e-36.west);
    \path[pointer-child] (e-39.south) edge[bend right] (e-40.north);
    \path[pointer-left] (e-40.west) edge[in=20, out=160, looseness=7] (e-40.east);
    \path[pointer-right] (e-40.east) edge[in=-160, out=-20, looseness=7] (e-40.west);
    \path[pointer-parent] (e-40.north) edge[bend right] (e-39.south);
    \path[pointer-left] (e-36.west) edge[bend right] (e-39.east);
    \path[pointer-right] (e-36.east) edge[bend right] (e-38.west);
    \path[pointer-left] (e-38.west) edge[bend right] (e-36.east);
    \path[pointer-right] (e-38.east) edge[bend right] (e-37.west);
    \path[pointer-left] (e-37.west) edge[bend right] (e-38.east);
    \path[pointer-right] (e-37.east) edge[in=-160, out=-20, looseness=0.75] (e-51.west);

    \draw[pointer-min] (e-36.north) -- ++(0, 0.5);

    \node[right = 1 of e-37] (lra) {$\Longrightarrow$};
    \node[above = 0 of lra] {$deletemin()$};

    \node[elem, right = 1.5 of lra] (e-51) {$51$};
    \node[tree, below = -0.95 of e-51] (b1) {$B_1$};

    \node[elem, right = 3 of e-51] (e-37) {$37$};
    \node[elem, below left = 1.5 and 1 of e-37] (e-38) {$38$};
    \node[elem, below right = 1.5 and 1 of e-37] (e-39) {$39$};
    \node[elem, below = 1 of e-39] (e-40) {$40$};

    \path[pointer-left] (e-51.west) edge[in=20, out=160, looseness=2] (e-37.east);
    \path[pointer-right] (e-51.east) edge[bend right] (e-37.west);
    \path[pointer-left] (e-37.west) edge[bend right] (e-51.east);
    \path[pointer-right] (e-37.east) edge[in=-160, out=-20, looseness=2] (e-51.west);
    \path[pointer-child] (e-37.south west) edge[bend right] (e-38.north);
    \path[pointer-left] (e-38.west) edge[in=20, out=160, looseness=2] (e-39.east);
    \path[pointer-right] (e-38.east) edge[bend right] (e-39.west);
    \path[pointer-parent] (e-38.north) edge[bend right] (e-37.south west);
    \path[pointer-left] (e-39.west) edge[bend right] (e-38.east);
    \path[pointer-right] (e-39.east) edge[in=-160, out=-20, looseness=2] (e-38.west);
    \path[pointer-child] (e-39.south) edge[bend right] (e-40.north);
    \path[pointer-parent] (e-39.north west) edge[bend left] (e-37.south);
    \path[pointer-left] (e-40.west) edge[in=20, out=160, looseness=7] (e-40.east);
    \path[pointer-right] (e-40.east) edge[in=-160, out=-20, looseness=7] (e-40.west);
    \path[pointer-parent] (e-40.north) edge[bend right] (e-39.south);

    \draw[pointer-min] (e-37.north) -- ++(0, 0.5);

    \node[below = 2 of e-38] (da) {$\Downarrow$};
    \node[left = 0 of da] {$decreasekey(38, 36)$};

    \node[elem, below = 5 of e-51] (e-51) {$51$};
    \node[tree, below = -0.95 of e-51] (b1) {$B_1$};

    \node[elem, right = 1 of e-51] (e-37) {$37$};
    \node[elem, right = 1 of e-37] (e-36) {$36$};
    \node[elem, below = 1 of e-37] (e-39) {$39$};
    \node[elem, below = 1 of e-39] (e-40) {$40$};

    \path[pointer-left] (e-51.west) edge[in=20, out=160, looseness=1.5] (e-36.east);
    \path[pointer-right] (e-51.east) edge[bend right] (e-37.west);
    \path[pointer-left] (e-37.west) edge[bend right] (e-51.east);
    \path[pointer-right] (e-37.east) edge[bend right] (e-36.west);
    \path[pointer-child] (e-37.south) edge[bend right] (e-39.north);
    \path[pointer-left] (e-36.west) edge[bend right] (e-37.east);
    \path[pointer-right] (e-36.east) edge[in=-160, out=-20, looseness=1.5] (e-51.west);
    \path[pointer-left] (e-39.west) edge[in=20, out=160, looseness=7] (e-39.east);
    \path[pointer-right] (e-39.east) edge[in=-160, out=-20, looseness=7] (e-39.west);
    \path[pointer-child] (e-39.south) edge[bend right] (e-40.north);
    \path[pointer-parent] (e-39.north) edge[bend right] (e-37.south);
    \path[pointer-left] (e-40.west) edge[in=20, out=160, looseness=7] (e-40.east);
    \path[pointer-right] (e-40.east) edge[in=-160, out=-20, looseness=7] (e-40.west);
    \path[pointer-parent] (e-40.north) edge[bend right] (e-39.south);

    \draw[pointer-min] (e-36.north) -- ++(0, 0.5);

    \node[left = 1 of e-51] (lla) {$\Longleftarrow$};
    \node[above = 0 of lla] {$deletemin()$};

    \node[elem, left = 3 of lla] (e-51) {$51$};
    \node[tree, below = -0.95 of e-51] (b1) {$B_1$};

    \node[elem, right = 1 of e-51] (e-37) {$37$};
    \node[elem, below = 1 of e-37] (e-39) {$39$};
    \node[elem, below = 1 of e-39] (e-40) {$40$};

    \path[pointer-left] (e-51.west) edge[in=20, out=160, looseness=2] (e-37.east);
    \path[pointer-right] (e-51.east) edge[bend right] (e-37.west);
    \path[pointer-left] (e-37.west) edge[bend right] (e-51.east);
    \path[pointer-right] (e-37.east) edge[in=-160, out=-20, looseness=2] (e-51.west);
    \path[pointer-child] (e-37.south) edge[bend right] (e-39.north);
    \path[pointer-left] (e-39.west) edge[in=20, out=160, looseness=7] (e-39.east);
    \path[pointer-right] (e-39.east) edge[in=-160, out=-20, looseness=7] (e-39.west);
    \path[pointer-child] (e-39.south) edge[bend right] (e-40.north);
    \path[pointer-parent] (e-39.north) edge[bend right] (e-37.south);
    \path[pointer-left] (e-40.west) edge[in=20, out=160, looseness=7] (e-40.east);
    \path[pointer-right] (e-40.east) edge[in=-160, out=-20, looseness=7] (e-40.west);
    \path[pointer-parent] (e-40.north) edge[bend right] (e-39.south);
  \end{tikzpicture}
  \caption{$insert(36)$, $insert(37)$, $insert(38)$, $deletemin()$,
    $decreasekey(38, 36)$, $deletemin()$.
  }
\end{figure}

\begin{figure}[H]
  \centering
  \begin{tikzpicture}[scale=0.65, every node/.style={scale=0.65}, trim left = 0]
    \node[elem] (e-51) {$51$};
    \node[tree, below = -0.95 of e-51] (b1) {$B_1$};

    \node[elem, right = 1 of e-51] (e-37) {$37$};
    \node[elem, right = 1 of e-37] (e-34) {$34$};
    \node[elem, right = 1 of e-34] (e-36) {$36$};
    \node[elem, right = 1 of e-36] (e-35) {$35$};
    \node[elem, below = 1 of e-37] (e-39) {$39$};
    \node[elem, below = 1 of e-39] (e-40) {$40$};

    \path[pointer-left] (e-51.west) edge[in=20, out=160, looseness=1] (e-35.east);
    \path[pointer-right] (e-51.east) edge[bend right] (e-37.west);
    \path[pointer-left] (e-37.west) edge[bend right] (e-51.east);
    \path[pointer-right] (e-37.east) edge[bend right] (e-34.west);
    \path[pointer-child] (e-37.south) edge[bend right] (e-39.north);
    \path[pointer-left] (e-34.west) edge[bend right] (e-37.east);
    \path[pointer-right] (e-34.east) edge[bend right] (e-36.west);
    \path[pointer-left] (e-36.west) edge[bend right] (e-34.east);
    \path[pointer-right] (e-36.east) edge[bend right] (e-35.west);
    \path[pointer-left] (e-35.west) edge[bend right] (e-36.east);
    \path[pointer-right] (e-35.east) edge[in=-160, out=-20, looseness=1] (e-51.west);
    \path[pointer-left] (e-39.west) edge[in=20, out=160, looseness=7] (e-39.east);
    \path[pointer-right] (e-39.east) edge[in=-160, out=-20, looseness=7] (e-39.west);
    \path[pointer-child] (e-39.south) edge[bend right] (e-40.north);
    \path[pointer-parent] (e-39.north) edge[bend right] (e-37.south);
    \path[pointer-left] (e-40.west) edge[in=20, out=160, looseness=7] (e-40.east);
    \path[pointer-right] (e-40.east) edge[in=-160, out=-20, looseness=7] (e-40.west);
    \path[pointer-parent] (e-40.north) edge[bend right] (e-39.south);

    \draw[pointer-min] (e-34.north) -- ++(0, 0.5);

    \node[right = 1 of e-35] (lra) {$\Longrightarrow$};
    \node[above = 0 of lra] {$deletemin()$};

    \node[elem, right = 1.5 of lra] (e-51) {$51$};
    \node[tree, below = -0.95 of e-51] (b1) {$B_1$};

    \node[elem, right = 3 of e-51] (e-35) {$35$};
    \node[elem, below left = 1.5 and 1 of e-35] (e-36) {$36$};
    \node[elem, below right = 1.5 and 1 of e-35] (e-37) {$37$};
    \node[elem, below = 1 of e-37] (e-39) {$39$};
    \node[elem, below = 1 of e-39] (e-40) {$40$};

    \path[pointer-left] (e-51.west) edge[in=20, out=160, looseness=2] (e-35.east);
    \path[pointer-right] (e-51.east) edge[bend right] (e-35.west);
    \path[pointer-left] (e-35.west) edge[bend right] (e-51.east);
    \path[pointer-right] (e-35.east) edge[in=-160, out=-20, looseness=2] (e-51.west);
    \path[pointer-child] (e-35.south west) edge[bend right] (e-36.north);
    \path[pointer-left] (e-36.west) edge[in=20, out=160, looseness=2] (e-37.east);
    \path[pointer-right] (e-36.east) edge[bend right] (e-37.west);
    \path[pointer-parent] (e-36.north) edge[bend right] (e-35.south west);
    \path[pointer-left] (e-37.west) edge[bend right] (e-36.east);
    \path[pointer-right] (e-37.east) edge[in=-160, out=-20, looseness=2] (e-36.west);
    \path[pointer-child] (e-37.south) edge[bend right] (e-39.north);
    \path[pointer-parent] (e-37.north west) edge[bend left] (e-35.south);
    \path[pointer-left] (e-39.west) edge[in=20, out=160, looseness=7] (e-39.east);
    \path[pointer-right] (e-39.east) edge[in=-160, out=-20, looseness=7] (e-39.west);
    \path[pointer-child] (e-39.south) edge[bend right] (e-40.north);
    \path[pointer-parent] (e-39.north) edge[bend right] (e-37.south);
    \path[pointer-left] (e-40.west) edge[in=20, out=160, looseness=7] (e-40.east);
    \path[pointer-right] (e-40.east) edge[in=-160, out=-20, looseness=7] (e-40.west);
    \path[pointer-parent] (e-40.north) edge[bend right] (e-39.south);

    \draw[pointer-min] (e-35.north) -- ++(0, 0.5);

    \node[below = 4 of e-36] (da) {$\Downarrow$};
    \node[left = 0 of da] {$decreasekey(36, 34)$};

    \node[elem, below = 7 of e-51] (e-51) {$51$};
    \node[tree, below = -0.95 of e-51] (b1) {$B_1$};

    \node[elem, right = 3 of e-51] (e-35) {$35$};
    \node[elem, right = 1 of e-35] (e-34) {$34$};
    \node[elem, below = 1 of e-35] (e-37) {$37$};
    \node[elem, below = 1 of e-37] (e-39) {$39$};
    \node[elem, below = 1 of e-39] (e-40) {$40$};

    \path[pointer-left] (e-51.west) edge[in=20, out=160, looseness=1] (e-34.east);
    \path[pointer-right] (e-51.east) edge[bend right] (e-35.west);
    \path[pointer-left] (e-35.west) edge[bend right] (e-51.east);
    \path[pointer-right] (e-35.east) edge[bend right] (e-34.west);
    \path[pointer-child] (e-35.south) edge[bend right] (e-37.north);
    \path[pointer-left] (e-37.west) edge[in=20, out=160, looseness=7] (e-37.east);
    \path[pointer-right] (e-37.east) edge[in=-160, out=-20, looseness=7] (e-37.west);
    \path[pointer-child] (e-37.south) edge[bend right] (e-39.north);
    \path[pointer-parent] (e-37.north) edge[bend right] (e-35.south);
    \path[pointer-left] (e-39.west) edge[in=20, out=160, looseness=7] (e-39.east);
    \path[pointer-right] (e-39.east) edge[in=-160, out=-20, looseness=7] (e-39.west);
    \path[pointer-child] (e-39.south) edge[bend right] (e-40.north);
    \path[pointer-parent] (e-39.north) edge[bend right] (e-37.south);
    \path[pointer-left] (e-40.west) edge[in=20, out=160, looseness=7] (e-40.east);
    \path[pointer-right] (e-40.east) edge[in=-160, out=-20, looseness=7] (e-40.west);
    \path[pointer-parent] (e-40.north) edge[bend right] (e-39.south);
    \path[pointer-left] (e-34.west) edge[bend right] (e-35.east);
    \path[pointer-right] (e-34.east) edge[in=-160, out=-20, looseness=1] (e-51.west);

    \draw[pointer-min] (e-34.north) -- ++(0, 0.5);

    \node[left = 1 of e-51] (lla) {$\Longleftarrow$};
    \node[above = 0 of lla] {$deletemin()$};

    \node[elem, left = 3 of lla] (e-51) {$51$};
    \node[tree, below = -0.95 of e-51] (b1) {$B_1$};

    \node[elem, right = 1 of e-51] (e-35) {$35$};
    \node[elem, below = 1 of e-35] (e-37) {$37$};
    \node[elem, below = 1 of e-37] (e-39) {$39$};
    \node[elem, below = 1 of e-39] (e-40) {$40$};

    \path[pointer-left] (e-51.west) edge[in=20, out=160, looseness=2] (e-35.east);
    \path[pointer-right] (e-51.east) edge[bend right] (e-35.west);
    \path[pointer-left] (e-35.west) edge[bend right] (e-51.east);
    \path[pointer-right] (e-35.east) edge[in=-160, out=-20, looseness=2] (e-51.west);
    \path[pointer-child] (e-35.south) edge[bend right] (e-37.north);
    \path[pointer-left] (e-37.west) edge[in=20, out=160, looseness=7] (e-37.east);
    \path[pointer-right] (e-37.east) edge[in=-160, out=-20, looseness=7] (e-37.west);
    \path[pointer-child] (e-37.south) edge[bend right] (e-39.north);
    \path[pointer-parent] (e-37.north) edge[bend right] (e-35.south);
    \path[pointer-left] (e-39.west) edge[in=20, out=160, looseness=7] (e-39.east);
    \path[pointer-right] (e-39.east) edge[in=-160, out=-20, looseness=7] (e-39.west);
    \path[pointer-child] (e-39.south) edge[bend right] (e-40.north);
    \path[pointer-parent] (e-39.north) edge[bend right] (e-37.south);
    \path[pointer-left] (e-40.west) edge[in=20, out=160, looseness=7] (e-40.east);
    \path[pointer-right] (e-40.east) edge[in=-160, out=-20, looseness=7] (e-40.west);
    \path[pointer-parent] (e-40.north) edge[bend right] (e-39.south);
  \end{tikzpicture}
  \caption{$insert(34)$, $insert(35)$, $insert(36)$, $deletemin()$,
    $decreasekey(36, 34)$, $deletemin()$.
  }
\end{figure}

Setzt man dies nun in dieser Weise fort, dann erh{\"a}lt man am Ende das
gew{\"u}nschte Ergebnis (mit $\frac{n}{2} = 16 = 2^4$):

\begin{figure}[H]
  \centering
  \begin{tikzpicture}[scale=0.65, every node/.style={scale=0.65}, trim left = 0]
    \node[elem] (e-51) {$51$};
    \node[tree, below = -0.95 of e-51] (b1) {$B_1$};

    \node[elem, right = 1 of e-51] (e-11) {$11$};
    \node[elem, below = 1 of e-11] (e-13) {$13$};
    \node[elem, below = 1 of e-13] (e-15) {$15$};

    \node[below = 1 of e-15] (e-ldots) {\ldots};

    \node[elem, below = 1 of e-ldots] (e-37) {$37$};
    \node[elem, below = 1 of e-37] (e-39) {$39$};
    \node[elem, below = 1 of e-39] (e-40) {$40$};

    \path[pointer-left] (e-51.west) edge[in=20, out=160, looseness=2] (e-11.east);
    \path[pointer-right] (e-51.east) edge[bend right] (e-11.west);
    \path[pointer-left] (e-11.west) edge[bend right] (e-51.east);
    \path[pointer-right] (e-11.east) edge[in=-160, out=-20, looseness=2] (e-51.west);
    \path[pointer-child] (e-11.south) edge[bend right] (e-13.north);
    \path[pointer-left] (e-13.west) edge[in=20, out=160, looseness=7] (e-13.east);
    \path[pointer-right] (e-13.east) edge[in=-160, out=-20, looseness=7] (e-13.west);
    \path[pointer-child] (e-13.south) edge[bend right] (e-15.north);
    \path[pointer-parent] (e-13.north) edge[bend right] (e-11.south);
    \path[pointer-left] (e-15.west) edge[in=20, out=160, looseness=7] (e-15.east);
    \path[pointer-right] (e-15.east) edge[in=-160, out=-20, looseness=7] (e-15.west);
    \path[pointer-child] (e-15.south) edge[bend right] (e-ldots.north);
    \path[pointer-parent] (e-15.north) edge[bend right] (e-13.south);
    \path[pointer-child] (e-ldots.south) edge[bend right] (e-37.north);
    \path[pointer-parent] (e-ldots.north) edge[bend right] (e-15.south);
    \path[pointer-left] (e-37.west) edge[in=20, out=160, looseness=7] (e-37.east);
    \path[pointer-right] (e-37.east) edge[in=-160, out=-20, looseness=7] (e-37.west);
    \path[pointer-child] (e-37.south) edge[bend right] (e-39.north);
    \path[pointer-parent] (e-37.north) edge[bend right] (e-ldots.south);
    \path[pointer-left] (e-39.west) edge[in=20, out=160, looseness=7] (e-39.east);
    \path[pointer-right] (e-39.east) edge[in=-160, out=-20, looseness=7] (e-39.west);
    \path[pointer-child] (e-39.south) edge[bend right] (e-40.north);
    \path[pointer-parent] (e-39.north) edge[bend right] (e-37.south);
    \path[pointer-left] (e-40.west) edge[in=20, out=160, looseness=7] (e-40.east);
    \path[pointer-right] (e-40.east) edge[in=-160, out=-20, looseness=7] (e-40.west);
    \path[pointer-parent] (e-40.north) edge[bend right] (e-39.south);
  \end{tikzpicture}
  \caption{Gew{\"u}nschtes Ergebnis: $B_1$ dessen Wurzel min. $4$ S{\"o}hne hat
    und ein entarteter Baum $B_2$. Beide B{\"a}ume haben jeweils
    $\frac{n}{2} = 16$ Knoten.
  }
\end{figure}

\end{document}