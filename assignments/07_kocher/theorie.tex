\documentclass{article}

\usepackage{amsmath,amsthm,amssymb}
\usepackage{commath}
\usepackage{mathtools}
\usepackage{enumerate}
\usepackage{subcaption}
\usepackage{float}
\usepackage{tikz}
\usepackage[margin=1in]{geometry}
\usepackage{multicol}

\usetikzlibrary{positioning, shapes.geometric}

\setlength{\parindent}{0pt}
\setlength{\parskip}{8pt}

\usepackage[utf8]{inputenc}
\begin{document}
\title{Assignment 7 \\ Advanced Algorithms \& Data Structures PS}%
\author{Christian Müller 1123410 \\ Daniel Kocher, 0926293}%
\maketitle

{\bfseries Aufgabe 14}%

Sei $U = \left\{ 0, \ldots, 30 \right\}$ und
$S = \left\{ 2, 4, 7, 11, 12, 18, 19, 22, 26, 29 \right\}$. Benutzen Sie das
zweistufige Schema aus der Vorlesung um eine perfekte Hash-Funktion mit $k = 3$,
$N = 31$ und $n = 10$ zu finden. Berechnen Sie f{\"u}r $i = 0, \ldots, n - 1$ die
Werte $W_i$, $b_i$, $m_i$, $k_i$ und die Funktion $h_{k_i}$. Geben Sie
anschlie{\ss}end f{\"u}r jedes Element aus $S$ die Position dieses Elementes in
der Hash-Tafel an.

Die Hash-Funktion f{\"u}r die erste Stufe wird allgemein wie folgt berechnet:
\begin{equation}
  h_k \left( x \right) = \left( \left( k \cdot x \right)\text{ mod }N \right)\text{ mod }m
\end{equation}
wobei $m = n = |S| = 10$ ist.

Setzt man nun die Werte aus der Angabe ein - $k = 3$, $N = 31$ und $n = 10$ -
bekommt man die folgende konkrete Hash-Funktion f{\"u}r die erste Stufe:
\begin{equation}
  h_3 \left( x \right) = \left( \left( 3 \cdot x \right)\text{ mod }31 \right)\text{ mod }10
  \label{eq:hk}
\end{equation}

Nun wird f{\"u}r $i = 0, \ldots, n - 1 = 9$
\parskip0pt\begin{enumerate}
  \item $W_i = \left\{ x \in S: h_k \left( x \right) = i \right\}$ mit $h_k = h_3$
  \item $b_i = | W_i |$
  \item $m_i = 2 \cdot b_i \cdot \left( b_i - 1 \right) + 1$
\end{enumerate}
berechnet, siehe Table~\ref{tbl:step-2}.

\begin{multicols}{2}
\begin{table}[H]
  \centering
  \begin{tabular}{l|l|l|l}
    $i$ & $W_i$           & $b_i$ & $m_i$ \tabularnewline
    \hline\hline
    $0$ &                 & $0$   & $1$   \tabularnewline
    \hline
    $1$ & $7$             & $1$   & $1$   \tabularnewline
    \hline
    $2$ & $4$, $11$       & $2$   & $5$   \tabularnewline
    \hline
    $3$ & $18$            & $1$   & $1$   \tabularnewline
    \hline
    $4$ & $22$            & $1$   & $1$   \tabularnewline
    \hline
    $5$ & $12$, $29$      & $2$   & $5$   \tabularnewline
    \hline
    $6$ & $2$, $19$, $26$ & $3$   & $13$   \tabularnewline
    \hline
    $7$ &                 & $0$   & $1$   \tabularnewline
    \hline
    $8$ &                 & $0$   & $1$   \tabularnewline
    \hline
    $9$ &                 & $0$   & $1$   \tabularnewline
    \hline
  \end{tabular}
  \caption{Werte f{\"u}r Schritt 2 des zweistufigen Schemas aus der Vorlesung.}
  \label{tbl:step-2}
\end{table}
\columnbreak%
\begin{table}[H]
  \centering
  \begin{tabular}{l|l}
    $i$ & $s_i$ \tabularnewline
    \hline\hline
    $0$ & $0$   \tabularnewline
    \hline
    $1$ & $1$   \tabularnewline
    \hline
    $2$ & $2$   \tabularnewline
    \hline
    $3$ & $7$   \tabularnewline
    \hline
    $4$ & $8$   \tabularnewline
    \hline
    $5$ & $9$   \tabularnewline
    \hline
    $6$ & $14$  \tabularnewline
    \hline
    $7$ & $27$  \tabularnewline
    \hline
    $8$ & $28$  \tabularnewline
    \hline
    $9$ & $29$  \tabularnewline
    \hline
  \end{tabular}
  \caption{Werte f{\"u}r Schritt 3 des zweistufigen Schemas aus der Vorlesung.}
  \label{tbl:step-3}
\end{table}
\end{multicols}

Nun m{\"u}ssen $k_i$ so gew{\"a}hlt werden, dass
\begin{equation}
  h_{k_i}: x \longrightarrow \left( \left( k_i \cdot x \right)\text{ mod }N \right)\text{ mod }m_i
  \label{eq:step-2}
\end{equation}
eingeschr{\"a}nkt auf $W_i$ injektiv ist. Durch diese zweite Stufe werden jene
Eintr{\"a}ge $W_i$ mit $b_i = | W_i | > 1$ auf mehrere "Unterbuckets" abgebildet.

$b_i > 1$ ist f{\"u}r $i = \left\{ 2, 5, 6 \right\}$ der Fall. F{\"u}r
$i = \left\{ 1, 3, 4 \right\}$ ben{\"o}tigen wir keine zus{\"a}tzliche
Hash-Funktion $h_{k_i}$, da nur ein einziger Wert im Bucket vorhanden ist - eine
weitere Aufteilung auf "Unterbuckets" macht also keinen Sinn.

Wir berechnen nun $k_i$ nach Gleichung~\ref{eq:step-2} f{\"u}r
$\left( i, W_i, m_i \right) \in \left\{ \left( 2, \left\{ 4, 11 \right\}, 5 \right), \left( 5, \left\{ 12, 29 \right\}, 5 \right), \left( 6, \left\{ 2, 19, 26 \right\}, 13 \right) \right\}$
und $N = 31$:
\parskip0pt\begin{itemize}
  \item $i = 2$, $W_2 = \left\{ 4, 11 \right\}$ und $m_2 = 5$: \newline
    W{\"a}hle $k_2 = 1$: Ist $h_{k_2}$ eingeschr{\"a}nkt auf $W_2$ injektiv?
    \newline $h_{k_2} \left( 4 \right) = 4$ und $h_{k_2} \left( 11 \right) = 1$
    $\Rightarrow$ $h_{k_2}$ ist eingeschr{\"a}nkt auf $W_2$ injektiv.
  \item $i = 5$, $W_5 = \left\{ 12, 29 \right\}$ und $m_5 = 5$: \newline
    W{\"a}hle $k_5 = 1$: Ist $h_{k_5}$ eingeschr{\"a}nkt auf $W_5$ injektiv?
    \newline $h_{k_5} \left( 12 \right) = 2$ und $h_{k_5} \left( 29 \right) = 4$
    $\Rightarrow$ $h_{k_5}$ ist eingeschr{\"a}nkt auf $W_5$ injektiv.
  \item $i = 6$, $W_6 = \left\{ 2, 19, 26 \right\}$ und $m_6 = 13$: \newline
    W{\"a}hle $k_6 = 1$: Ist $h_{k_6}$ eingeschr{\"a}nkt auf $W_6$ injektiv?
    \newline $h_{k_6} \left( 2 \right) = 2$, $h_{k_6} \left( 19 \right) = 6$ und
    $h_{k_6} \left( 26 \right) = 0$ $\Rightarrow$ $h_{k_6}$ ist eingeschr{\"a}nkt
    auf $W_6$ injektiv.
\end{itemize}

Daraus ergeben sich f{\"u}r die zweite Stufe folgende Hash-Funktionen f{\"u}r
$i \in \left\{ 2, 5, 6 \right\}$
\begin{equation}
  h_{k_2} \left( x \right) = h_{k_5} \left( x \right) = \left( x\text{ mod }31 \right)\text{ mod }5
  \label{eq:hk1}
\end{equation}
\begin{equation}
  h_{k_6} \left( x \right) = \left( x\text{ mod }31 \right)\text{ mod }13
\end{equation}
und f{\"u}r $i \in \left\{ 0, 1, 3, 4, 7, 8, 9 \right\}$
\begin{equation}
  h_{k_0} \left( x \right) = h_{k_1} \left( x \right) = h_{k_3} \left( x \right) = h_{k_4} \left( x \right) = h_{k_7} \left( x \right) = h_{k_8} \left( x \right) = h_{k_9} \left( x \right) = \left( x\text{ mod }31 \right)\text{ mod }1
  \label{eq:hki}
\end{equation}

Basierend auf diesen Hash-Funktionen $h_{k_0}$ bis $h_{k_9}$ wird nun der finale
Schritt vorgenommen. Zuerst werden $s_i = \sum_{j < i} m_j$ berechnet und dann
wird $x \in S$ in Tafelposition $T\left[ s_i + j \right]$ gespeichert, wobei
$i = h_k \left( x \right)$ (erste Hash-Funktion, siehe Gleichung~\ref{eq:hk}) und
$j = h_{k_i} \left( x \right)$ (zweite Hash-Funktion, siehe
Gleichungen~\ref{eq:hk1} -~\ref{eq:hki}).

F{\"u}r die Berechnung von $s_i$ f{\"u}r $i = \left\{ 0, \ldots, 9 \right\}$ siehe
Table~\ref{tbl:step-3}. Die resultierenden Positionen in der Hash-Tafel sind in
Table~\ref{tbl:final-pos} aufgelistet.

\begin{table}[H]
  \centering
  \begin{tabular}{l|l}
    $x$ & $s_i + j$ (Position von $x$)  \tabularnewline
    \hline\hline
    $2$ & $14 + 2 = \textbf{16}$        \tabularnewline
    \hline
    $4$ & $2 + 4 = \textbf{6}$          \tabularnewline
    \hline
    $7$ & $1 + 0 = \textbf{1}$          \tabularnewline
    \hline
    $11$ & $2 + 1 = \textbf{3}$         \tabularnewline
    \hline
    $12$ & $9 + 2 = \textbf{11}$         \tabularnewline
    \hline
    $18$ & $7 + 0 = \textbf{7}$         \tabularnewline
    \hline
    $19$ & $14 + 6 = \textbf{20}$       \tabularnewline
    \hline
    $22$ & $8 + 0 = \textbf{8}$         \tabularnewline
    \hline
    $26$ & $14 + 0 = \textbf{14}$       \tabularnewline
    \hline
    $29$ & $9 + 4 = \textbf{13}$        \tabularnewline
    \hline
  \end{tabular}
  \caption{Finale Positionen von $x \in S$ in der Hash-Tafel.}
  \label{tbl:final-pos}
\end{table}

\clearpage

{\bfseries Aufgabe 15}%

Wir nehmen an, dass in einer Datenstruktur die $i$-te Operation Kosten $c_i$
verursacht, wobei $c_i = i$ f{\"u}r alle $i$ der Form $2^k + 1$ ($k$ ist eine
nat{\"u}rliche Zahl) und $c_i = 2$ sonst. Zeigen Sie, dass die amortisierten
Kosten einer Operation immer konstant sind.
\begin{multicols}{2}
\begin{center}
$c_i=\begin{cases} i$, falls $i=2^k+1$ mit $k \in \mathbb{N}  \\ 2$, sonst.$  \end{cases}$
\end{center}
\begin{table}[H]
  \centering
  \begin{tabular}{l|l}
    $Op_i$ & $c_i$ \tabularnewline
    \hline\hline
    $1$ & $2$        \tabularnewline
    \hline
    $2$ & $2$          \tabularnewline
    \hline
    $3$ & $3$          \tabularnewline
    \hline
    $4$ & $2$         \tabularnewline
    \hline
    $5$ & $5$         \tabularnewline
    \hline
    $6$ & $2$         \tabularnewline
    \hline
    $7$ & $2$       \tabularnewline
    \hline
    $8$ & $2$         \tabularnewline
    \hline
    $9$ & $9$       \tabularnewline
    \hline
    $10$ & $2$        \tabularnewline
    \hline
  \end{tabular}
  \caption{Beispiel für $n=10$ Operationen}
  \label{tbl:10-ops}
\end{table}
\end{multicols}
Aggregat-Methode mit $n$ Operationen\\
O. B. d. A. betrachten wir den Fall $n=2^k$ mit $k \in \mathbb{N}$\\
\begin{enumerate}

\item  Gesamtkosten $C_1$ der Menge ($A_1$) der Operationen ($a_i$)  für die $i=2^j+1$ mit $j \in \mathbb{N}$ und $i\leq n$ gilt:\\
Für alle $a_i$ aus $A_1$ gilt $ 0 \leq j \leq k-1$:
\begin{equation}
C_1=(2^0+1)+(2^1+1)+(2^2+1)+...+(2^{k-1}+1)
\end{equation}
wegen $n=2^k \implies k=\operatorname{ld}(n)$ gilt:
\begin{equation}
C_1=(2^0+1)+(2^1+1)+(2^2+1)+...+(2^{\operatorname{ld}(n)-1}+1)
\end{equation}
\begin{equation}
C_1={\operatorname{ld}(n)}+\sum_{i=0}^{\operatorname{ld}(n)-1} 2^{i}
\end{equation}
Es gilt: (siehe \ref{proof})
\begin{equation}
\sum_{i=0}^{n} 2^{i}=2^{n+1}-1
\end{equation}
Also:\\
\begin{equation}
C_1={\operatorname{ld}(n)}+2^{\operatorname{ld}(n)}-1
\end{equation}
\item Gesamtkosten $C_2$ der Menge ($A_2$) der Operationen ($a_i$) für die $i\neq2^j+1$ mit $j \in \mathbb{N}$ gilt:\\
$|A1|={\operatorname{ld}(n)}$ und $|A_1|+|A_2|=n \implies |A_2|=n-{\operatorname{ld}(n)} $
\begin{equation}
C_2=2 \cdot ({n-\operatorname{ld}(n)})
\end{equation}
\end{enumerate}
Seien $C$ die Gesamtkosten der $n$ Operationen dann:
\begin{equation}
C=C_1+C_2={\operatorname{ld}(n)}+2^{\operatorname{ld}(n)}-1+2 \cdot ({n-\operatorname{ld}(n)})
\end{equation}
\begin{equation}
C={\operatorname{ld}(n)}+n-1+2 n-2{\operatorname{ld}(n)}
\end{equation}
\begin{equation}
C=3n-{\operatorname{ld}(n)}-1
\end{equation}
\begin{equation}
3n-{\operatorname{ld}(n)}-1
\end{equation}
\begin{equation}
a_i=\frac{3n-{\operatorname{ld}(n)}-1}{n} < 3
\end{equation}
Die amortisierten Kosten einer Operation sind höchstens $3$ und somit in $O(1)$.
\newpage
\begin{proof}


\begin{equation}
\sum_{i=0}^{n} 2^{i}=2^{n+1}-1
\end{equation}
IB: $n=0$ \\
\begin{equation}
\sum_{i=0}^{0} 2^{i}= 2^0 = 1 = 2^1-1 =2^{0+1}-1
\end{equation}
IH: Für ein beliebes aber festes $k \in \mathbb{N}$ gilt: 
\begin{equation}
\sum_{i=0}^{k} 2^{i}=2^{k+1}-1
\end{equation}
IS: $k \rightarrow k+1$\\
Zu zeigen:
\begin{equation}
 \sum_{i=0}^{k+1} 2^{i}=2^{k+2}-1
\end{equation}
\begin{align*}
 \sum_{i=0}^{k+1} 2^{i}=\sum_{i=0}^{k} 2^{i} +2^{k+1} \\
 =2^{k+1}-1+2^{k+1} \\
 =2\cdot2^{k+1}-1 \\
 =2^{k+2}-1
\end{align*}

\end{proof}
\end{document}
