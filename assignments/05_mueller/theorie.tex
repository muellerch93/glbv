\documentclass{article}

\usepackage{amsmath,amsthm,amssymb}
\usepackage{commath}
\usepackage{mathtools}

\usepackage[utf8]{inputenc}
\begin{document}
\title{Assignment 5 \\ Advanced Algorithms \& Data Structures PS}%
\author{Christian Müller 1123410 \\ Daniel Kocher, 0926293}%
\maketitle

{\noindent\bfseries Aufgabe 9}%
\medskip%

\noindent
Zu zeigen: Sei $m < i$: $x_{i}$ Vorfahre von $x_{m}$ $\Leftrightarrow$ $P_{min}(\lbrace x_{m},...,x_{i} \rbrace)=x_{i}$ \\

\begin{proof}
\noindent

\noindent
i) $"\Leftarrow"$\\
Annahme: $P_{min}(\lbrace x_{m},...,x_{i} \rbrace)=x_{i}$\\
Knoten werden nach Prioritäten in den Suchbaum eingefügt $\Rightarrow$ aus der Menge $\lbrace x_{m},...,x_{i} \rbrace$ wird $x_{i}$ als erster eingefügt.\\
Alle Knoten $x_{k}$ mit Schlüsseln $k$, die vor $x_{i}$ eingefügt wurden, haben die Eigenschaft: $k < key(x_{m})$ oder $k > key(x_{i})$\footnote[1]{Würde $key(x_{m}) \leq k \leq key(x_{i})$ gelten wäre der Knoten mit dem Schlüssel $k$ Teil der Menge $\lbrace x_{m},...,x_{i} \rbrace$ und würde wegen $P_{min}(\lbrace x_{m},...,x_{i} \rbrace)=x_{i}$ nach $x_{i}$ eingefügt werden}.\\
Wenn $x_{j} \in \lbrace x_{m},...,x_{i-1} \rbrace$ eingefügt wird, durchläuft $x_{j}$ denselben Pfad wie $x_{i}$ \footnote[2]{Es gilt für alle sich im Baum befindlichen Schlüssel $k$, 
$k < key(x_{m}) \leq key(x_{j}) \leq key(x_{i})$  oder $k > key(x_{i}) \geq key(x_{j}) \geq key(x_{m})$ } und wird im linken Unterbaum von $x_{i}$ eingefügt.
Es gilt daher: $x_{j}$ ist Nachfahre von $x_{i}$ und insbesonders $x_{m}$ ist Nachfahre von $x_{i}$ $\implies$ $x_{i}$ ist Vorfahre von $x_{m}$\\

\noindent
ii) $"\Rightarrow"$\\
Sei: $P_{min}(\lbrace x_{m},...,x_{i} \rbrace)=x_{j}$; Zeige: $i=j$\\
Annahme: $x_{i}$ Vorfahre von $x_{m}$\\
Knoten werden nach Prioritäten in den Suchbaum eingefügt $\Rightarrow$ aus der Menge $\lbrace x_{m},...,x_{i} \rbrace$ wird $x_{j}$ als erster eingefügt.\\
Alle Knoten $x_{k}$ mit Schlüssel $k$, die vor $x_{j}$ eingefügt wurden, haben die Eigenschaft: $k < key(x_{m})$ oder $k > key(x_{i})$.
Jeder Knoten $x_{l}$ aus $\lbrace x_{m},...,x_{i} \rbrace$ mit $l \neq j$ muss beim Einfügen denselben Pfad durchlaufen wie $x_{j}$.\\
Falls $j \neq i,m$: $x_{m}$ landet im linken Unterbaum von $x_{j}$ und $x_{i}$ im rechten Unterbaum von $x_{j}$ $\implies$ $x_{i}$ ist kein Vorfahr von $x_{m}$.\\
Falls $j=m$: $x_{i}$ landet im rechten Unterbaum von $x_{m} \implies x_{m}$ ist Vorfahre von $x_{i}$\\
$\implies j=i$ 
\end{proof}
{\noindent\bfseries Aufgabe 10}%
\medskip%\\

\noindent
Zu zeigen: Sei $m$ die Anzahl der Schlüssel, die kleiner als der gesuchte Schlüssel $k$ sind. Die erwartete Anzahl von Knoten auf dem Suchpfad ist $H_{m}+H_{n-m}$. 
\begin{proof}
\noindent

\noindent
Sei $x_{k}$ der Knoten mit Schlüssel $k$ der ohne Berücksichtigung der Prioritäten in den gegebenen Suchbaum imaginär eingefügt wird.\\

\noindent
Sei:\\
  \[
    X_{k,i}=\left\{
                \begin{array}{ll}
                  1\text{, } x_{i} \text{ ist Vorfahr des neuen imaginären Knotens } x_{k} \text{ mit } key(x_{k})=k\\
                  0 \text{, sonst}\\
                
                \end{array}
              \right.
  \]
und:\\
Sei $X_{k}$ die Anzahl der Knoten auf dem Pfad von der Wurzel nach Knoten $x_{k}$ (ohne $x_{k}$, denn $x_k$ existiert nur imaginär).
\begin{equation}
X_{k}=\sum_{i<k} X_{k,i} +\sum_{i>k} X_{k,i}
\end{equation}
\begin{equation}
E\left[X_{k}\right]=E\left[\sum_{i<k} X_{k,i}\right] +E\left[\sum_{i>k} X_{k,i}\right]
\end{equation}
\begin{equation}
E\left[X_{k}\right]=\sum_{i<k} E\left[X_{k,i}\right] +\sum_{i>k} E\left[X_{k,i}\right]
\end{equation}
F1$(i<k)$: gilt für $m$ Knoten des Suchbaums (aus der Angabe)\\
\begin{equation}
E\left[X_{k,i}\right]=Pr\left[P_{min}(\lbrace x_{1},...,x_{i}\rbrace)=x_{i}\right]=\frac{1}{i}
\end{equation}
F2$(i>k)$: gilt für $n-m$ Knoten des Suchbaums\\
\begin{equation}
E\left[X_{k,i}\right]=Pr\left[P_{min}(\lbrace x_{i},...,x_{n}\rbrace)=x_{i}\right]=\frac{1}{n-i+1}
\end{equation}
Also:\\
\begin{equation}
E\left[X_{k}\right]=\sum_{i<k} E\left[X_{k,i}\right] +\sum_{i>k} E\left[X_{k,i}\right]
\end{equation}
\begin{equation}
E\left[X_{k}\right]=\sum_{i=1}^{m} \frac{1}{i} +\sum_{i=m+1}^{n} \frac{1}{n-i+1}
\end{equation}
\begin{equation}
E\left[X_{k}\right]=\left(1+\frac{1}{2}+...+\frac{1}{m}\right)+\left(\frac{1}{n-(m+1)+1}+...+\frac{1}{n-n+1}\right)
\end{equation}
\begin{equation}
E\left[X_{k}\right]=\left(1+\frac{1}{2}+...+\frac{1}{m}\right)+\left(\frac{1}{n-m}+\frac{1}{n-m-1}+...+1\right)
\end{equation}
\begin{equation}
E\left[X_{k}\right]=H_{m}+H_{n-m}
\end{equation}
\end{proof}
\end{document}