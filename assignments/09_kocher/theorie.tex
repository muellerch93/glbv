\documentclass{article}

\usepackage{amsmath,amsthm,amssymb}
\usepackage{commath}
\usepackage{mathtools}
\usepackage{enumerate}
\usepackage{subcaption}
\usepackage{float}
\usepackage{tikz}
\usepackage[margin=1in]{geometry}
\usepackage{multicol}

\usetikzlibrary{positioning, shapes, automata, arrows, backgrounds}

\setlength{\parindent}{0pt}
\setlength{\parskip}{8pt}

\usepackage[utf8]{inputenc}
\begin{document}
\title{Assignment 9 \\ Advanced Algorithms \& Data Structures PS}%
\author{Christian Müller 1123410 \\ Daniel Kocher, 0926293}%
\maketitle

{\bfseries Aufgabe 18}%

Sei $Q$ eine Binomial Queue, die anfangs genau einen Binomialbaum $B_1$ mit den
Schl{\"u}sseln $13$ und $21$ enth{\"a}lt. F{\"u}gen Sie die Schl{\"u}ssel $3$,
$7$, $15$, $18$, $8$, $14$ und $27$ in die Queue ein. L{\"o}schen Sie
anschlie{\ss}end die Elemente $15$ und $27$ und wenden Sie
\texttt{decreasekey(18, 4)} an. Geben Sie nach jedem Schritt die resultierende
Queue an.

\tikzstyle{elem}=[draw, circle, thick, fill=blue!20, minimum size=10mm]
\tikzstyle{pointer}=[->, >=stealth, thick]
\tikzstyle{pointer-sibling}=[pointer, dotted]
\tikzstyle{pointer-child}=[pointer, solid]
\tikzstyle{pointer-parent}=[pointer, dashed]

Im Folgenden werden die einzelnen Schritte dargestellt, wobei f{\"u}r die
Child-Sibling-Parent Darstellung die folgenden Pfeile verwendet werden:
\begin{figure}[H]
  \centering
  \begin{tikzpicture}
    \draw[pointer-child] (0, 0) -- ++(0.5, 0) node[right] {\ldots Child-Pointer};
    \draw[pointer-parent] (5, 0) -- ++(0.5, 0) node[right] {\ldots Parent-Pointer};
    \draw[pointer-sibling] (10, 0) -- ++(0.5, 0) node[right] {\ldots Sibling-Pointer};
  \end{tikzpicture}
\end{figure}

\begin{multicols}{2}
\begin{figure}[H]
  \centering
  \begin{tikzpicture}
    \node[elem] (e-13) {$13$};
    \node[elem, below = 1 of e-13] (e-21) {$21$}
      edge [pointer-parent, bend right] (e-13);
    \draw[pointer-child] (e-13.south) -- (e-21.north);
  \end{tikzpicture}
  \caption{Ausgangs-Queue $Q$}
\end{figure}
\columnbreak
\begin{figure}[H]
  \centering
  \begin{tikzpicture}
    \node[elem] (e-13) {$13$};
    \node[elem, below = 1 of e-13] (e-21) {$21$}
      edge [pointer-parent, bend right] (e-13);
    \draw[pointer-child] (e-13.south) -- (e-21.north);

    \node[elem, right = 1 of e-13] (e-3) {$3$};

    \node[below right = 1 of e-3] (meld) {$\Longrightarrow$};
    \node[above = 0 of meld] (meld-t) {$meld$};

    \node[elem, right = 2 of e-3] (e-3) {$3$};
    \node[elem, right = 1 of e-3] (e-13) {$13$};
    \node[elem, below = 1 of e-13] (e-21) {$21$}
      edge [pointer-parent, bend right] (e-13);
    \draw[pointer-child] (e-13.south) -- (e-21.north);
    \draw[pointer-sibling] (e-3.east) -- (e-13.west);
  \end{tikzpicture}
  \caption{Einf{\"u}gen von $3$ ($meld$: $B_0$ vor $B_1$)} 
\end{figure}
\end{multicols}

\begin{figure}[H]
  \centering
  \begin{tikzpicture}
    \node[elem] (e-3) {$3$};
    \node[elem, right = 1 of e-3] (e-13) {$13$};
    \node[elem, below = 1 of e-13] (e-21) {$21$}
      edge[pointer-parent, bend right] (e-13);
    \draw[pointer-child] (e-13.south) -- (e-21.north);
    \draw[pointer-sibling] (e-3.east) -- (e-13.west);
    
    \node[elem, right = 1 of e-13] (e-7) {$7$};
    
    \node[below right = 1 of e-7] (meld-1) {$\Longrightarrow$};
    \node[above = 0 of meld-1] (meld-1-t) {$meld$};

    \node[elem, right = 2 of e-7] (e-3) {$3$};
    \node[elem, below = 1 of e-3] (e-7) {$7$}
      edge[pointer-parent, bend right] (e-3);
    \draw[pointer-child] (e-3.south) -- (e-7.north);
    \node[elem, right = 1 of e-3] (e-13) {$13$};
    \node[elem, below = 1 of e-13] (e-21) {$21$}
      edge[pointer-parent, bend right] (e-13);
    \draw[pointer-child] (e-13.south) -- (e-21.north);
    \draw[pointer-sibling] (e-3.east) -- (e-13.west);

    \node[below right = 1 of e-13] (meld-2) {$\Longrightarrow$};
    %\node[above = 0 of meld-2] (meld-2-t) {$meld$};

    \node[elem, right = 4 of e-13] (e-3) {$3$};
    \node[elem, below = 1 of e-3] (e-7) {$7$}
      edge[pointer-parent, bend right] (e-3);
    \node[elem, left = 1 of e-7] (e-13) {$13$}
      edge[pointer-parent, bend left] (e-3);
    \draw[pointer-sibling] (e-13.east) -- (e-7.west);
    \node[elem, below = 1 of e-13] (e-21) {$21$}
      edge[pointer-parent, bend right] (e-13);
    \draw[pointer-child] (e-13.south) -- (e-21.north);
    \draw[pointer-child] (e-3.south west) -- (e-13.north east);
  \end{tikzpicture}
  \caption{Einf{\"u}gen von $7$ ($meld$ vereinigt zweimal: $B_0$ und $B_1$)} 
\end{figure}

\begin{figure}[H]
  \centering
  \begin{tikzpicture}
    \node[elem] (e-3) {$3$};
    \node[elem, below = 1 of e-3] (e-7) {$7$}
      edge[pointer-parent, bend right] (e-3);
    \node[elem, left = 1 of e-7] (e-13) {$13$}
      edge[pointer-parent, bend left] (e-3);
    \draw[pointer-sibling] (e-13.east) -- (e-7.west);
    \node[elem, below = 1 of e-13] (e-21) {$21$}
      edge[pointer-parent, bend right] (e-13);
    \draw[pointer-child] (e-13.south) -- (e-21.north);
    \draw[pointer-child] (e-3.south west) -- (e-13.north east);

    \node[elem, right = 1 of e-3] (e-15) {$15$};

    \node[right = 3 of e-7] (meld) {$\Longrightarrow$};
    \node[above = 0 of meld] (meld-t) {$meld$};
    
    \node[elem, right = 5 of e-3] (e-15) {$15$};
    \node[elem, right = 3 of e-15] (e-3) {$3$};
    \draw[pointer-sibling] (e-15.east) -- (e-3.west);
    \node[elem, below = 1 of e-3] (e-7) {$7$}
      edge[pointer-parent, bend right] (e-3);
    \node[elem, left = 1 of e-7] (e-13) {$13$}
      edge[pointer-parent, bend left] (e-3);
    \draw[pointer-sibling] (e-13.east) -- (e-7.west);
    \node[elem, below = 1 of e-13] (e-21) {$21$}
      edge[pointer-parent, bend right] (e-13);
    \draw[pointer-child] (e-13.south) -- (e-21.north);
    \draw[pointer-child] (e-3.south west) -- (e-13.north east);
  \end{tikzpicture}
  \caption{Einf{\"u}gen von $15$ ($meld$: $B_0$ vor $B_2$)} 
\end{figure}

\begin{figure}[H]
  \centering
  \begin{tikzpicture}
    \node[elem] (e-15) {$15$};
    \node[elem, right = 3 of e-15] (e-3) {$3$};
    \draw[pointer-sibling] (e-15.east) -- (e-3.west);
    \node[elem, below = 1 of e-3] (e-7) {$7$}
      edge[pointer-parent, bend right] (e-3);
    \node[elem, left = 1 of e-7] (e-13) {$13$}
      edge[pointer-parent, bend left] (e-3);
    \draw[pointer-sibling] (e-13.east) -- (e-7.west);
    \node[elem, below = 1 of e-13] (e-21) {$21$}
      edge[pointer-parent, bend right] (e-13);
    \draw[pointer-child] (e-13.south) -- (e-21.north);
    \draw[pointer-child] (e-3.south west) -- (e-13.north east);

    \node[elem, right = 1 of e-3] (e-18) {$18$};

    \node[right = 3 of e-7] (meld) {$\Longrightarrow$};
    \node[above = 0 of meld] (meld-t) {$meld$};

    \node[elem, right = 3 of e-18] (e-15) {$15$};
    \node[elem, below = 1 of e-15] (e-18) {$18$}
      edge[pointer-parent, bend right] (e-15);
    \draw[pointer-child] (e-15.south) -- (e-18.north);
    \node[elem, right = 3 of e-15] (e-3) {$3$};
    \draw[pointer-sibling] (e-15.east) -- (e-3.west);
    \node[elem, below = 1 of e-3] (e-7) {$7$}
      edge[pointer-parent, bend right] (e-3);
    \node[elem, left = 1 of e-7] (e-13) {$13$}
      edge[pointer-parent, bend left] (e-3);
    \draw[pointer-sibling] (e-13.east) -- (e-7.west);
    \node[elem, below = 1 of e-13] (e-21) {$21$}
      edge[pointer-parent, bend right] (e-13);
    \draw[pointer-child] (e-13.south) -- (e-21.north);
    \draw[pointer-child] (e-3.south west) -- (e-13.north east);

  \end{tikzpicture}
  \caption{Einf{\"u}gen von $18$ ($meld$ vereinigt einmal: $B_0$)} 
\end{figure}

\begin{figure}[H]
  \centering
  \begin{tikzpicture}
    \node[elem] (e-15) {$15$};
    \node[elem, below = 1 of e-15] (e-18) {$18$}
      edge[pointer-parent, bend right] (e-15);
    \draw[pointer-child] (e-15.south) -- (e-18.north);
    \node[elem, right = 3 of e-15] (e-3) {$3$};
    \draw[pointer-sibling] (e-15.east) -- (e-3.west);
    \node[elem, below = 1 of e-3] (e-7) {$7$}
      edge[pointer-parent, bend right] (e-3);
    \node[elem, left = 1 of e-7] (e-13) {$13$}
      edge[pointer-parent, bend left] (e-3);
    \draw[pointer-sibling] (e-13.east) -- (e-7.west);
    \node[elem, below = 1 of e-13] (e-21) {$21$}
      edge[pointer-parent, bend right] (e-13);
    \draw[pointer-child] (e-13.south) -- (e-21.north);
    \draw[pointer-child] (e-3.south west) -- (e-13.north east);

    \node[elem, right = 1 of e-3] (e-8) {$8$};

    \node[right = 3 of e-7] (meld) {$\Longrightarrow$};
    \node[above = 0 of meld] (meld-t) {$meld$};

    \node[elem, right = 3 of e-8] (e-8) {$8$};
    \node[elem, right = 1 of e-8] (e-15) {$15$};
    \draw[pointer-sibling] (e-8.east) -- (e-15.west);
    \node[elem, below = 1 of e-15] (e-18) {$18$}
      edge[pointer-parent, bend right] (e-15);
    \draw[pointer-child] (e-15.south) -- (e-18.north);
    \node[elem, right = 3 of e-15] (e-3) {$3$};
    \draw[pointer-sibling] (e-15.east) -- (e-3.west);
    \node[elem, below = 1 of e-3] (e-7) {$7$}
      edge[pointer-parent, bend right] (e-3);
    \node[elem, left = 1 of e-7] (e-13) {$13$}
      edge[pointer-parent, bend left] (e-3);
    \draw[pointer-sibling] (e-13.east) -- (e-7.west);
    \node[elem, below = 1 of e-13] (e-21) {$21$}
      edge[pointer-parent, bend right] (e-13);
    \draw[pointer-child] (e-13.south) -- (e-21.north);
    \draw[pointer-child] (e-3.south west) -- (e-13.north east);
  \end{tikzpicture}
  \caption{Einf{\"u}gen von $8$ ($meld$: $B_0$ vor $B_1$ vor $B_2$)} 
\end{figure}

\begin{figure}[H]
  \centering
  \begin{tikzpicture}[scale = 0.8, every node/.style={scale=0.8}]
    \node[elem] (e-8) {$8$};
    \node[elem, right = 1 of e-8] (e-15) {$15$};
    \draw[pointer-sibling] (e-8.east) -- (e-15.west);
    \node[elem, below = 1 of e-15] (e-18) {$18$}
      edge[pointer-parent, bend right] (e-15);
    \draw[pointer-child] (e-15.south) -- (e-18.north);
    \node[elem, right = 3 of e-15] (e-3) {$3$};
    \draw[pointer-sibling] (e-15.east) -- (e-3.west);
    \node[elem, below = 1 of e-3] (e-7) {$7$}
      edge[pointer-parent, bend right] (e-3);
    \node[elem, left = 1 of e-7] (e-13) {$13$}
      edge[pointer-parent, bend left] (e-3);
    \draw[pointer-sibling] (e-13.east) -- (e-7.west);
    \node[elem, below = 1 of e-13] (e-21) {$21$}
      edge[pointer-parent, bend right] (e-13);
    \draw[pointer-child] (e-13.south) -- (e-21.north);
    \draw[pointer-child] (e-3.south west) -- (e-13.north east);

    \node[elem, right = 1 of e-3] (e-14) {$14$};

    \node[right = 3 of e-7] (meld-1) {$\Longrightarrow$};
    \node[above = 0 of meld-1] (meld-1-t) {$meld$};

    \node[elem, right = 3 of e-14] (e-8) {$8$};
    \node[elem, below = 1 of e-8] (e-14) {$14$}
      edge[pointer-parent, bend right] (e-8);
    \draw[pointer-child] (e-8.south) -- (e-14.north);
    \node[elem, right = 1 of e-8] (e-15) {$15$};
    \draw[pointer-sibling] (e-8.east) -- (e-15.west);
    \node[elem, below = 1 of e-15] (e-18) {$18$}
      edge[pointer-parent, bend right] (e-15);
    \draw[pointer-child] (e-15.south) -- (e-18.north);
    \node[elem, right = 3 of e-15] (e-3) {$3$};
    \draw[pointer-sibling] (e-15.east) -- (e-3.west);
    \node[elem, below = 1 of e-3] (e-7) {$7$}
      edge[pointer-parent, bend right] (e-3);
    \node[elem, left = 1 of e-7] (e-13) {$13$}
      edge[pointer-parent, bend left] (e-3);
    \draw[pointer-sibling] (e-13.east) -- (e-7.west);
    \node[elem, below = 1 of e-13] (e-21) {$21$}
      edge[pointer-parent, bend right] (e-13);
    \draw[pointer-child] (e-13.south) -- (e-21.north);
    \draw[pointer-child] (e-3.south west) -- (e-13.north east);

    \node[below = 0.5 of e-21] (meld-2) {$\Downarrow$};

    \node[elem, below = 4.5 of e-15] (e-8) {$8$};
    \node[elem, below = 1 of e-8] (e-14) {$14$}
      edge[pointer-parent, bend right] (e-8);
    \node[elem, left = 1 of e-14] (e-15) {$15$}
      edge[pointer-parent, bend left] (e-8);
    \draw[pointer-child] (e-8.south west) -- (e-15.north east);
    \draw[pointer-sibling] (e-15.east) -- (e-14.west);
    \node[elem, below = 1 of e-15] (e-18) {$18$}
      edge[pointer-parent, bend right] (e-15);
    \draw[pointer-child] (e-15.south) -- (e-18.north);
    \node[elem, right = 3 of e-8] (e-3) {$3$};
    \draw[pointer-sibling] (e-8.east) -- (e-3.west);
    \node[elem, below = 1 of e-3] (e-7) {$7$}
      edge[pointer-parent, bend right] (e-3);
    \node[elem, left = 1 of e-7] (e-13) {$13$}
      edge[pointer-parent, bend left] (e-3);
    \draw[pointer-sibling] (e-13.east) -- (e-7.west);
    \node[elem, below = 1 of e-13] (e-21) {$21$}
      edge[pointer-parent, bend right] (e-13);
    \draw[pointer-child] (e-13.south) -- (e-21.north);
    \draw[pointer-child] (e-3.south west) -- (e-13.north east);

    \node[left = 1 of e-15] (meld-3) {$\Longleftarrow$};

    \node[elem, left = 5 of e-8] (e-3) {$3$};
    \node[elem, below = 1 of e-3] (e-7) {$7$}
      edge[pointer-parent, bend right] (e-3);
    \node[elem, left = 1 of e-7] (e-13) {$13$}
      edge[pointer-parent, bend right] (e-3);
    \node[elem, left = 1 of e-13] (e-8) {$8$}
      edge[pointer-parent, bend left] (e-3);
    \draw[pointer-sibling] (e-13.east) -- (e-7.west);
    \node[elem, below = 1 of e-13] (e-21) {$21$}
      edge[pointer-parent, bend right] (e-13);
    \draw[pointer-child] (e-13.south) -- (e-21.north);
    \draw[pointer-child] (e-3.south west) -- (e-8.north east);
    \draw[pointer-sibling] (e-8.east) -- (e-13.west);
    \node[elem, below = 1 of e-8] (e-14) {$14$}
      edge[pointer-parent, bend right] (e-8);
    \node[elem, left = 1 of e-14] (e-15) {$15$}
      edge[pointer-parent, bend right] (e-8);
    \draw[pointer-sibling] (e-15.east) -- (e-14.west);
    \node[elem, below = 1 of e-15] (e-18) {$18$}
      edge[pointer-parent, bend right] (e-15);
    \draw[pointer-child] (e-8.south west) -- (e-15.north east);
    \draw[pointer-child] (e-15.south) -- (e-18.north);
  \end{tikzpicture}
  \caption{
    Einf{\"u}gen von $14$ ($meld$ vereinigt dreimal: $B_0$, $B_1$ und $B_2$)
  } 
\end{figure}

\begin{figure}[H]
  \centering
  \begin{tikzpicture}[scale = 0.8, every node/.style={scale=0.8}]
    \node[elem] (e-3) {$3$};
    \node[elem, below = 1 of e-3] (e-7) {$7$}
      edge[pointer-parent, bend right] (e-3);
    \node[elem, left = 1 of e-7] (e-13) {$13$}
      edge[pointer-parent, bend right] (e-3);
    \node[elem, left = 1 of e-13] (e-8) {$8$}
      edge[pointer-parent, bend left] (e-3);
    \draw[pointer-sibling] (e-13.east) -- (e-7.west);
    \node[elem, below = 1 of e-13] (e-21) {$21$}
      edge[pointer-parent, bend right] (e-13);
    \draw[pointer-child] (e-13.south) -- (e-21.north);
    \draw[pointer-child] (e-3.south west) -- (e-8.north east);
    \draw[pointer-sibling] (e-8.east) -- (e-13.west);
    \node[elem, below = 1 of e-8] (e-14) {$14$}
      edge[pointer-parent, bend right] (e-8);
    \node[elem, left = 1 of e-14] (e-15) {$15$}
      edge[pointer-parent, bend right] (e-8);
    \draw[pointer-sibling] (e-15.east) -- (e-14.west);
    \node[elem, below = 1 of e-15] (e-18) {$18$}
      edge[pointer-parent, bend right] (e-15);
    \draw[pointer-child] (e-8.south west) -- (e-15.north east);
    \draw[pointer-child] (e-15.south) -- (e-18.north);

    \node[elem, right = 1 of e-3] (e-27) {$27$};

    \node[right = 3 of e-7] (meld) {$\Longrightarrow$};
    \node[above = 0 of meld] (meld-t) {$meld$};

    \node[elem, right = 5 of e-3] (e-27) {$27$};  
    \node[elem, right = 5 of e-27] (e-3) {$3$};
    \draw[pointer-sibling] (e-27.east) -- (e-3.west);
    \node[elem, below = 1 of e-3] (e-7) {$7$}
      edge[pointer-parent, bend right] (e-3);
    \node[elem, left = 1 of e-7] (e-13) {$13$}
      edge[pointer-parent, bend right] (e-3);
    \node[elem, left = 1 of e-13] (e-8) {$8$}
      edge[pointer-parent, bend left] (e-3);
    \draw[pointer-sibling] (e-13.east) -- (e-7.west);
    \node[elem, below = 1 of e-13] (e-21) {$21$}
      edge[pointer-parent, bend right] (e-13);
    \draw[pointer-child] (e-13.south) -- (e-21.north);
    \draw[pointer-child] (e-3.south west) -- (e-8.north east);
    \draw[pointer-sibling] (e-8.east) -- (e-13.west);
    \node[elem, below = 1 of e-8] (e-14) {$14$}
      edge[pointer-parent, bend right] (e-8);
    \node[elem, left = 1 of e-14] (e-15) {$15$}
      edge[pointer-parent, bend right] (e-8);
    \draw[pointer-sibling] (e-15.east) -- (e-14.west);
    \node[elem, below = 1 of e-15] (e-18) {$18$}
      edge[pointer-parent, bend right] (e-15);
    \draw[pointer-child] (e-8.south west) -- (e-15.north east);
    \draw[pointer-child] (e-15.south) -- (e-18.north);  
  \end{tikzpicture}
  \caption{
    Einf{\"u}gen von $27$ ($meld$: $B_0$ vor $B_3$)
  } 
\end{figure}

\begin{figure}[H]
  \centering
  \begin{tikzpicture}[scale = 0.8, every node/.style={scale=0.8}]
    \node[elem] (e-27) {$27$};  
    \node[elem, right = 5 of e-27] (e-3) {$3$};
    \draw[pointer-sibling] (e-27.east) -- (e-3.west);
    \node[elem, below = 1 of e-3] (e-7) {$7$}
      edge[pointer-parent, bend right] (e-3);
    \node[elem, left = 1 of e-7] (e-13) {$13$}
      edge[pointer-parent, bend right] (e-3);
    \node[elem, left = 1 of e-13] (e-8) {$8$}
      edge[pointer-parent, bend left] (e-3);
    \draw[pointer-sibling] (e-13.east) -- (e-7.west);
    \node[elem, below = 1 of e-13] (e-21) {$21$}
      edge[pointer-parent, bend right] (e-13);
    \draw[pointer-child] (e-13.south) -- (e-21.north);
    \draw[pointer-child] (e-3.south west) -- (e-8.north east);
    \draw[pointer-sibling] (e-8.east) -- (e-13.west);
    \node[elem, below = 1 of e-8] (e-14) {$14$}
      edge[pointer-parent, bend right] (e-8);
    \node[elem, left = 1 of e-14] (e-15) {$-\infty$}
      edge[pointer-parent, bend right] (e-8);
    \draw[pointer-sibling] (e-15.east) -- (e-14.west);
    \node[elem, below = 1 of e-15] (e-18) {$18$}
      edge[pointer-parent, bend right] (e-15);
    \draw[pointer-child] (e-8.south west) -- (e-15.north east);
    \draw[pointer-child] (e-15.south) -- (e-18.north);

    \node[right = 1 of e-7] {$\Longrightarrow$};

    \node[elem, right = 3 of e-3] (e-27) {$27$};  
    \node[elem, right = 5 of e-27] (e-3) {$3$};
    \draw[pointer-sibling] (e-27.east) -- (e-3.west);
    \node[elem, below = 1 of e-3] (e-7) {$7$}
      edge[pointer-parent, bend right] (e-3);
    \node[elem, left = 1 of e-7] (e-13) {$13$}
      edge[pointer-parent, bend right] (e-3);
    \node[elem, left = 1 of e-13] (e-8) {$-\infty$}
      edge[pointer-parent, bend left] (e-3);
    \draw[pointer-sibling] (e-13.east) -- (e-7.west);
    \node[elem, below = 1 of e-13] (e-21) {$21$}
      edge[pointer-parent, bend right] (e-13);
    \draw[pointer-child] (e-13.south) -- (e-21.north);
    \draw[pointer-child] (e-3.south west) -- (e-8.north east);
    \draw[pointer-sibling] (e-8.east) -- (e-13.west);
    \node[elem, below = 1 of e-8] (e-14) {$14$}
      edge[pointer-parent, bend right] (e-8);
    \node[elem, left = 1 of e-14] (e-15) {$8$}
      edge[pointer-parent, bend right] (e-8);
    \draw[pointer-sibling] (e-15.east) -- (e-14.west);
    \node[elem, below = 1 of e-15] (e-18) {$18$}
      edge[pointer-parent, bend right] (e-15);
    \draw[pointer-child] (e-8.south west) -- (e-15.north east);
    \draw[pointer-child] (e-15.south) -- (e-18.north);

    \node[below = 3 of e-21] {$\Downarrow$};

    \node[elem, below = 8 of e-27] (e-27) {$27$};  
    \node[elem, right = 5 of e-27] (e-3) {$-\infty$};
    \draw[pointer-sibling] (e-27.east) -- (e-3.west);
    \node[elem, below = 1 of e-3] (e-7) {$7$}
      edge[pointer-parent, bend right] (e-3);
    \node[elem, left = 1 of e-7] (e-13) {$13$}
      edge[pointer-parent, bend right] (e-3);
    \node[elem, left = 1 of e-13] (e-8) {$3$}
      edge[pointer-parent, bend left] (e-3);
    \draw[pointer-sibling] (e-13.east) -- (e-7.west);
    \node[elem, below = 1 of e-13] (e-21) {$21$}
      edge[pointer-parent, bend right] (e-13);
    \draw[pointer-child] (e-13.south) -- (e-21.north);
    \draw[pointer-child] (e-3.south west) -- (e-8.north east);
    \draw[pointer-sibling] (e-8.east) -- (e-13.west);
    \node[elem, below = 1 of e-8] (e-14) {$14$}
      edge[pointer-parent, bend right] (e-8);
    \node[elem, left = 1 of e-14] (e-15) {$8$}
      edge[pointer-parent, bend right] (e-8);
    \draw[pointer-sibling] (e-15.east) -- (e-14.west);
    \node[elem, below = 1 of e-15] (e-18) {$18$}
      edge[pointer-parent, bend right] (e-15);
    \draw[pointer-child] (e-8.south west) -- (e-15.north east);
    \draw[pointer-child] (e-15.south) -- (e-18.north);
  \end{tikzpicture}
  \caption{
    L{\"o}schen von $15$: Ersetze $15$ mit $-\infty$ und lasse diesen Knoten nach
    oben wandern.
  } 
\end{figure}

\clearpage

\begin{figure}[H]
  \centering
  \begin{tikzpicture}[scale = 0.8, every node/.style={scale=0.8}]
    \node[elem] (e-27) {$27$};  
    \node[elem, right = 5 of e-27] (e-3) {$-\infty$};
    \draw[pointer-sibling] (e-27.east) -- (e-3.west);
    \node[elem, below = 1 of e-3] (e-7) {$7$}
      edge[pointer-parent, bend right] (e-3);
    \node[elem, left = 1 of e-7] (e-13) {$13$}
      edge[pointer-parent, bend right] (e-3);
    \node[elem, left = 1 of e-13] (e-8) {$3$}
      edge[pointer-parent, bend left] (e-3);
    \draw[pointer-sibling] (e-13.east) -- (e-7.west);
    \node[elem, below = 1 of e-13] (e-21) {$21$}
      edge[pointer-parent, bend right] (e-13);
    \draw[pointer-child] (e-13.south) -- (e-21.north);
    \draw[pointer-child] (e-3.south west) -- (e-8.north east);
    \draw[pointer-sibling] (e-8.east) -- (e-13.west);
    \node[elem, below = 1 of e-8] (e-14) {$14$}
      edge[pointer-parent, bend right] (e-8);
    \node[elem, left = 1 of e-14] (e-15) {$8$}
      edge[pointer-parent, bend right] (e-8);
    \draw[pointer-sibling] (e-15.east) -- (e-14.west);
    \node[elem, below = 1 of e-15] (e-18) {$18$}
      edge[pointer-parent, bend right] (e-15);
    \draw[pointer-child] (e-8.south west) -- (e-15.north east);
    \draw[pointer-child] (e-15.south) -- (e-18.north);

    \node[right = 1 of e-7] (deletemin) {$\Longrightarrow$};
    \node[above = 0 of deletemin] (deletemin-t) {$deletemin$};

    \node[elem, right = 3 of e-3] (e-27) {$27$};

    \node[elem, right = 1 of e-27] (e-7) {$7$};
    \node[elem, right = 1 of e-7] (e-13) {$13$};
    \node[elem, below = 1 of e-13] (e-21) {$21$}
      edge[pointer-parent, bend right] (e-13);
    \draw[pointer-child] (e-13.south) -- (e-21.north);
    \draw[pointer-sibling] (e-7.east) -- (e-13.west);
    \node[elem, right = 3 of e-13] (e-3) {$3$};
    \node[elem, below = 1 of e-3] (e-14) {$14$}
      edge[pointer-parent, bend right] (e-3);
    \node[elem, left = 1 of e-14] (e-8) {$8$}
      edge[pointer-parent, bend right] (e-3);
    \draw[pointer-sibling] (e-8.east) -- (e-14.west);
    \node[elem, below = 1 of e-8] (e-18) {$18$}
      edge[pointer-parent, bend right] (e-8);
    \draw[pointer-child] (e-8.south) -- (e-18.north);
    \draw[pointer-child] (e-3.south west) -- (e-8.north east);
    \draw[pointer-sibling] (e-13.east) -- (e-3.west);

    \node[below = 1.5 of e-18] (meld-1) {$\Downarrow$};
    \node[right = 0 of meld-1] (meld-1-t) {$meld$};

    \node[elem, below = 6.5 of e-7] (e-7) {$7$};
    \node[elem, below = 1 of e-7] (e-27) {$27$}
      edge[pointer-parent, bend right] (e-7);
    \draw[pointer-child] (e-7.south) -- (e-27.north);
    \node[elem, right = 1 of e-7] (e-13) {$13$};
    \node[elem, below = 1 of e-13] (e-21) {$21$}
      edge[pointer-parent, bend right] (e-13);
    \draw[pointer-child] (e-13.south) -- (e-21.north);
    \draw[pointer-sibling] (e-7.east) -- (e-13.west);
    \node[elem, right = 3 of e-13] (e-3) {$3$};
    \node[elem, below = 1 of e-3] (e-14) {$14$}
      edge[pointer-parent, bend right] (e-3);
    \node[elem, left = 1 of e-14] (e-8) {$8$}
      edge[pointer-parent, bend right] (e-3);
    \draw[pointer-child] (e-3.south west) -- (e-8.north east);
    \node[elem, below = 1 of e-8] (e-18) {$18$}
      edge[pointer-parent, bend right] (e-8);
    \draw[pointer-child] (e-8.south) -- (e-18.north);
    \draw[pointer-sibling] (e-8.east) -- (e-14.west);
    \draw[pointer-sibling] (e-13.east) -- (e-3.west);
  
    \node[left = 1 of e-27] (meld-2) {$\Longleftarrow$};

    \node[elem, left = 8 of e-7] (e-7) {$7$};
    \node[elem, below = 1 of e-7] (e-27) {$27$}
      edge[pointer-parent, bend right] (e-7);
    \node[elem, left = 1 of e-27] (e-13) {$13$}
      edge[pointer-parent, bend right] (e-7);
    \draw[pointer-child] (e-7.south west) -- (e-13.north east);
    \node[elem, below = 1 of e-13] (e-21) {$21$}
      edge[pointer-parent, bend right] (e-13);
    \draw[pointer-child] (e-13.south) -- (e-21.north);
    \draw[pointer-sibling] (e-13.east) -- (e-27.west);
    \node[elem, right = 3 of e-7] (e-3) {$3$};
    \node[elem, below = 1 of e-3] (e-14) {$14$}
      edge[pointer-parent, bend right] (e-3);
    \node[elem, left = 1 of e-14] (e-8) {$8$}
      edge[pointer-parent, bend right] (e-3);
    \draw[pointer-child] (e-3.south west) -- (e-8.north east);
    \node[elem, below = 1 of e-8] (e-18) {$18$}
      edge[pointer-parent, bend right] (e-8);
    \draw[pointer-child] (e-8.south) -- (e-18.north);
    \draw[pointer-sibling] (e-8.east) -- (e-14.west);
    \draw[pointer-sibling] (e-7.east) -- (e-3.west);

    \node[below = 1 of e-18] (meld-3) {$\Downarrow$};

    \node[elem, below = 6 of e-3] (e-3) {$3$};
    \node[elem, below = 1 of e-3] (e-14) {$14$}
      edge[pointer-parent, bend right] (e-3);
    \node[elem, left = 1 of e-14] (e-8) {$8$}
      edge[pointer-parent, bend right] (e-3);
    \draw[pointer-sibling] (e-8.east) -- (e-14.west);
    \node[elem, left = 1 of e-8] (e-7) {$7$}
      edge[pointer-parent, bend left] (e-3);
    \draw[pointer-sibling] (e-7.east) -- (e-8.west);
    \draw[pointer-child] (e-3.south west) -- (e-7.north east);
    \node[elem, below = 1 of e-8] (e-18) {$18$}
      edge[pointer-parent, bend right] (e-8);
    \draw[pointer-child] (e-8.south) -- (e-18.north);
    \node[elem, left = 1 of e-18] (e-27) {$27$}
      edge[pointer-parent, bend right] (e-7);
    \node[elem, left = 1 of e-27] (e-13) {$13$}
      edge[pointer-parent, bend right] (e-7);
    \draw[pointer-sibling] (e-13.east) -- (e-27.west);
    \draw[pointer-child] (e-7.south west) -- (e-13.north east);
    \node[elem, below = 1 of e-13] (e-21) {$21$}
      edge[pointer-parent, bend right] (e-13);
    \draw[pointer-child] (e-13.south) -- (e-21.north);
  \end{tikzpicture}
  \caption{
    \texttt{deletemin()} ($-\infty$ hat minimalen Schl{\"u}ssel in der
    Wurzelliste $\Rightarrow$ entferne $-\infty$ aus $Q$ (liefert $Q'$); Drehe
    Reihenfolge der S{\"o}hne von $-\infty$ um (liefert $Q''$); $Q'.meld(Q'')$)
  } 
\end{figure}

\begin{figure}[H]
  \centering
  \begin{tikzpicture}[scale = 0.8, every node/.style={scale=0.8}]
    \node[elem] (e-3) {$3$};
    \node[elem, below = 1 of e-3] (e-14) {$14$}
      edge[pointer-parent, bend right] (e-3);
    \node[elem, left = 1 of e-14] (e-8) {$8$}
      edge[pointer-parent, bend right] (e-3);
    \draw[pointer-sibling] (e-8.east) -- (e-14.west);
    \node[elem, left = 1 of e-8] (e-7) {$7$}
      edge[pointer-parent, bend left] (e-3);
    \draw[pointer-sibling] (e-7.east) -- (e-8.west);
    \draw[pointer-child] (e-3.south west) -- (e-7.north east);
    \node[elem, below = 1 of e-8] (e-18) {$18$}
      edge[pointer-parent, bend right] (e-8);
    \draw[pointer-child] (e-8.south) -- (e-18.north);
    \node[elem, left = 1 of e-18] (e-27) {$-\infty$}
      edge[pointer-parent, bend right] (e-7);
    \node[elem, left = 1 of e-27] (e-13) {$13$}
      edge[pointer-parent, bend right] (e-7);
    \draw[pointer-sibling] (e-13.east) -- (e-27.west);
    \draw[pointer-child] (e-7.south west) -- (e-13.north east);
    \node[elem, below = 1 of e-13] (e-21) {$21$}
      edge[pointer-parent, bend right] (e-13);
    \draw[pointer-child] (e-13.south) -- (e-21.north);

    \node[right = 1 of e-14] {$\Longrightarrow$};

    \node[elem, right = 8 of e-3] (e-3) {$3$};
    \node[elem, below = 1 of e-3] (e-14) {$14$}
      edge[pointer-parent, bend right] (e-3);
    \node[elem, left = 1 of e-14] (e-8) {$8$}
      edge[pointer-parent, bend right] (e-3);
    \draw[pointer-sibling] (e-8.east) -- (e-14.west);
    \node[elem, left = 1 of e-8] (e-7) {$-\infty$}
      edge[pointer-parent, bend left] (e-3);
    \draw[pointer-sibling] (e-7.east) -- (e-8.west);
    \draw[pointer-child] (e-3.south west) -- (e-7.north east);
    \node[elem, below = 1 of e-8] (e-18) {$18$}
      edge[pointer-parent, bend right] (e-8);
    \draw[pointer-child] (e-8.south) -- (e-18.north);
    \node[elem, left = 1 of e-18] (e-27) {$7$}
      edge[pointer-parent, bend right] (e-7);
    \node[elem, left = 1 of e-27] (e-13) {$13$}
      edge[pointer-parent, bend right] (e-7);
    \draw[pointer-sibling] (e-13.east) -- (e-27.west);
    \draw[pointer-child] (e-7.south west) -- (e-13.north east);
    \node[elem, below = 1 of e-13] (e-21) {$21$}
      edge[pointer-parent, bend right] (e-13);
    \draw[pointer-child] (e-13.south) -- (e-21.north);

    \node[below = 2.5 of e-18] {$\Downarrow$};

    \node[elem, below = 7 of e-3] (e-3) {$-\infty$};
    \node[elem, below = 1 of e-3] (e-14) {$14$}
      edge[pointer-parent, bend right] (e-3);
    \node[elem, left = 1 of e-14] (e-8) {$8$}
      edge[pointer-parent, bend right] (e-3);
    \draw[pointer-sibling] (e-8.east) -- (e-14.west);
    \node[elem, left = 1 of e-8] (e-7) {$3$}
      edge[pointer-parent, bend left] (e-3);
    \draw[pointer-sibling] (e-7.east) -- (e-8.west);
    \draw[pointer-child] (e-3.south west) -- (e-7.north east);
    \node[elem, below = 1 of e-8] (e-18) {$18$}
      edge[pointer-parent, bend right] (e-8);
    \draw[pointer-child] (e-8.south) -- (e-18.north);
    \node[elem, left = 1 of e-18] (e-27) {$7$}
      edge[pointer-parent, bend right] (e-7);
    \node[elem, left = 1 of e-27] (e-13) {$13$}
      edge[pointer-parent, bend right] (e-7);
    \draw[pointer-sibling] (e-13.east) -- (e-27.west);
    \draw[pointer-child] (e-7.south west) -- (e-13.north east);
    \node[elem, below = 1 of e-13] (e-21) {$21$}
      edge[pointer-parent, bend right] (e-13);
    \draw[pointer-child] (e-13.south) -- (e-21.north);
  \end{tikzpicture}
  \caption{
    L{\"o}schen von $27$: Ersetze $27$ mit $-\infty$ und lasse diesen Knoten nach
    oben wandern.
  } 
\end{figure}

\begin{figure}[H]
  \centering
  \begin{tikzpicture}[scale = 0.8, every node/.style={scale=0.8}]
    \node[elem, below = 7 of e-3] (e-3) {$-\infty$};
    \node[elem, below = 1 of e-3] (e-14) {$14$}
      edge[pointer-parent, bend right] (e-3);
    \node[elem, left = 1 of e-14] (e-8) {$8$}
      edge[pointer-parent, bend right] (e-3);
    \draw[pointer-sibling] (e-8.east) -- (e-14.west);
    \node[elem, left = 1 of e-8] (e-7) {$3$}
      edge[pointer-parent, bend left] (e-3);
    \draw[pointer-sibling] (e-7.east) -- (e-8.west);
    \draw[pointer-child] (e-3.south west) -- (e-7.north east);
    \node[elem, below = 1 of e-8] (e-18) {$18$}
      edge[pointer-parent, bend right] (e-8);
    \draw[pointer-child] (e-8.south) -- (e-18.north);
    \node[elem, left = 1 of e-18] (e-27) {$7$}
      edge[pointer-parent, bend right] (e-7);
    \node[elem, left = 1 of e-27] (e-13) {$13$}
      edge[pointer-parent, bend right] (e-7);
    \draw[pointer-sibling] (e-13.east) -- (e-27.west);
    \draw[pointer-child] (e-7.south west) -- (e-13.north east);
    \node[elem, below = 1 of e-13] (e-21) {$21$}
      edge[pointer-parent, bend right] (e-13);
    \draw[pointer-child] (e-13.south) -- (e-21.north);

    \node[right = 1 of e-14] (deletemin) {$\Longrightarrow$};
    \node[above = 0 of deletemin] (deletemin-t) {$deletemin$};

    \node[elem, right = 3 of e-3] (e-14) {$14$};
    \node[elem, right = 1 of e-14] (e-8) {$8$};
    \draw[pointer-sibling] (e-14.east) -- (e-8.west);
    \node[elem, below = 1 of e-8] (e-18) {$18$}
      edge[pointer-parent, bend right] (e-8);
    \draw[pointer-child] (e-8.south) -- (e-18.north);
    \node[elem, right = 3 of e-8] (e-3) {$3$};
    \draw[pointer-sibling] (e-8.east) -- (e-3.west);
    \node[elem, below = 1 of e-3] (e-7) {$7$}
      edge[pointer-parent, bend right] (e-3);
    \node[elem, left = 1 of e-7] (e-13) {$13$}
      edge[pointer-parent, bend right] (e-3);
    \draw[pointer-sibling] (e-13.east) -- (e-7.west);
    \draw[pointer-child] (e-3.south west) -- (e-13.north east);
    \node[elem, below = 1 of e-13] (e-21) {$21$}
      edge[pointer-parent, bend right] (e-13);
    \draw[pointer-child] (e-13.south) -- (e-21.north);
  \end{tikzpicture}
  \caption{
    \texttt{deletemin()} ($-\infty$ hat minimalen Schl{\"u}ssel in der
    Wurzelliste $\Rightarrow$ entferne $-\infty$ aus $Q$ (liefert $Q'$); Drehe
    Reihenfolge der S{\"o}hne von $-\infty$ um (liefert $Q''$); $Q'.meld(Q'')$).
    In diesem Fall ist $Q'$ leer, wodurch nurmehr $Q''$ bleibt.
  } 
\end{figure}

\begin{figure}[H]
  \centering
  \begin{tikzpicture}[scale = 0.8, every node/.style={scale=0.8}]
    \node[elem] (e-14) {$14$};
    \node[elem, right = 1 of e-14] (e-8) {$8$};
    \draw[pointer-sibling] (e-14.east) -- (e-8.west);
    \node[elem, below = 1 of e-8] (e-18) {$18$}
      edge[pointer-parent, bend right] (e-8);
    \draw[pointer-child] (e-8.south) -- (e-18.north);
    \node[elem, right = 3 of e-8] (e-3) {$3$};
    \draw[pointer-sibling] (e-8.east) -- (e-3.west);
    \node[elem, below = 1 of e-3] (e-7) {$7$}
      edge[pointer-parent, bend right] (e-3);
    \node[elem, left = 1 of e-7] (e-13) {$13$}
      edge[pointer-parent, bend right] (e-3);
    \draw[pointer-sibling] (e-13.east) -- (e-7.west);
    \draw[pointer-child] (e-3.south west) -- (e-13.north east);
    \node[elem, below = 1 of e-13] (e-21) {$21$}
      edge[pointer-parent, bend right] (e-13);
    \draw[pointer-child] (e-13.south) -- (e-21.north);

    \node[right = 1 of e-7] {$\Longrightarrow$};

    \node[elem, right = 3 of e-3] (e-14) {$14$};
    \node[elem, right = 1 of e-14] (e-8) {$8$};
    \draw[pointer-sibling] (e-14.east) -- (e-8.west);
    \node[elem, below = 1 of e-8] (e-4) {$4$}
      edge[pointer-parent, bend right] (e-8);
    \draw[pointer-child] (e-8.south) -- (e-4.north);
    \node[elem, right = 3 of e-8] (e-3) {$3$};
    \draw[pointer-sibling] (e-8.east) -- (e-3.west);
    \node[elem, below = 1 of e-3] (e-7) {$7$}
      edge[pointer-parent, bend right] (e-3);
    \node[elem, left = 1 of e-7] (e-13) {$13$}
      edge[pointer-parent, bend right] (e-3);
    \draw[pointer-sibling] (e-13.east) -- (e-7.west);
    \draw[pointer-child] (e-3.south west) -- (e-13.north east);
    \node[elem, below = 1 of e-13] (e-21) {$21$}
      edge[pointer-parent, bend right] (e-13);
    \draw[pointer-child] (e-13.south) -- (e-21.north);  

    \node[below = 1 of e-21]  {$\Downarrow$};

    \node[elem, below = 5.5 of e-14] (e-14) {$14$};
    \node[elem, right = 1 of e-14] (e-8) {$4$};
    \draw[pointer-sibling] (e-14.east) -- (e-8.west);
    \node[elem, below = 1 of e-8] (e-4) {$8$}
      edge[pointer-parent, bend right] (e-8);
    \draw[pointer-child] (e-8.south) -- (e-4.north);
    \node[elem, right = 3 of e-8] (e-3) {$3$};
    \draw[pointer-sibling] (e-8.east) -- (e-3.west);
    \node[elem, below = 1 of e-3] (e-7) {$7$}
      edge[pointer-parent, bend right] (e-3);
    \node[elem, left = 1 of e-7] (e-13) {$13$}
      edge[pointer-parent, bend right] (e-3);
    \draw[pointer-sibling] (e-13.east) -- (e-7.west);
    \draw[pointer-child] (e-3.south west) -- (e-13.north east);
    \node[elem, below = 1 of e-13] (e-21) {$21$}
      edge[pointer-parent, bend right] (e-13);
    \draw[pointer-child] (e-13.south) -- (e-21.north);  
  \end{tikzpicture}
  \caption{
    \texttt{decreasekey(18, 4)} (1. $v.entry.key = k$; 2. $v.entry$ nach oben
    steigen lassen in dem geg. Baum, bis die Heapbedingung erf{\"u}llt ist). In
    diesem Fall ersetzen wir $18$ durch $4$ und lassen $4$ um eine Ebene nach
    oben steigen.
  } 
\end{figure}

\clearpage

{\bfseries Aufgabe 18}%

F{\"u}gen Sie die Elemente $13$, $21$, $3$, $7$, $15$, $18$, $8$, $14$ und $27$
in einen (anfangs leeren) Fibonacci-Heap ein. Entfernen Sie anschlie{\ss}end das
Minimum und geben Sie den resultierenden Fibonacci-Heap an. Dekrementieren Sie
zum Schluss den Wert eines jeden Schl{\"u}ssels (Schritt f{\"u}r Schritt) um $7$
und geben Sie die resultierende Datenstruktur an.

\tikzstyle{elem}=[draw, circle, thick, fill=blue!20, minimum size=10mm]
\tikzstyle{elem-marked}=[elem, fill=red!20]
\tikzstyle{pointer}=[->, >=stealth, thick, solid]
\tikzstyle{pointer-child}=[pointer, black]
\tikzstyle{pointer-parent}=[pointer, red]
\tikzstyle{pointer-left}=[pointer, blue]
\tikzstyle{pointer-right}=[pointer, green!65!black]
\tikzstyle{pointer-min}=[pointer, <-, ultra thick]

Im Folgenden werden die einzelnen Schritte dargestellt, wobei f{\"u}r die
Child-Parent-Left-Right Darstellung die folgenden Pfeile und Knoten verwendet
werden:
\begin{figure}[H]
  \centering
  \begin{tikzpicture}
    \draw[pointer-child] (0, 0) -- ++(0.5, 0) node[right] (child) {\ldots Child-/Min-Pointer};
    \draw[pointer-parent] (5, 0) -- ++(0.5, 0) node[right] {\ldots Parent-Pointer};
    \draw[pointer-left] (9, 0) -- ++(0.5, 0) node[right] (left) {\ldots Left-Pointer};
    \draw[pointer-right] (13, 0) -- ++(0.5, 0) node[right] {\ldots Right-Pointer};
    \node[elem, below left = 0.5 and 0 of child] (normal-node) {$x$};
    \node[right = 0 of normal-node] {\ldots Normaler Knoten};
    \node[elem-marked, below  left = 0.5 and 0 of left] (marked-node) {$x$};
    \node[right = 0 of marked-node] {\ldots Markierter Knoten};
  \end{tikzpicture}
\end{figure}

Beim Einf{\"u}gen wurde immer rechts vom aktuellen Minimum eingef{\"u}gt, da
ansonsten dies nicht in $O(1)$ m{\"o}glich w{\"a}re (es ist in den Folien zur VO
nicht genau spezifiziert, an welcher Position eingef{\"u}gt wird; daher diese
Annahme).

\begin{figure}[H]
  \centering
  \begin{tikzpicture}[scale = 0.9, every node/.style={scale = 0.9}]
    \node[elem] (e-13) {$13$};
    \draw[pointer-min] (e-13.north) -- ++(0, 0.5);
    \path[pointer-left] (e-13.west) edge[in=20, out=160, looseness=7] (e-13.east);
    \path[pointer-right] (e-13.east) edge[in=-160, out=-20, looseness=7] (e-13.west);

    \node[right = 0.75 of e-13] {$\Longrightarrow$};

    \node[elem, right = 3 of e-13] (e-13) {$13$};
    \node[elem, right = 1 of e-13] (e-21) {$21$};
    \draw[pointer-min] (e-13.north) -- ++(0, 0.5);
    \path[pointer-left] (e-13.west) edge[in=20, out=160, looseness=2.5] (e-21.east);
    \path[pointer-right] (e-13.east) edge[bend right] (e-21.west);
    \path[pointer-left] (e-21.west) edge[bend right] (e-13.east);
    \path[pointer-right] (e-21.east) edge[in=-160, out=-20, looseness=2.5] (e-13.west);

    \node[right = 0.75 of e-21] {$\Longrightarrow$};

    \node[elem, right = 2 of e-21] (e-13) {$13$};
    \node[elem, right = 1 of e-13] (e-3) {$3$};
    \node[elem, right = 1 of e-3] (e-21) {$21$};
    \draw[pointer-min] (e-3.north) -- ++(0, 0.5);
    \path[pointer-left] (e-13.west) edge[in=20, out=160, looseness=1.65] (e-21.east);
    \path[pointer-right] (e-13.east) edge[bend right] (e-3.west);
    \path[pointer-left] (e-3.west) edge[bend right] (e-13.east);
    \path[pointer-right] (e-3.east) edge[bend right] (e-21.west);
    \path[pointer-left] (e-21.west) edge[bend right] (e-3.east);
    \path[pointer-right] (e-21.east) edge[in=-160, out=-20, looseness=1.65] (e-13.west);
  \end{tikzpicture}
  \caption{Einf{\"u}gen von $13$, $21$, $3$.}
\end{figure}

\begin{figure}[H]
  \centering
  \begin{tikzpicture}[scale = 0.7, every node/.style={scale = 0.7}, trim left = -1cm]
    \node[elem] (e-13) {$13$};
    \node[elem, right = 1 of e-13] (e-3) {$3$};
    \node[elem, right = 1 of e-3] (e-7) {$7$};
    \node[elem, right = 1 of e-7] (e-21) {$21$};
    \draw[pointer-min] (e-3.north) -- ++(0, 0.5);
    \path[pointer-left] (e-13.west) edge[in=20, out=160, looseness=1] (e-21.east);
    \path[pointer-right] (e-13.east) edge[bend right] (e-3.west);
    \path[pointer-left] (e-3.west) edge[bend right] (e-13.east);
    \path[pointer-right] (e-3.east) edge[bend right] (e-7.west);
    \path[pointer-left] (e-7.west) edge[bend right] (e-3.east);
    \path[pointer-right] (e-7.east) edge[bend right] (e-21.west);
    \path[pointer-left] (e-21.west) edge[bend right] (e-7.east);
    \path[pointer-right] (e-21.east) edge[in=-160, out=-20, looseness=1] (e-13.west);

    \node[right = 0.75 of e-21] {$\Longrightarrow$};

    \node[elem, right = 2 of e-21] (e-13) {$13$};
    \node[elem, right = 1 of e-13] (e-3) {$3$};
    \node[elem, right = 1 of e-3] (e-15) {$15$};
    \node[elem, right = 1 of e-15] (e-7) {$7$};
    \node[elem, right = 1 of e-7] (e-21) {$21$};
    \draw[pointer-min] (e-3.north) -- ++(0, 0.5);
    \path[pointer-left] (e-13.west) edge[in=20, out=160, looseness=0.75] (e-21.east);
    \path[pointer-right] (e-13.east) edge[bend right] (e-3.west);
    \path[pointer-left] (e-3.west) edge[bend right] (e-13.east);
    \path[pointer-right] (e-3.east) edge[bend right] (e-15.west);
    \path[pointer-left] (e-15.west) edge[bend right] (e-3.east);
    \path[pointer-right] (e-15.east) edge[bend right] (e-7.west);
    \path[pointer-left] (e-7.west) edge[bend right] (e-15.east);
    \path[pointer-right] (e-7.east) edge[bend right] (e-21.west);
    \path[pointer-left] (e-21.west) edge[bend right] (e-7.east);
    \path[pointer-right] (e-21.east) edge[in=-160, out=-20, looseness=0.75] (e-13.west);
  \end{tikzpicture}
  \caption{Einf{\"u}gen von $7$ und $15$.}
\end{figure}

\begin{figure}[H]
  \centering
  \begin{tikzpicture}
    \node[elem] (e-13) {$13$};
    \node[elem, right = 1 of e-13] (e-3) {$3$};
    \node[elem, right = 1 of e-3] (e-18) {$18$};
    \node[elem, right = 1 of e-18] (e-15) {$15$};
    \node[elem, right = 1 of e-15] (e-7) {$7$};
    \node[elem, right = 1 of e-7] (e-21) {$21$};
    \draw[pointer-min] (e-3.north) -- ++(0, 0.5);
    \path[pointer-left] (e-13.west) edge[in=20, out=160, looseness=0.75] (e-21.east);
    \path[pointer-right] (e-13.east) edge[bend right] (e-3.west);
    \path[pointer-left] (e-3.west) edge[bend right] (e-13.east);
    \path[pointer-right] (e-3.east) edge[bend right] (e-18.west);
    \path[pointer-left] (e-18.west) edge[bend right] (e-3.east);
    \path[pointer-right] (e-18.east) edge[bend right] (e-15.west);
    \path[pointer-left] (e-15.west) edge[bend right] (e-18.east);
    \path[pointer-right] (e-15.east) edge[bend right] (e-7.west);
    \path[pointer-left] (e-7.west) edge[bend right] (e-15.east);
    \path[pointer-right] (e-7.east) edge[bend right] (e-21.west);
    \path[pointer-left] (e-21.west) edge[bend right] (e-7.east);
    \path[pointer-right] (e-21.east) edge[in=-160, out=-20, looseness=0.75] (e-13.west);
  \end{tikzpicture}
  \caption{Nach Einf{\"u}gen von $18$.}
\end{figure}

\begin{figure}[H]
  \centering
  \begin{tikzpicture}[trim left = -2cm]
    \node[elem] (e-13) {$13$};
    \node[elem, right = 1 of e-13] (e-3) {$3$};
    \node[elem, right = 1 of e-3] (e-8) {$8$};
    \node[elem, right = 1 of e-8] (e-18) {$18$};
    \node[elem, right = 1 of e-18] (e-15) {$15$};
    \node[elem, right = 1 of e-15] (e-7) {$7$};
    \node[elem, right = 1 of e-7] (e-21) {$21$};
    \draw[pointer-min] (e-3.north) -- ++(0, 0.5);
    \path[pointer-left] (e-13.west) edge[in=20, out=160, looseness=0.75] (e-21.east);
    \path[pointer-right] (e-13.east) edge[bend right] (e-3.west);
    \path[pointer-left] (e-3.west) edge[bend right] (e-13.east);
    \path[pointer-right] (e-3.east) edge[bend right] (e-8.west);
    \path[pointer-left] (e-8.west) edge[bend right] (e-3.east);
    \path[pointer-right] (e-8.east) edge[bend right] (e-18.west);
    \path[pointer-left] (e-18.west) edge[bend right] (e-8.east);
    \path[pointer-right] (e-18.east) edge[bend right] (e-15.west);
    \path[pointer-left] (e-15.west) edge[bend right] (e-18.east);
    \path[pointer-right] (e-15.east) edge[bend right] (e-7.west);
    \path[pointer-left] (e-7.west) edge[bend right] (e-15.east);
    \path[pointer-right] (e-7.east) edge[bend right] (e-21.west);
    \path[pointer-left] (e-21.west) edge[bend right] (e-7.east);
    \path[pointer-right] (e-21.east) edge[in=-160, out=-20, looseness=0.75] (e-13.west);
  \end{tikzpicture}
  \caption{Nach Einf{\"u}gen von $8$.}
\end{figure}

\begin{figure}[H]
  \centering
  \begin{tikzpicture}[trim left = -1.5cm]
    \node[elem] (e-13) {$13$};
    \node[elem, right = 1 of e-13] (e-3) {$3$};
    \node[elem, right = 1 of e-3] (e-14) {$14$};
    \node[elem, right = 1 of e-14] (e-8) {$8$};
    \node[elem, right = 1 of e-8] (e-18) {$18$};
    \node[elem, right = 1 of e-18] (e-15) {$15$};
    \node[elem, right = 1 of e-15] (e-7) {$7$};
    \node[elem, right = 1 of e-7] (e-21) {$21$};
    \draw[pointer-min] (e-3.north) -- ++(0, 0.5);
    \path[pointer-left] (e-13.west) edge[in=20, out=160, looseness=0.75] (e-21.east);
    \path[pointer-right] (e-13.east) edge[bend right] (e-3.west);
    \path[pointer-left] (e-3.west) edge[bend right] (e-13.east);
    \path[pointer-right] (e-3.east) edge[bend right] (e-14.west);
    \path[pointer-left] (e-14.west) edge[bend right] (e-3.east);
    \path[pointer-right] (e-14.east) edge[bend right] (e-8.west);
    \path[pointer-left] (e-8.west) edge[bend right] (e-14.east);
    \path[pointer-right] (e-8.east) edge[bend right] (e-18.west);
    \path[pointer-left] (e-18.west) edge[bend right] (e-8.east);
    \path[pointer-right] (e-18.east) edge[bend right] (e-15.west);
    \path[pointer-left] (e-15.west) edge[bend right] (e-18.east);
    \path[pointer-right] (e-15.east) edge[bend right] (e-7.west);
    \path[pointer-left] (e-7.west) edge[bend right] (e-15.east);
    \path[pointer-right] (e-7.east) edge[bend right] (e-21.west);
    \path[pointer-left] (e-21.west) edge[bend right] (e-7.east);
    \path[pointer-right] (e-21.east) edge[in=-160, out=-20, looseness=0.75] (e-13.west);
  \end{tikzpicture}
  \caption{Nach Einf{\"u}gen von $14$.}
\end{figure}

\begin{figure}[H]
  \centering
  \begin{tikzpicture}[scale=0.9, every node/.style={scale=0.9}, trim left = -1cm]
    \node[elem] (e-13) {$13$};
    \node[elem, right = 1 of e-13] (e-3) {$3$};
    \node[elem, right = 1 of e-3] (e-27) {$27$};
    \node[elem, right = 1 of e-27] (e-14) {$14$};
    \node[elem, right = 1 of e-14] (e-8) {$8$};
    \node[elem, right = 1 of e-8] (e-18) {$18$};
    \node[elem, right = 1 of e-18] (e-15) {$15$};
    \node[elem, right = 1 of e-15] (e-7) {$7$};
    \node[elem, right = 1 of e-7] (e-21) {$21$};
    \draw[pointer-min] (e-3.north) -- ++(0, 0.5);
    \path[pointer-left] (e-13.west) edge[in=20, out=160, looseness=0.75] (e-21.east);
    \path[pointer-right] (e-13.east) edge[bend right] (e-3.west);
    \path[pointer-left] (e-3.west) edge[bend right] (e-13.east);
    \path[pointer-right] (e-3.east) edge[bend right] (e-27.west);
    \path[pointer-left] (e-27.west) edge[bend right] (e-3.east);
    \path[pointer-right] (e-27.east) edge[bend right] (e-14.west);
    \path[pointer-left] (e-14.west) edge[bend right] (e-27.east);
    \path[pointer-right] (e-14.east) edge[bend right] (e-8.west);
    \path[pointer-left] (e-8.west) edge[bend right] (e-14.east);
    \path[pointer-right] (e-8.east) edge[bend right] (e-18.west);
    \path[pointer-left] (e-18.west) edge[bend right] (e-8.east);
    \path[pointer-right] (e-18.east) edge[bend right] (e-15.west);
    \path[pointer-left] (e-15.west) edge[bend right] (e-18.east);
    \path[pointer-right] (e-15.east) edge[bend right] (e-7.west);
    \path[pointer-left] (e-7.west) edge[bend right] (e-15.east);
    \path[pointer-right] (e-7.east) edge[bend right] (e-21.west);
    \path[pointer-left] (e-21.west) edge[bend right] (e-7.east);
    \path[pointer-right] (e-21.east) edge[in=-160, out=-20, looseness=0.75] (e-13.west);
  \end{tikzpicture}
  \caption{Nach Einf{\"u}gen von $27$.}
\end{figure}

%\clearpage

Im Zuge von \texttt{deletemin()} wird \texttt{consolidate()} aufgerufen. F{\"u}r
\texttt{consolidate()} ben{\"o}tigt man ein Array der Gr{\"o}{\ss}e
$\left\lceil 2\cdot\text{log}_2 n \right\rceil$. Entfernt man nun das Minimum
($3$), dann verbleiben $n = 8$ Knoten. Wir ben{\"o}tigen daher ein Array der
Gr{\"o}{\ss}e
$\left\lceil 2\cdot\text{log}_2 8 \right\rceil = \left\lceil 2\cdot 3 \right\rceil = 6$.

Im Folgenden werden die einzelnen \texttt{consolidate()}-Schritte dargestellt,
die nach dem Entfernen von $3$ notwendig sind. Punktierte Pointer vom Array zu
Knoten stehen dabei daf{\"u}r, dass die betroffene Stelle im Array bereits besetzt
ist. In diesem Fall werden die beiden Heaps, auf die die Pointer verweisen,
verschmolzen.

\texttt{consolidate()} startet dabei wiederum beim rechten Knoten des zuvor
entfernten Minimums - in diesem Fall also bei Knoten $27$.

\tikzstyle{elem-array}=[draw, rectangle, thick, minimum size=6mm]
\tikzstyle{pointer-conflict}=[pointer, dotted]

\begin{figure}[H]
  \centering
  \begin{tikzpicture}[trim left = -1cm]
    \node[elem] (e-13) {$13$};
    \node[elem, right = 1 of e-13] (e-27) {$27$};
    \node[elem, right = 1 of e-27] (e-14) {$14$};
    \node[elem, right = 1 of e-14] (e-8) {$8$};
    \node[elem, right = 1 of e-8] (e-18) {$18$};
    \node[elem, right = 1 of e-18] (e-15) {$15$};
    \node[elem, right = 1 of e-15] (e-7) {$7$};
    \node[elem, right = 1 of e-7] (e-21) {$21$};
    %\draw[pointer-min] (e-3.north) -- ++(0, 0.5);
    \path[pointer-left] (e-13.west) edge[in=20, out=160, looseness=0.75] (e-21.east);
    \path[pointer-right] (e-13.east) edge[bend right] (e-27.west);
    \path[pointer-left] (e-27.west) edge[bend right] (e-13.east);
    \path[pointer-right] (e-27.east) edge[bend right] (e-14.west);
    \path[pointer-left] (e-14.west) edge[bend right] (e-27.east);
    \path[pointer-right] (e-14.east) edge[bend right] (e-8.west);
    \path[pointer-left] (e-8.west) edge[bend right] (e-14.east);
    \path[pointer-right] (e-8.east) edge[bend right] (e-18.west);
    \path[pointer-left] (e-18.west) edge[bend right] (e-8.east);
    \path[pointer-right] (e-18.east) edge[bend right] (e-15.west);
    \path[pointer-left] (e-15.west) edge[bend right] (e-18.east);
    \path[pointer-right] (e-15.east) edge[bend right] (e-7.west);
    \path[pointer-left] (e-7.west) edge[bend right] (e-15.east);
    \path[pointer-right] (e-7.east) edge[bend right] (e-21.west);
    \path[pointer-left] (e-21.west) edge[bend right] (e-7.east);
    \path[pointer-right] (e-21.east) edge[in=-160, out=-20, looseness=0.75] (e-13.west);

    \node[elem-array, above = 2 of e-27] (array-0) {};
    \node[above = 0 of array-0] (array-0-t) {\footnotesize$0$};
    \foreach\i [count = \xi from 0] in {1, ..., 5}{
      \node[elem-array, right = -0.015 of array-\xi] (array-\i) {};
      \node[above = 0 of array-\i] (array-\i-t) {\footnotesize$\i$};
    }
    \draw[pointer] (array-0.center) -- (e-27.north);
    \draw[pointer-conflict] (array-0.center) -- (e-14.north);

    \node[below = 1 of e-8] {$\Downarrow$};

    \node[elem, below = 5 of e-27] (e-13) {$13$};
    \node[elem, right = 1 of e-13] (e-14) {$14$};
    \node[elem, below = 1 of e-14] (e-27) {$27$};
    \node[elem, right = 1 of e-14] (e-8) {$8$};
    \node[elem, right = 1 of e-8] (e-18) {$18$};
    \node[elem, right = 1 of e-18] (e-15) {$15$};
    \node[elem, right = 1 of e-15] (e-7) {$7$};
    \node[elem, right = 1 of e-7] (e-21) {$21$};
    %\draw[pointer-min] (e-3.north) -- ++(0, 0.5);
    \path[pointer-left] (e-13.west) edge[in=20, out=160, looseness=0.75] (e-21.east);
    \path[pointer-right] (e-13.east) edge[bend right] (e-14.west);
    \path[pointer-left] (e-14.west) edge[bend right] (e-13.east);
    \path[pointer-right] (e-14.east) edge[bend right] (e-8.west);
    \path[pointer-child] (e-14.south) edge[bend right] (e-27.north);
    \path[pointer-left] (e-27.west) edge[in=20, out=160, looseness=7] (e-27.east);
    \path[pointer-right] (e-27.east) edge[in=-160, out=-20, looseness=7] (e-27.west);
    \path[pointer-parent] (e-27.north) edge[bend right] (e-14.south);
    \path[pointer-left] (e-8.west) edge[bend right] (e-14.east);
    \path[pointer-right] (e-8.east) edge[bend right] (e-18.west);
    \path[pointer-left] (e-18.west) edge[bend right] (e-8.east);
    \path[pointer-right] (e-18.east) edge[bend right] (e-15.west);
    \path[pointer-left] (e-15.west) edge[bend right] (e-18.east);
    \path[pointer-right] (e-15.east) edge[bend right] (e-7.west);
    \path[pointer-left] (e-7.west) edge[bend right] (e-15.east);
    \path[pointer-right] (e-7.east) edge[bend right] (e-21.west);
    \path[pointer-left] (e-21.west) edge[bend right] (e-7.east);
    \path[pointer-right] (e-21.east) edge[in=-160, out=-20, looseness=0.75] (e-13.west);

    \node[elem-array, above = 2 of e-14] (array-0) {};
    \node[above = 0 of array-0] (array-0-t) {\footnotesize$0$};
    \foreach\i [count = \xi from 0] in {1, ..., 5}{
      \node[elem-array, right = -0.015 of array-\xi] (array-\i) {};
      \node[above = 0 of array-\i] (array-\i-t) {\footnotesize$\i$};
    }
    \draw[pointer] (array-1.center) -- (e-14.north);
    \draw[pointer] (array-0.center) -- (e-8.north);
    \draw[pointer-conflict] (array-0.center) -- (e-18.north);
  \end{tikzpicture}
  \caption{Nach Entfernen von $3$.}
\end{figure}

\clearpage

\begin{figure}[H]
  \centering
  \begin{tikzpicture}[trim left = -2.5cm]
    \node[elem] (e-13) {$13$};
    \node[elem, right = 1 of e-13] (e-14) {$14$};
    \node[elem, below = 1 of e-14] (e-27) {$27$};
    \node[elem, right = 2 of e-14] (e-8) {$8$};
    \node[elem, below = 1 of e-8] (e-18) {$18$};
    \node[elem, right = 1 of e-8] (e-15) {$15$};
    \node[elem, right = 1 of e-15] (e-7) {$7$};
    \node[elem, right = 1 of e-7] (e-21) {$21$};
    %\draw[pointer-min] (e-3.north) -- ++(0, 0.5);
    \path[pointer-left] (e-13.west) edge[in=20, out=160, looseness=0.75] (e-21.east);
    \path[pointer-right] (e-13.east) edge[bend right] (e-14.west);
    \path[pointer-left] (e-14.west) edge[bend right] (e-13.east);
    \path[pointer-right] (e-14.east) edge[bend right] (e-8.west);
    \path[pointer-child] (e-14.south) edge[bend right] (e-27.north);
    \path[pointer-left] (e-27.west) edge[in=20, out=160, looseness=7] (e-27.east);
    \path[pointer-right] (e-27.east) edge[in=-160, out=-20, looseness=7] (e-27.west);
    \path[pointer-parent] (e-27.north) edge[bend right] (e-14.south);
    \path[pointer-left] (e-8.west) edge[bend right] (e-14.east);
    \path[pointer-right] (e-8.east) edge[bend right] (e-15.west);
    \path[pointer-child] (e-8.south) edge[bend right] (e-18.north);
    \path[pointer-left] (e-18.west) edge[in=20, out=160, looseness=7] (e-18.east);
    \path[pointer-right] (e-18.east) edge[in=-160, out=-20, looseness=7] (e-18.west);
    \path[pointer-parent] (e-18.north) edge[bend right] (e-8.south);
    \path[pointer-left] (e-15.west) edge[bend right] (e-8.east);
    \path[pointer-right] (e-15.east) edge[bend right] (e-7.west);
    \path[pointer-left] (e-7.west) edge[bend right] (e-15.east);
    \path[pointer-right] (e-7.east) edge[bend right] (e-21.west);
    \path[pointer-left] (e-21.west) edge[bend right] (e-7.east);
    \path[pointer-right] (e-21.east) edge[in=-160, out=-20, looseness=0.75] (e-13.west);

    \node[elem-array, above = 2 of e-14] (array-0) {};
    \node[above = 0 of array-0] (array-0-t) {\footnotesize$0$};
    \foreach\i [count = \xi from 0] in {1, ..., 5}{
      \node[elem-array, right = -0.015 of array-\xi] (array-\i) {};
      \node[above = 0 of array-\i] (array-\i-t) {\footnotesize$\i$};
    }
    \draw[pointer] (array-1.center) -- (e-14.north);
    \draw[pointer-conflict] (array-1.center) -- (e-8.north);

    \node[below = 1 of e-18] {$\Downarrow$};

    \node[elem, below = 7 of e-14] (e-13) {$13$};
    \node[elem, right = 1 of e-13] (e-8) {$8$};
    \node[elem, below right = 1.5 and 0.5 of e-8] (e-14) {$14$};
    \node[elem, below = 1 of e-14] (e-27) {$27$};
    \node[elem, below left = 1.5 and 0.5 of e-8] (e-18) {$18$};
    \node[elem, right = 1 of e-8] (e-15) {$15$};
    \node[elem, right = 1 of e-15] (e-7) {$7$};
    \node[elem, right = 1 of e-7] (e-21) {$21$};
    %\draw[pointer-min] (e-3.north) -- ++(0, 0.5);
    \path[pointer-left] (e-13.west) edge[in=20, out=160, looseness=1] (e-21.east);
    \path[pointer-right] (e-13.east) edge[bend right] (e-8.west);
    \path[pointer-left] (e-14.west) edge[bend right] (e-18.east);
    \path[pointer-right] (e-14.east) edge[in=-160, out=-20, looseness=2.5] (e-18.west);
    \path[pointer-child] (e-14.south) edge[bend right] (e-27.north);
    \path[pointer-parent] (e-14.north) edge[bend right] (e-8.south east);
    \path[pointer-left] (e-27.west) edge[in=20, out=160, looseness=7] (e-27.east);
    \path[pointer-right] (e-27.east) edge[in=-160, out=-20, looseness=7] (e-27.west);
    \path[pointer-parent] (e-27.north) edge[bend right] (e-14.south);
    \path[pointer-left] (e-8.west) edge[bend right] (e-13.east);
    \path[pointer-right] (e-8.east) edge[bend right] (e-15.west);
    \path[pointer-child] (e-8.south west) edge[bend right] (e-18.north);
    \path[pointer-left] (e-18.west) edge[in=20, out=160, looseness=2.5] (e-14.east);
    \path[pointer-right] (e-18.east) edge[bend right] (e-14.west);
    \path[pointer-parent] (e-18.north) edge[bend right] (e-8.south west);
    \path[pointer-left] (e-15.west) edge[bend right] (e-8.east);
    \path[pointer-right] (e-15.east) edge[bend right] (e-7.west);
    \path[pointer-left] (e-7.west) edge[bend right] (e-15.east);
    \path[pointer-right] (e-7.east) edge[bend right] (e-21.west);
    \path[pointer-left] (e-21.west) edge[bend right] (e-7.east);
    \path[pointer-right] (e-21.east) edge[in=-160, out=-20, looseness=1] (e-13.west);

    \node[elem-array, above = 2 of e-8] (array-0) {};
    \node[above = 0 of array-0] (array-0-t) {\footnotesize$0$};
    \foreach\i [count = \xi from 0] in {1, ..., 5}{
      \node[elem-array, right = -0.015 of array-\xi] (array-\i) {};
      \node[above = 0 of array-\i] (array-\i-t) {\footnotesize$\i$};
    }
    \draw[pointer] (array-2.center) -- (e-8.north);
    \draw[pointer] (array-0.center) -- (e-15.north);
    \draw[pointer-conflict] (array-0.center) -- (e-7.north);
  \end{tikzpicture}
  \caption{Nach Entfernen von $3$ (Cont'd [1]).}
\end{figure}

\clearpage

\begin{figure}[H]
  \centering
  \begin{tikzpicture}[trim left = -2.5cm]
    \node[elem] (e-13) {$13$};
    \node[elem, right = 1 of e-13] (e-8) {$8$};
    \node[elem, below right = 1.5 and 0.5 of e-8] (e-14) {$14$};
    \node[elem, below = 1 of e-14] (e-27) {$27$};
    \node[elem, below left = 1.5 and 0.5 of e-8] (e-18) {$18$};
    \node[elem, right = 3 of e-8] (e-7) {$7$};
    \node[elem, below = 1.25 of e-7] (e-15) {$15$};
    \node[elem, right = 1 of e-7] (e-21) {$21$};
    %\draw[pointer-min] (e-3.north) -- ++(0, 0.5);
    \path[pointer-left] (e-13.west) edge[in=20, out=160, looseness=1] (e-21.east);
    \path[pointer-right] (e-13.east) edge[bend right] (e-8.west);
    \path[pointer-left] (e-14.west) edge[bend right] (e-18.east);
    \path[pointer-right] (e-14.east) edge[in=-160, out=-20, looseness=2.5] (e-18.west);
    \path[pointer-child] (e-14.south) edge[bend right] (e-27.north);
    \path[pointer-parent] (e-14.north) edge[bend right] (e-8.south east);
    \path[pointer-left] (e-27.west) edge[in=20, out=160, looseness=7] (e-27.east);
    \path[pointer-right] (e-27.east) edge[in=-160, out=-20, looseness=7] (e-27.west);
    \path[pointer-parent] (e-27.north) edge[bend right] (e-14.south);
    \path[pointer-left] (e-8.west) edge[bend right] (e-13.east);
    \path[pointer-right] (e-8.east) edge[bend right] (e-7.west);
    \path[pointer-child] (e-8.south west) edge[bend right] (e-18.north);
    \path[pointer-left] (e-18.west) edge[in=20, out=160, looseness=2.5] (e-14.east);
    \path[pointer-right] (e-18.east) edge[bend right] (e-14.west);
    \path[pointer-parent] (e-18.north) edge[bend right] (e-8.south west);
    \path[pointer-parent] (e-15.north) edge[bend right] (e-7.south);
    \path[pointer-left] (e-15.west) edge[in=20, out=160, looseness=7] (e-15.east);
    \path[pointer-right] (e-15.east) edge[in=-160, out=-20, looseness=7] (e-15.west);
    \path[pointer-left] (e-7.west) edge[bend right] (e-8.east);
    \path[pointer-right] (e-7.east) edge[bend right] (e-21.west);
    \path[pointer-child] (e-7.south) edge[bend right] (e-15.north);
    \path[pointer-left] (e-21.west) edge[bend right] (e-7.east);
    \path[pointer-right] (e-21.east) edge[in=-160, out=-20, looseness=1] (e-13.west);

    \node[elem-array, above = 2 of e-8] (array-0) {};
    \node[above = 0 of array-0] (array-0-t) {\footnotesize$0$};
    \foreach\i [count = \xi from 0] in {1, ..., 5}{
      \node[elem-array, right = -0.015 of array-\xi] (array-\i) {};
      \node[above = 0 of array-\i] (array-\i-t) {\footnotesize$\i$};
    }
    \draw[pointer] (array-2.center) -- (e-8.north);
    \draw[pointer] (array-1.center) -- (e-7.north);
    \draw[pointer] (array-0.center) -- (e-21.north west);
    \draw[pointer-conflict] (array-0.center) -- (e-13.north);

    \node[below = 0.5 of e-27] {$\Downarrow$};

    \node[elem, below = 8.5 of e-8] (e-8) {$8$};
    \node[elem, below right = 1.5 and 0.5 of e-8] (e-14) {$14$};
    \node[elem, below = 1 of e-14] (e-27) {$27$};
    \node[elem, below left = 1.5 and 0.5 of e-8] (e-18) {$18$};
    \node[elem, right = 3 of e-8] (e-7) {$7$};
    \node[elem, below = 1.25 of e-7] (e-15) {$15$};
    \node[elem, right = 2 of e-7] (e-13) {$13$};
    \node[elem, below = 1.25 of e-13] (e-21) {$21$};
    %\draw[pointer-min] (e-3.north) -- ++(0, 0.5);
    \path[pointer-left] (e-14.west) edge[bend right] (e-18.east);
    \path[pointer-right] (e-14.east) edge[in=-160, out=-20, looseness=2.5] (e-18.west);
    \path[pointer-child] (e-14.south) edge[bend right] (e-27.north);
    \path[pointer-parent] (e-14.north) edge[bend right] (e-8.south east);
    \path[pointer-left] (e-27.west) edge[in=20, out=160, looseness=7] (e-27.east);
    \path[pointer-right] (e-27.east) edge[in=-160, out=-20, looseness=7] (e-27.west);
    \path[pointer-parent] (e-27.north) edge[bend right] (e-14.south);
    \path[pointer-left] (e-8.west) edge[in=20, out=160, looseness=1] (e-13.east);
    \path[pointer-right] (e-8.east) edge[bend right] (e-7.west);
    \path[pointer-child] (e-8.south west) edge[bend right] (e-18.north);
    \path[pointer-left] (e-18.west) edge[in=20, out=160, looseness=2.5] (e-14.east);
    \path[pointer-right] (e-18.east) edge[bend right] (e-14.west);
    \path[pointer-parent] (e-18.north) edge[bend right] (e-8.south west);
    \path[pointer-parent] (e-15.north) edge[bend right] (e-7.south);
    \path[pointer-left] (e-15.west) edge[in=20, out=160, looseness=7] (e-15.east);
    \path[pointer-right] (e-15.east) edge[in=-160, out=-20, looseness=7] (e-15.west);
    \path[pointer-left] (e-7.west) edge[bend right] (e-8.east);
    \path[pointer-right] (e-7.east) edge[bend right] (e-13.west);
    \path[pointer-child] (e-7.south) edge[bend right] (e-15.north);
    \path[pointer-left] (e-13.west) edge[bend right] (e-7.east);
    \path[pointer-right] (e-13.east) edge[in=-160, out=-20, looseness=1] (e-8.west);
    \path[pointer-child] (e-13.south) edge[bend right] (e-21.north);
    \path[pointer-left] (e-21.west) edge[in=20, out=160, looseness=7] (e-21.east);
    \path[pointer-right] (e-21.east) edge[in=-160, out=-20, looseness=7] (e-21.west);
    \path[pointer-parent] (e-21.north) edge[bend right] (e-13.south);

    \node[elem-array, above = 2 of e-8] (array-0) {};
    \node[above = 0 of array-0] (array-0-t) {\footnotesize$0$};
    \foreach\i [count = \xi from 0] in {1, ..., 5}{
      \node[elem-array, right = -0.015 of array-\xi] (array-\i) {};
      \node[above = 0 of array-\i] (array-\i-t) {\footnotesize$\i$};
    }

    \draw[pointer] (array-2.center) -- (e-8.north);
    \draw[pointer] (array-1.center) -- (e-7.north);
    \draw[pointer-conflict] (array-1.center) -- (e-13.north west);
  \end{tikzpicture}
  \caption{Nach Entfernen von $3$ (Cont'd [2]).}
\end{figure}

\clearpage

\begin{figure}[H]
  \centering
  \begin{tikzpicture}[trim left = -2.5cm]
    \node[elem] (e-8) {$8$};
    \node[elem, below left = 1.5 and 0.5 of e-8] (e-18) {$18$};
    \node[elem, below right = 1.5 and 0.5 of e-8] (e-14) {$14$};
    \node[elem, below = 1 of e-14] (e-27) {$27$};
    \node[elem, right = 5 of e-8] (e-7) {$7$};
    \node[elem, below left = 1.5 and 0.5 of e-7] (e-15) {$15$};
    \node[elem, below right = 1.5 and 0.5 of e-7] (e-13) {$13$};
    \node[elem, below = 1 of e-13] (e-21) {$21$};
    %\draw[pointer-min] (e-3.north) -- ++(0, 0.5);
    \path[pointer-left] (e-14.west) edge[bend right] (e-18.east);
    \path[pointer-right] (e-14.east) edge[in=-160, out=-20, looseness=2.5] (e-18.west);
    \path[pointer-child] (e-14.south) edge[bend right] (e-27.north);
    \path[pointer-parent] (e-14.north) edge[bend right] (e-8.south east);
    \path[pointer-left] (e-27.west) edge[in=20, out=160, looseness=7] (e-27.east);
    \path[pointer-right] (e-27.east) edge[in=-160, out=-20, looseness=7] (e-27.west);
    \path[pointer-parent] (e-27.north) edge[bend right] (e-14.south);
    \path[pointer-left] (e-8.west) edge[in=20, out=160, looseness=1.25] (e-7.east);
    \path[pointer-right] (e-8.east) edge[bend right] (e-7.west);
    \path[pointer-child] (e-8.south west) edge[bend right] (e-18.north);
    \path[pointer-left] (e-18.west) edge[in=20, out=160, looseness=2.5] (e-14.east);
    \path[pointer-right] (e-18.east) edge[bend right] (e-14.west);
    \path[pointer-parent] (e-18.north) edge[bend right] (e-8.south west);
    \path[pointer-left] (e-13.west) edge[bend right] (e-15.east);
    \path[pointer-right] (e-13.east) edge[in=-160, out=-20, looseness=2.5] (e-15.west);
    \path[pointer-child] (e-13.south) edge[bend right] (e-21.north);
    \path[pointer-parent] (e-13.north) edge[bend right] (e-7.south east);
    \path[pointer-left] (e-21.west) edge[in=20, out=160, looseness=7] (e-21.east);
    \path[pointer-right] (e-21.east) edge[in=-160, out=-20, looseness=7] (e-21.west);
    \path[pointer-parent] (e-21.north) edge[bend right] (e-13.south);
    \path[pointer-left] (e-7.west) edge[bend right] (e-8.east);
    \path[pointer-right] (e-7.east) edge[in=-160, out=-20, looseness=1.25] (e-8.west);
    \path[pointer-child] (e-7.south west) edge[bend right] (e-15.north);
    \path[pointer-left] (e-15.west) edge[in=20, out=160, looseness=2.5] (e-13.east);
    \path[pointer-right] (e-15.east) edge[bend right] (e-13.west);
    \path[pointer-parent] (e-15.north) edge[bend right] (e-7.south west);

    \node[elem-array, above = 2 of e-8] (array-0) {};
    \node[above = 0 of array-0] (array-0-t) {\footnotesize$0$};
    \foreach\i [count = \xi from 0] in {1, ..., 5}{
      \node[elem-array, right = -0.015 of array-\xi] (array-\i) {};
      \node[above = 0 of array-\i] (array-\i-t) {\footnotesize$\i$};
    }

    \draw[pointer] (array-2.center) -- (e-8.north);
    \draw[pointer-conflict] (array-2.center) -- (e-7.north west);

    \node[below = 2.5 of e-15] {$\Downarrow$};

    \node[elem, below right = 4 and 2 of e-27] (e-7) {$7$};
    \node[elem, below left = 1.5 and 3 of e-7] (e-15) {$15$};
    \node[elem, below = 1.225 of e-7] (e-13) {$13$};
    \node[elem, below right = 1.5 and 3 of e-7] (e-8) {$8$};
    \node[elem, below = 1.225 of e-13] (e-21) {$21$};
    \node[elem, below left = 1.5 and 0.5 of e-8] (e-18) {$18$};
    \node[elem, below right = 1.5 and 0.5 of e-8] (e-14) {$14$};
    \node[elem, below = 1.225 of e-14] (e-27) {$27$};
    %\draw[pointer-min] (e-3.north) -- ++(0, 0.5);
    \path[pointer-left] (e-7.west) edge[in=20, out=160, looseness=7] (e-7.east);
    \path[pointer-right] (e-7.east) edge[in=-160, out=-20, looseness=7] (e-7.west);
    \path[pointer-child] (e-7.south west) edge[bend right] (e-15.north east);
    \path[pointer-parent] (e-15.north east) edge[bend right] (e-7.south west);
    \path[pointer-child] (e-8.south west) edge[bend right] (e-18.north);
    \path[pointer-parent] (e-8.north west) edge[bend left] (e-7.south east);
    \path[pointer-parent] (e-18.north) edge[bend right] (e-8.south west);
    \path[pointer-parent] (e-13.north) edge[bend right] (e-7.south);
    \path[pointer-child] (e-13.south) edge[bend right] (e-21.north);
    \path[pointer-parent] (e-21.north) edge[bend right] (e-13.south);
    \path[pointer-parent] (e-14.north) edge[bend right] (e-8.south east);
    \path[pointer-child] (e-14.south) edge[bend right] (e-27.north);
    \path[pointer-parent] (e-27.north) edge[bend right] (e-14.south);
    \path[pointer-left] (e-15.west) edge[in=20, out=160, looseness=1] (e-8.east);
    \path[pointer-right] (e-15.east) edge[bend right] (e-13.west);
    \path[pointer-left] (e-8.west) edge[bend right] (e-13.east);
    \path[pointer-right] (e-13.east) edge[bend right] (e-8.west);
    \path[pointer-left] (e-18.west) edge[in=20, out=160, looseness=2.5] (e-14.east);
    \path[pointer-right] (e-18.east) edge[bend right] (e-14.west);
    \path[pointer-left] (e-13.west) edge[bend right] (e-15.east);
    \path[pointer-right] (e-8.east) edge[in=-160, out=-20, looseness=1] (e-15.west);
    \path[pointer-left] (e-14.west) edge[bend right] (e-18.east);
    \path[pointer-right] (e-14.east) edge[in=-160, out=-20, looseness=2.5] (e-18.west);
    \path[pointer-left] (e-21.west) edge[in=20, out=160, looseness=7] (e-21.east);
    \path[pointer-right] (e-21.east) edge[in=-160, out=-20, looseness=7] (e-21.west);
    \path[pointer-left] (e-27.west) edge[in=20, out=160, looseness=7] (e-27.east);
    \path[pointer-right] (e-27.east) edge[in=-160, out=-20, looseness=7] (e-27.west);

    \node[elem-array, above = 1 of e-7] (array-0) {};
    \node[above = 0 of array-0] (array-0-t) {\footnotesize$0$};
    \foreach\i [count = \xi from 0] in {1, ..., 5}{
      \node[elem-array, right = -0.015 of array-\xi] (array-\i) {};
      \node[above = 0 of array-\i] (array-\i-t) {\footnotesize$\i$};
    }

    \draw[pointer] (array-3.center) -- (e-7.north);   
  \end{tikzpicture}
  \caption{Nach Entfernen von $3$ (Cont'd [3]).}
\end{figure}

\clearpage

\begin{figure}[H]
  \centering
  \begin{tikzpicture}[trim left=4cm, scale=0.9, every node/.style={scale=0.9}]
    \node[elem, below right = 4 and 2 of e-27] (e-7) {$7$};
    \node[elem, below left = 1.5 and 3 of e-7] (e-15) {$15$};
    \node[elem, below = 1.225 of e-7] (e-13) {$13$};
    \node[elem, below right = 1.5 and 3 of e-7] (e-8) {$8$};
    \node[elem, below = 1.225 of e-13] (e-21) {$21$};
    \node[elem, below left = 1.5 and 0.5 of e-8] (e-18) {$18$};
    \node[elem, below right = 1.5 and 0.5 of e-8] (e-14) {$14$};
    \node[elem, below = 1.225 of e-14] (e-27) {$27$};

    \path[pointer-left] (e-7.west) edge[in=20, out=160, looseness=7] (e-7.east);
    \path[pointer-right] (e-7.east) edge[in=-160, out=-20, looseness=7] (e-7.west);
    \path[pointer-child] (e-7.south west) edge[bend right] (e-15.north east);
    \path[pointer-parent] (e-15.north east) edge[bend right] (e-7.south west);
    \path[pointer-child] (e-8.south west) edge[bend right] (e-18.north);
    \path[pointer-parent] (e-8.north west) edge[bend left] (e-7.south east);
    \path[pointer-parent] (e-18.north) edge[bend right] (e-8.south west);
    \path[pointer-parent] (e-13.north) edge[bend right] (e-7.south);
    \path[pointer-child] (e-13.south) edge[bend right] (e-21.north);
    \path[pointer-parent] (e-21.north) edge[bend right] (e-13.south);
    \path[pointer-parent] (e-14.north) edge[bend right] (e-8.south east);
    \path[pointer-child] (e-14.south) edge[bend right] (e-27.north);
    \path[pointer-parent] (e-27.north) edge[bend right] (e-14.south);
    \path[pointer-left] (e-15.west) edge[in=20, out=160, looseness=1] (e-8.east);
    \path[pointer-right] (e-15.east) edge[bend right] (e-13.west);
    \path[pointer-left] (e-8.west) edge[bend right] (e-13.east);
    \path[pointer-right] (e-13.east) edge[bend right] (e-8.west);
    \path[pointer-left] (e-18.west) edge[in=20, out=160, looseness=2.5] (e-14.east);
    \path[pointer-right] (e-18.east) edge[bend right] (e-14.west);
    \path[pointer-left] (e-13.west) edge[bend right] (e-15.east);
    \path[pointer-right] (e-8.east) edge[in=-160, out=-20, looseness=1] (e-15.west);
    \path[pointer-left] (e-14.west) edge[bend right] (e-18.east);
    \path[pointer-right] (e-14.east) edge[in=-160, out=-20, looseness=2.5] (e-18.west);
    \path[pointer-left] (e-21.west) edge[in=20, out=160, looseness=7] (e-21.east);
    \path[pointer-right] (e-21.east) edge[in=-160, out=-20, looseness=7] (e-21.west);
    \path[pointer-left] (e-27.west) edge[in=20, out=160, looseness=7] (e-27.east);
    \path[pointer-right] (e-27.east) edge[in=-160, out=-20, looseness=7] (e-27.west);

    \draw[pointer-min] (e-7.north) -- ++(0, 0.5);
  \end{tikzpicture}
  \caption{Finaler Fibonacci-Heap nach Entfernen von $3$.}
\end{figure}

Nun wird jeder Schl{\"u}ssel um 7 dekrementiert (in der Reihenfolge, in der sie
eingef{\"u}gt wurden, d.h. $13$, $21$, $7$, $15$, $18$, $8$, $14$ und $27$).

\begin{figure}[H]
  \centering
  \begin{tikzpicture}[trim left=-4cm, scale=0.9, every node/.style={scale=0.9}]
    \node[elem] (e-6) {$6$};
    \node[elem, below = 1.225 of e-6] (e-21) {$21$};
    \node[elem, right = 3 of e-6] (e-7) {$7$};
    \node[elem, below left = 1.5 and 0.5 of e-7] (e-15) {$15$};
    \node[elem, below right = 1.5 and 0.5 of e-7] (e-8) {$8$};
    \node[elem, below left = 1.5 and 0.5 of e-8] (e-18) {$18$};
    \node[elem, below right = 1.5 and 0.5 of e-8] (e-14) {$14$};
    \node[elem, below = 1.225 of e-14] (e-27) {$27$};
    
    \path[pointer-left] (e-6.west) edge[in=20, out=160, looseness=2] (e-7.east);
    \path[pointer-right] (e-6.east) edge[bend right] (e-7.west);
    \path[pointer-child] (e-6.south) edge[bend right] (e-21.north);
    \path[pointer-left] (e-7.west) edge[bend right] (e-6.east);
    \path[pointer-right] (e-7.east) edge[in=-160, out=-20, looseness=2] (e-6.west);
    \path[pointer-child] (e-7.south west) edge[bend right] (e-15.north);
    \path[pointer-left] (e-21.west) edge[in=20, out=160, looseness=7] (e-21.east);
    \path[pointer-right] (e-21.east) edge[in=-160, out=-20, looseness=7] (e-21.west);
    \path[pointer-parent] (e-21.north) edge[bend right] (e-6.south);
    \path[pointer-left] (e-15.west) edge[in=20, out=160, looseness=2.5] (e-8.east);
    \path[pointer-right] (e-15.east) edge[bend right] (e-8.west);
    \path[pointer-parent] (e-15.north) edge[bend right] (e-7.south west);
    \path[pointer-left] (e-8.west) edge[bend right] (e-15.east);
    \path[pointer-right] (e-8.east) edge[in=-160, out=-20, looseness=2.5] (e-15.west);
    \path[pointer-child] (e-8.south west) edge[bend right] (e-18.north);
    \path[pointer-parent] (e-8.north) edge[bend right] (e-7.south east);
    \path[pointer-left] (e-18.west) edge[in=20, out=160, looseness=2.5] (e-14.east);
    \path[pointer-right] (e-18.east) edge[bend right] (e-14.west);
    \path[pointer-parent] (e-18.north) edge[bend right] (e-8.south west);
    \path[pointer-right] (e-14.east) edge[in=-160, out=-20, looseness=2.5] (e-18.west);
    \path[pointer-left] (e-14.west) edge[bend right] (e-18.east);
    \path[pointer-child] (e-14.south) edge[bend right] (e-27.north);
    \path[pointer-parent] (e-14.north) edge[bend right] (e-8.south east);
    \path[pointer-left] (e-27.west) edge[in=20, out=160, looseness=7] (e-27.east);
    \path[pointer-right] (e-27.east) edge[in=-160, out=-20, looseness=7] (e-27.west);
    \path[pointer-parent] (e-27.north) edge[bend right] (e-14.south);

    \draw[pointer-min] (e-6.north) -- ++(0, 0.5);
  \end{tikzpicture}
  \caption{Nach \texttt{decreasekey(13, 6)} ($Q.min = 6$; Subtree mit Wurzel $6$
    wird abgetrennt; Vater von $6$ wird nicht gef{\"a}rbt, da $7$ ein Root-Knoten
    ist).
  }
\end{figure}

\begin{figure}[H]
  \centering
  \begin{tikzpicture}[trim left=-4cm, scale=0.9, every node/.style={scale=0.9}]
    \node[elem] (e-6) {$6$};
    \node[elem, below = 1.225 of e-6] (e-21) {$14$};
    \node[elem, right = 3 of e-6] (e-7) {$7$};
    \node[elem, below left = 1.5 and 0.5 of e-7] (e-15) {$15$};
    \node[elem, below right = 1.5 and 0.5 of e-7] (e-8) {$8$};
    \node[elem, below left = 1.5 and 0.5 of e-8] (e-18) {$18$};
    \node[elem, below right = 1.5 and 0.5 of e-8] (e-14) {$14$};
    \node[elem, below = 1.225 of e-14] (e-27) {$27$};
    
    \path[pointer-left] (e-6.west) edge[in=20, out=160, looseness=2] (e-7.east);
    \path[pointer-right] (e-6.east) edge[bend right] (e-7.west);
    \path[pointer-child] (e-6.south) edge[bend right] (e-21.north);
    \path[pointer-left] (e-7.west) edge[bend right] (e-6.east);
    \path[pointer-right] (e-7.east) edge[in=-160, out=-20, looseness=2] (e-6.west);
    \path[pointer-child] (e-7.south west) edge[bend right] (e-15.north);
    \path[pointer-left] (e-21.west) edge[in=20, out=160, looseness=7] (e-21.east);
    \path[pointer-right] (e-21.east) edge[in=-160, out=-20, looseness=7] (e-21.west);
    \path[pointer-parent] (e-21.north) edge[bend right] (e-6.south);
    \path[pointer-left] (e-15.west) edge[in=20, out=160, looseness=2.5] (e-8.east);
    \path[pointer-right] (e-15.east) edge[bend right] (e-8.west);
    \path[pointer-parent] (e-15.north) edge[bend right] (e-7.south west);
    \path[pointer-left] (e-8.west) edge[bend right] (e-15.east);
    \path[pointer-right] (e-8.east) edge[in=-160, out=-20, looseness=2.5] (e-15.west);
    \path[pointer-child] (e-8.south west) edge[bend right] (e-18.north);
    \path[pointer-parent] (e-8.north) edge[bend right] (e-7.south east);
    \path[pointer-left] (e-18.west) edge[in=20, out=160, looseness=2.5] (e-14.east);
    \path[pointer-right] (e-18.east) edge[bend right] (e-14.west);
    \path[pointer-parent] (e-18.north) edge[bend right] (e-8.south west);
    \path[pointer-right] (e-14.east) edge[in=-160, out=-20, looseness=2.5] (e-18.west);
    \path[pointer-left] (e-14.west) edge[bend right] (e-18.east);
    \path[pointer-child] (e-14.south) edge[bend right] (e-27.north);
    \path[pointer-parent] (e-14.north) edge[bend right] (e-8.south east);
    \path[pointer-left] (e-27.west) edge[in=20, out=160, looseness=7] (e-27.east);
    \path[pointer-right] (e-27.east) edge[in=-160, out=-20, looseness=7] (e-27.west);
    \path[pointer-parent] (e-27.north) edge[bend right] (e-14.south);

    \draw[pointer-min] (e-6.north) -- ++(0, 0.5);
  \end{tikzpicture}
  \caption{Nach \texttt{decreasekey(21, 14)} (Heap-Bedingung wird nicht verletzt,
    daher kann der Key einfach aktualisiert werden).
  }
\end{figure}

\begin{figure}[H]
  \centering
  \begin{tikzpicture}[trim left=-4cm, scale=0.9, every node/.style={scale=0.9}]
    \node[elem] (e-6) {$6$};
    \node[elem, below = 1.225 of e-6] (e-21) {$14$};
    \node[elem, right = 3 of e-6] (e-7) {$0$};
    \node[elem, below left = 1.5 and 0.5 of e-7] (e-15) {$15$};
    \node[elem, below right = 1.5 and 0.5 of e-7] (e-8) {$8$};
    \node[elem, below left = 1.5 and 0.5 of e-8] (e-18) {$18$};
    \node[elem, below right = 1.5 and 0.5 of e-8] (e-14) {$14$};
    \node[elem, below = 1.225 of e-14] (e-27) {$27$};
    
    \path[pointer-left] (e-6.west) edge[in=20, out=160, looseness=2] (e-7.east);
    \path[pointer-right] (e-6.east) edge[bend right] (e-7.west);
    \path[pointer-child] (e-6.south) edge[bend right] (e-21.north);
    \path[pointer-left] (e-7.west) edge[bend right] (e-6.east);
    \path[pointer-right] (e-7.east) edge[in=-160, out=-20, looseness=2] (e-6.west);
    \path[pointer-child] (e-7.south west) edge[bend right] (e-15.north);
    \path[pointer-left] (e-21.west) edge[in=20, out=160, looseness=7] (e-21.east);
    \path[pointer-right] (e-21.east) edge[in=-160, out=-20, looseness=7] (e-21.west);
    \path[pointer-parent] (e-21.north) edge[bend right] (e-6.south);
    \path[pointer-left] (e-15.west) edge[in=20, out=160, looseness=2.5] (e-8.east);
    \path[pointer-right] (e-15.east) edge[bend right] (e-8.west);
    \path[pointer-parent] (e-15.north) edge[bend right] (e-7.south west);
    \path[pointer-left] (e-8.west) edge[bend right] (e-15.east);
    \path[pointer-right] (e-8.east) edge[in=-160, out=-20, looseness=2.5] (e-15.west);
    \path[pointer-child] (e-8.south west) edge[bend right] (e-18.north);
    \path[pointer-parent] (e-8.north) edge[bend right] (e-7.south east);
    \path[pointer-left] (e-18.west) edge[in=20, out=160, looseness=2.5] (e-14.east);
    \path[pointer-right] (e-18.east) edge[bend right] (e-14.west);
    \path[pointer-parent] (e-18.north) edge[bend right] (e-8.south west);
    \path[pointer-right] (e-14.east) edge[in=-160, out=-20, looseness=2.5] (e-18.west);
    \path[pointer-left] (e-14.west) edge[bend right] (e-18.east);
    \path[pointer-child] (e-14.south) edge[bend right] (e-27.north);
    \path[pointer-parent] (e-14.north) edge[bend right] (e-8.south east);
    \path[pointer-left] (e-27.west) edge[in=20, out=160, looseness=7] (e-27.east);
    \path[pointer-right] (e-27.east) edge[in=-160, out=-20, looseness=7] (e-27.west);
    \path[pointer-parent] (e-27.north) edge[bend right] (e-14.south);

    \draw[pointer-min] (e-7.north) -- ++(0, 0.5);
  \end{tikzpicture}
  \caption{Nach \texttt{decreasekey(7, 0)} (Heap-Bedingung wird nicht verletzt,
    daher kann der Key einfach aktualisiert werden; $Q.min$ wird $= 0$ gesetzt).
  }
\end{figure}

\begin{figure}[H]
  \centering
  \begin{tikzpicture}[trim left=-4cm, scale=0.9, every node/.style={scale=0.9}]
    \node[elem] (e-6) {$6$};
    \node[elem, below = 1.225 of e-6] (e-21) {$14$};
    \node[elem, right = 3 of e-6] (e-7) {$0$};
    \node[elem, below left = 1.5 and 0.5 of e-7] (e-15) {$8$};
    \node[elem, below right = 1.5 and 0.5 of e-7] (e-8) {$8$};
    \node[elem, below left = 1.5 and 0.5 of e-8] (e-18) {$18$};
    \node[elem, below right = 1.5 and 0.5 of e-8] (e-14) {$14$};
    \node[elem, below = 1.225 of e-14] (e-27) {$27$};
    
    \path[pointer-left] (e-6.west) edge[in=20, out=160, looseness=2] (e-7.east);
    \path[pointer-right] (e-6.east) edge[bend right] (e-7.west);
    \path[pointer-child] (e-6.south) edge[bend right] (e-21.north);
    \path[pointer-left] (e-7.west) edge[bend right] (e-6.east);
    \path[pointer-right] (e-7.east) edge[in=-160, out=-20, looseness=2] (e-6.west);
    \path[pointer-child] (e-7.south west) edge[bend right] (e-15.north);
    \path[pointer-left] (e-21.west) edge[in=20, out=160, looseness=7] (e-21.east);
    \path[pointer-right] (e-21.east) edge[in=-160, out=-20, looseness=7] (e-21.west);
    \path[pointer-parent] (e-21.north) edge[bend right] (e-6.south);
    \path[pointer-left] (e-15.west) edge[in=20, out=160, looseness=2.5] (e-8.east);
    \path[pointer-right] (e-15.east) edge[bend right] (e-8.west);
    \path[pointer-parent] (e-15.north) edge[bend right] (e-7.south west);
    \path[pointer-left] (e-8.west) edge[bend right] (e-15.east);
    \path[pointer-right] (e-8.east) edge[in=-160, out=-20, looseness=2.5] (e-15.west);
    \path[pointer-child] (e-8.south west) edge[bend right] (e-18.north);
    \path[pointer-parent] (e-8.north) edge[bend right] (e-7.south east);
    \path[pointer-left] (e-18.west) edge[in=20, out=160, looseness=2.5] (e-14.east);
    \path[pointer-right] (e-18.east) edge[bend right] (e-14.west);
    \path[pointer-parent] (e-18.north) edge[bend right] (e-8.south west);
    \path[pointer-right] (e-14.east) edge[in=-160, out=-20, looseness=2.5] (e-18.west);
    \path[pointer-left] (e-14.west) edge[bend right] (e-18.east);
    \path[pointer-child] (e-14.south) edge[bend right] (e-27.north);
    \path[pointer-parent] (e-14.north) edge[bend right] (e-8.south east);
    \path[pointer-left] (e-27.west) edge[in=20, out=160, looseness=7] (e-27.east);
    \path[pointer-right] (e-27.east) edge[in=-160, out=-20, looseness=7] (e-27.west);
    \path[pointer-parent] (e-27.north) edge[bend right] (e-14.south);

    \draw[pointer-min] (e-7.north) -- ++(0, 0.5);
  \end{tikzpicture}
  \caption{Nach \texttt{decreasekey(15, 8)} (Heap-Bedingung wird nicht verletzt,
    daher kann der Key einfach aktualisiert werden).
  }
\end{figure}

\begin{figure}[H]
  \centering
  \begin{tikzpicture}[trim left=-4cm, scale=0.9, every node/.style={scale=0.9}]
    \node[elem] (e-6) {$6$};
    \node[elem, below = 1.225 of e-6] (e-21) {$14$};
    \node[elem, right = 3 of e-6] (e-7) {$0$};
    \node[elem, below left = 1.5 and 0.5 of e-7] (e-15) {$8$};
    \node[elem, below right = 1.5 and 0.5 of e-7] (e-8) {$8$};
    \node[elem, below left = 1.5 and 0.5 of e-8] (e-18) {$11$};
    \node[elem, below right = 1.5 and 0.5 of e-8] (e-14) {$14$};
    \node[elem, below = 1.225 of e-14] (e-27) {$27$};
    
    \path[pointer-left] (e-6.west) edge[in=20, out=160, looseness=2] (e-7.east);
    \path[pointer-right] (e-6.east) edge[bend right] (e-7.west);
    \path[pointer-child] (e-6.south) edge[bend right] (e-21.north);
    \path[pointer-left] (e-7.west) edge[bend right] (e-6.east);
    \path[pointer-right] (e-7.east) edge[in=-160, out=-20, looseness=2] (e-6.west);
    \path[pointer-child] (e-7.south west) edge[bend right] (e-15.north);
    \path[pointer-left] (e-21.west) edge[in=20, out=160, looseness=7] (e-21.east);
    \path[pointer-right] (e-21.east) edge[in=-160, out=-20, looseness=7] (e-21.west);
    \path[pointer-parent] (e-21.north) edge[bend right] (e-6.south);
    \path[pointer-left] (e-15.west) edge[in=20, out=160, looseness=2.5] (e-8.east);
    \path[pointer-right] (e-15.east) edge[bend right] (e-8.west);
    \path[pointer-parent] (e-15.north) edge[bend right] (e-7.south west);
    \path[pointer-left] (e-8.west) edge[bend right] (e-15.east);
    \path[pointer-right] (e-8.east) edge[in=-160, out=-20, looseness=2.5] (e-15.west);
    \path[pointer-child] (e-8.south west) edge[bend right] (e-18.north);
    \path[pointer-parent] (e-8.north) edge[bend right] (e-7.south east);
    \path[pointer-left] (e-18.west) edge[in=20, out=160, looseness=2.5] (e-14.east);
    \path[pointer-right] (e-18.east) edge[bend right] (e-14.west);
    \path[pointer-parent] (e-18.north) edge[bend right] (e-8.south west);
    \path[pointer-right] (e-14.east) edge[in=-160, out=-20, looseness=2.5] (e-18.west);
    \path[pointer-left] (e-14.west) edge[bend right] (e-18.east);
    \path[pointer-child] (e-14.south) edge[bend right] (e-27.north);
    \path[pointer-parent] (e-14.north) edge[bend right] (e-8.south east);
    \path[pointer-left] (e-27.west) edge[in=20, out=160, looseness=7] (e-27.east);
    \path[pointer-right] (e-27.east) edge[in=-160, out=-20, looseness=7] (e-27.west);
    \path[pointer-parent] (e-27.north) edge[bend right] (e-14.south);

    \draw[pointer-min] (e-7.north) -- ++(0, 0.5);
  \end{tikzpicture}
  \caption{Nach \texttt{decreasekey(18, 11)} (Heap-Bedingung wird nicht verletzt,
    daher kann der Key einfach aktualisiert werden).
  }
\end{figure}

\begin{figure}[H]
  \centering
  \begin{tikzpicture}[trim left=-4cm, scale=0.9, every node/.style={scale=0.9}]
    \node[elem] (e-6) {$6$};
    \node[elem, below = 1.225 of e-6] (e-21) {$14$};
    \node[elem, right = 3 of e-6] (e-7) {$0$};
    \node[elem, below left = 1.5 and 0.5 of e-7] (e-15) {$8$};
    \node[elem, below right = 1.5 and 0.5 of e-7] (e-8) {$1$};
    \node[elem, below left = 1.5 and 0.5 of e-8] (e-18) {$11$};
    \node[elem, below right = 1.5 and 0.5 of e-8] (e-14) {$14$};
    \node[elem, below = 1.225 of e-14] (e-27) {$27$};
    
    \path[pointer-left] (e-6.west) edge[in=20, out=160, looseness=2] (e-7.east);
    \path[pointer-right] (e-6.east) edge[bend right] (e-7.west);
    \path[pointer-child] (e-6.south) edge[bend right] (e-21.north);
    \path[pointer-left] (e-7.west) edge[bend right] (e-6.east);
    \path[pointer-right] (e-7.east) edge[in=-160, out=-20, looseness=2] (e-6.west);
    \path[pointer-child] (e-7.south west) edge[bend right] (e-15.north);
    \path[pointer-left] (e-21.west) edge[in=20, out=160, looseness=7] (e-21.east);
    \path[pointer-right] (e-21.east) edge[in=-160, out=-20, looseness=7] (e-21.west);
    \path[pointer-parent] (e-21.north) edge[bend right] (e-6.south);
    \path[pointer-left] (e-15.west) edge[in=20, out=160, looseness=2.5] (e-8.east);
    \path[pointer-right] (e-15.east) edge[bend right] (e-8.west);
    \path[pointer-parent] (e-15.north) edge[bend right] (e-7.south west);
    \path[pointer-left] (e-8.west) edge[bend right] (e-15.east);
    \path[pointer-right] (e-8.east) edge[in=-160, out=-20, looseness=2.5] (e-15.west);
    \path[pointer-child] (e-8.south west) edge[bend right] (e-18.north);
    \path[pointer-parent] (e-8.north) edge[bend right] (e-7.south east);
    \path[pointer-left] (e-18.west) edge[in=20, out=160, looseness=2.5] (e-14.east);
    \path[pointer-right] (e-18.east) edge[bend right] (e-14.west);
    \path[pointer-parent] (e-18.north) edge[bend right] (e-8.south west);
    \path[pointer-right] (e-14.east) edge[in=-160, out=-20, looseness=2.5] (e-18.west);
    \path[pointer-left] (e-14.west) edge[bend right] (e-18.east);
    \path[pointer-child] (e-14.south) edge[bend right] (e-27.north);
    \path[pointer-parent] (e-14.north) edge[bend right] (e-8.south east);
    \path[pointer-left] (e-27.west) edge[in=20, out=160, looseness=7] (e-27.east);
    \path[pointer-right] (e-27.east) edge[in=-160, out=-20, looseness=7] (e-27.west);
    \path[pointer-parent] (e-27.north) edge[bend right] (e-14.south);

    \draw[pointer-min] (e-7.north) -- ++(0, 0.5);
  \end{tikzpicture}
  \caption{Nach \texttt{decreasekey(8, 1)} (Heap-Bedingung wird nicht verletzt,
    daher kann der Key einfach aktualisiert werden).
  }
\end{figure}

\begin{figure}[H]
  \centering
  \begin{tikzpicture}[trim left=-4cm, scale=0.9, every node/.style={scale=0.9}]
    \node[elem] (e-6) {$6$};
    \node[elem, below = 1.225 of e-6] (e-21) {$14$};
    \node[elem, right = 3 of e-6] (e-7) {$0$};
    \node[elem, below left = 1.5 and 0.5 of e-7] (e-15) {$8$};
    \node[elem, below right = 1.5 and 0.5 of e-7] (e-8) {$1$};
    \node[elem, below left = 1.5 and 0.5 of e-8] (e-18) {$11$};
    \node[elem, below right = 1.5 and 0.5 of e-8] (e-14) {$7$};
    \node[elem, below = 1.225 of e-14] (e-27) {$27$};
    
    \path[pointer-left] (e-6.west) edge[in=20, out=160, looseness=2] (e-7.east);
    \path[pointer-right] (e-6.east) edge[bend right] (e-7.west);
    \path[pointer-child] (e-6.south) edge[bend right] (e-21.north);
    \path[pointer-left] (e-7.west) edge[bend right] (e-6.east);
    \path[pointer-right] (e-7.east) edge[in=-160, out=-20, looseness=2] (e-6.west);
    \path[pointer-child] (e-7.south west) edge[bend right] (e-15.north);
    \path[pointer-left] (e-21.west) edge[in=20, out=160, looseness=7] (e-21.east);
    \path[pointer-right] (e-21.east) edge[in=-160, out=-20, looseness=7] (e-21.west);
    \path[pointer-parent] (e-21.north) edge[bend right] (e-6.south);
    \path[pointer-left] (e-15.west) edge[in=20, out=160, looseness=2.5] (e-8.east);
    \path[pointer-right] (e-15.east) edge[bend right] (e-8.west);
    \path[pointer-parent] (e-15.north) edge[bend right] (e-7.south west);
    \path[pointer-left] (e-8.west) edge[bend right] (e-15.east);
    \path[pointer-right] (e-8.east) edge[in=-160, out=-20, looseness=2.5] (e-15.west);
    \path[pointer-child] (e-8.south west) edge[bend right] (e-18.north);
    \path[pointer-parent] (e-8.north) edge[bend right] (e-7.south east);
    \path[pointer-left] (e-18.west) edge[in=20, out=160, looseness=2.5] (e-14.east);
    \path[pointer-right] (e-18.east) edge[bend right] (e-14.west);
    \path[pointer-parent] (e-18.north) edge[bend right] (e-8.south west);
    \path[pointer-right] (e-14.east) edge[in=-160, out=-20, looseness=2.5] (e-18.west);
    \path[pointer-left] (e-14.west) edge[bend right] (e-18.east);
    \path[pointer-child] (e-14.south) edge[bend right] (e-27.north);
    \path[pointer-parent] (e-14.north) edge[bend right] (e-8.south east);
    \path[pointer-left] (e-27.west) edge[in=20, out=160, looseness=7] (e-27.east);
    \path[pointer-right] (e-27.east) edge[in=-160, out=-20, looseness=7] (e-27.west);
    \path[pointer-parent] (e-27.north) edge[bend right] (e-14.south);

    \draw[pointer-min] (e-7.north) -- ++(0, 0.5);
  \end{tikzpicture}
  \caption{Nach \texttt{decreasekey(14, 7)} (Heap-Bedingung wird nicht verletzt,
    daher kann der Key einfach aktualisiert werden).
  }
\end{figure}

\begin{figure}[H]
  \centering
  \begin{tikzpicture}[trim left=-4cm, scale=0.9, every node/.style={scale=0.9}]
    \node[elem] (e-6) {$6$};
    \node[elem, below = 1.225 of e-6] (e-21) {$14$};
    \node[elem, right = 3 of e-6] (e-7) {$0$};
    \node[elem, below left = 1.5 and 0.5 of e-7] (e-15) {$8$};
    \node[elem, below right = 1.5 and 0.5 of e-7] (e-8) {$1$};
    \node[elem, below left = 1.5 and 0.5 of e-8] (e-18) {$11$};
    \node[elem, below right = 1.5 and 0.5 of e-8] (e-14) {$7$};
    \node[elem, below = 1.225 of e-14] (e-27) {$20$};
    
    \path[pointer-left] (e-6.west) edge[in=20, out=160, looseness=2] (e-7.east);
    \path[pointer-right] (e-6.east) edge[bend right] (e-7.west);
    \path[pointer-child] (e-6.south) edge[bend right] (e-21.north);
    \path[pointer-left] (e-7.west) edge[bend right] (e-6.east);
    \path[pointer-right] (e-7.east) edge[in=-160, out=-20, looseness=2] (e-6.west);
    \path[pointer-child] (e-7.south west) edge[bend right] (e-15.north);
    \path[pointer-left] (e-21.west) edge[in=20, out=160, looseness=7] (e-21.east);
    \path[pointer-right] (e-21.east) edge[in=-160, out=-20, looseness=7] (e-21.west);
    \path[pointer-parent] (e-21.north) edge[bend right] (e-6.south);
    \path[pointer-left] (e-15.west) edge[in=20, out=160, looseness=2.5] (e-8.east);
    \path[pointer-right] (e-15.east) edge[bend right] (e-8.west);
    \path[pointer-parent] (e-15.north) edge[bend right] (e-7.south west);
    \path[pointer-left] (e-8.west) edge[bend right] (e-15.east);
    \path[pointer-right] (e-8.east) edge[in=-160, out=-20, looseness=2.5] (e-15.west);
    \path[pointer-child] (e-8.south west) edge[bend right] (e-18.north);
    \path[pointer-parent] (e-8.north) edge[bend right] (e-7.south east);
    \path[pointer-left] (e-18.west) edge[in=20, out=160, looseness=2.5] (e-14.east);
    \path[pointer-right] (e-18.east) edge[bend right] (e-14.west);
    \path[pointer-parent] (e-18.north) edge[bend right] (e-8.south west);
    \path[pointer-right] (e-14.east) edge[in=-160, out=-20, looseness=2.5] (e-18.west);
    \path[pointer-left] (e-14.west) edge[bend right] (e-18.east);
    \path[pointer-child] (e-14.south) edge[bend right] (e-27.north);
    \path[pointer-parent] (e-14.north) edge[bend right] (e-8.south east);
    \path[pointer-left] (e-27.west) edge[in=20, out=160, looseness=7] (e-27.east);
    \path[pointer-right] (e-27.east) edge[in=-160, out=-20, looseness=7] (e-27.west);
    \path[pointer-parent] (e-27.north) edge[bend right] (e-14.south);

    \draw[pointer-min] (e-7.north) -- ++(0, 0.5);
  \end{tikzpicture}
  \caption{Nach \texttt{decreasekey(27, 20)} (Heap-Bedingung wird nicht verletzt,
    daher kann der Key einfach aktualisiert werden).
  }
\end{figure}

\end{document}
